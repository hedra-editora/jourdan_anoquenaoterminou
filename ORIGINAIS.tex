%\textbf{2013}%

%\textbf{O ANO QUE NÃO TERMINOU}%

%\textbf{Memórias, Entrevistas e Análises}%

%\textbf{Camila Jourdan}

\chapter*{}

\vspace*{\fill}

\epigraph{No domínio especializado da erudição, tanto como no saber
desqualificado das pessoas jazia a memória dos combates, aquela,
precisamente, que até então tinha sido mantida sob tutela. E assim se
delineou o que se poderia chamar uma genealogia, ou, antes, assim se
delinearam pesquisas genealógicas múltiplas, a um só tempo redescoberta
exata das lutas e memória bruta dos combates; e essas genealogias, como
acoplamentos desse saber erudito e desse saber das pessoas, só foram
possíveis, e inclusive só puderam ser tentadas, com uma condição: que
fosse revogada a tirania dos discursos englobantes, com suas hierarquias
e com todos os privilégios das vanguardas teóricas.}{(Michel Foucault, Aula de 7 de janeiro de 1976)\footnotemark}

\footnotetext{\versal{FOUCAULT}, Michel. \emph{Em defesa da sociedade: curso no Collége de France
  (1975-1976)}. Trad. Maria Ermantina Galvão. São Paulo: Martins
  Fontes, 1999.}

\chapter*{As personalidades distorcidas e o desrespeito aos poderes constituídos \\
\emph{Comentário à sentença de prisão de 17 de julho de 2018}}

\addcontentsline{toc}{chapter}{Comentário à sentença de prisão de 17 de julho de 2018}

No momento em que este livro estava para ser lançado, saiu a sentença do
processo dos 23. Como se poderia esperar, uma condenação dura e, inclusive, acima do previsto para as acusações em questão. As penas iam de 5 a
13 anos em regime fechado. O pedido de absolvição de cinco de
nós por parte do Ministério Público foi ignorado pelo juiz do caso,
Itabaiana. As acusações de ``formação de quadrilha'' e ``corrupção de
menores'' foram tratadas em bloco, mesmo que a última só tenha surgido na
acusação ao final do processo de julgamento e sem ampla defesa por parte
dos réus. A arbitrariedade e a ausência de materialidade das
acusações não significou nada.

A condenação é justificada em um texto extremamente político, que trata
como inaceitável que o então governador do estado do Rio de Janeiro,
Sérgio Cabral, que se encontra neste momento preso, teve seu
direito de ir e vir restringido pelo movimento ``Ocupa Cabral'', no qual
alguns dos ativistas participaram. Chama a atenção que Sérgio Cabral
esteja agora condenado pelas práticas corruptas que este mesmo movimento
denunciava, pelo que deveria
ser premiado por sua clarividência, ao invés de condenado. Certo é que, encarcerado, Sérgio Cabral tem agora seu direito de ir e vir totalmente
cassado pelo próprio Estado. Como então as manifestações que alertavam
sobre isso podem ser criminosas? No mais, também ficaram
comprovadas todas as consequências nefastas que os megaeventos, Copa e
Olimpíadas, acarretaram no estado do Rio de Janeiro, verdadeiramente
saqueado e atualmente em estado de calamidade decretado. E foi contra
tais eventos que os movimentos políticos em questão se insurgiram. O poder constituído, entretanto, não pode abrir precedentes às
manifestações populares. A resposta penal a 2013 precisa ser rígida para
que seu legado seja esquecido, ou melhor, para que o povo não ouse
jamais se levantar contra as atrocidades que o Estado realiza. A
resposta penal é rígida precisamente porque o Estado reconhece que 2013
não terminou e nada voltará a ser como antes.

Por tudo isso, precisamos lembrar ainda melhor o que foi 2013: um
movimento contra a máfia dos ônibus, cada vez mais evidente e
ativa. Sim, nós lutávamos contra o pior e mais caro
transporte público do mundo, contra seus aumentos sucessivos e abusivos,
contra seus esquemas de corrupção com o poder público, que inclusive já
foram descobertos e processados. Bem, parece que também aqui nós
tínhamos razão. Parece que também aqui nossa luta permanece atual. E
quando somos assim condenados, hoje, ainda é a máfia dos transportes
que está sendo defendida.

Mas não é só o que gritávamos. Gritávamos ``cadê o Amarildo?'', isto é,
gritávamos contra o genocídio do povo pobre e negro das favelas,
pedíamos o fim da Polícia Militar. Gritávamos, portanto, contra a/uma
política de segurança assassina. E quem poderia dizer que
estávamos errados? Ainda hoje poderíamos gritar (e de fato gritamos) por
tantos outros: pelos mortos diários na intervenção militar no Rio de Janeiro e no
acirramento da suposta guerra às drogas, que é, de fato, guerra ao
povo favelado. Hoje, quando somos condenados,
é também a política mentirosa e genocida de segurança militar que
está sendo defendida.

Mas nós também gritávamos por mais coisas, nós compusemos a luta da
educação, a tão atual resistência dos professores no Rio de Janeiro. Há menos de um mês os professores da rede municipal foram brutalmente
agredidos pela Polícia Militar no centro do Rio, e uma professora levou
um tiro de bala de borracha. Em 2013, estávamos compondo a ocupação da
greve de professores, uma guerra em curso, por condições de trabalho,
contra o fechamento das escolas e a reforma do ensino médio. Nossa
resistência foi também contra o espancamento de professores nos
protestos daquele período, foi pelo seu direito a protestar. E é visível nos dias atuais o que isso ainda
significa, talvez o motivo pelo qual continuam tentando nos calar.

Sim, nós também gritamos contra as remoções, e não cansa lembrar que a
aldeia Maracanã teria se tornado um estacionamento de estádio de futebol
não fosse 2013 --- luta que ainda está em curso, com o espaço da aldeia ocupado. Nós questionamos o sistema capitalista,
repudiando o lucro dos banqueiros. E, ao lado disso, nós denunciamos
o suposto sistema eleitoral democrático como sendo uma grande fraude,
uma grande farsa. E talvez nunca tenha sido tão evidente a falência
desse sistema como agora, quando mais uma eleição espetacular forjada se
aproxima.

Não há dúvida de que nos condenar é um golpe numa luta bem atual. Nós
não esquecemos, tal como o Estado não esqueceu, simplesmente porque não
acabou. Nossa condenação é um ataque em todas essas guerras ainda
em curso, mas nossa luta permanece presente, nossas pautas estão
resistindo. Não pense que só se tratou da condenação de jovens baderneiros e delinquentes, que não tem relação nenhuma com você. O que se pretende é enterrar 2013, por isso precisam nos condenar, mas nossa
prisão não apaga nossas lutas e nossa história permanecerá, em qualquer
caso, viva.

Foram quase três anos aguardando uma sentença que não vinha, cumprindo
medidas restritivas que chegaram a suspender nosso direito de participar
de manifestações políticas e que permanecem durante o tempo de recurso
bloqueando nosso direito de ir e vir para além da comarca. A sentença
ocorre em um momento completamente significativo: pós"-Copa do Mundo de
2018 e antes do processo eleitoral --- ou seja, exatamente o mesmo
contexto no qual fomos inicialmente presos e processados. Durante todo
esse tempo, nossas vidas foram totalmente expostas, nossas atividades
profissionais foram atrapalhadas ou suspensas, nossas vidas pessoais
foram viradas de cabeça para baixo. Mas isso não basta ao Estado. A
sentença requenta mesmo aspectos da criminologia fundada na construção
de um sujeito criminoso, já que nossas penas são qualificadas com
afirmações como ``possuir uma personalidade distorcida'' e ``voltada ao
desrespeito aos poderes constituídos''. É a figura,
cunhada no final do século \versal{XIX}, do anarquista como um delinquente social com traços
psicopatas. Agora a mídia ainda repete mentiras a nosso respeito, como a
suposição, jamais justificada por qualquer evidência razoável, de que
pretendíamos explodir o Maracanã na final da Copa do Mundo de 2014.
Afirmação que só não consegue ser mais patética do que a alusão a
Bakunin como um suspeito no inquérito que levou à nossa prisão.

Falamos muito em Estado de exceção, mas deixamos de ressaltar
o que ele significa, quando se trata precisamente disso: a
ausência de separação evidente entre o âmbito jurídico e a política, a
fabricação de crises como estados de emergência permanentes que rompem
com a distinção entre poderes, permitindo totalitarismos evidentes no
cerne das sociedades pretensamente democráticas. O seu alvo fundamental
é qualquer potencial insurgência. Nesse sentido, 2013 é ainda o grande
alvo do ``Estado de exceção'' no qual estamos inseridos, e por isso tal
sentença absurda não pode ser outra coisa senão sua confirmação.
Adicionalmente, a única resposta possível ao Estado de exceção pertence
ao âmbito da resistência. Esta publicação não pertence fundamentalmente
à nossa defesa jurídica, mas essas fronteiras já foram borradas há
tempo e, por isso, a defesa política de 2013 e das suas práticas passa
pelo \emph{front} discursivo e compreende, acima de tudo, a luta social
agora em curso. A liberdade precisa ser conquistada permanentemente.

\chapter{Apresentação}

Esse volume reúne memórias, relatos, desabafos, entrevistas, análises e
comunicações públicas direta ou indiretamente relacionadas aos eventos
políticos de 2013 e à decorrente perseguição política que sofri. São textos escritos entre 2014 e 2017 e organizados aqui da melhor forma que me foi possível, de modo não linear, mas nem por isso arbitrário. Têm em comum a temática
anarquista, a insurgência discursiva e o registro histórico de um dos
momentos mais relevantes de nossa história recente. Ao mesmo tempo é
pessoal e geral; prático e teórico; sintético e analítico; uno e
múltiplo. Não é exaustivo e não se pretende último, mas é um recorte
coerente das potencialidades dos últimos acontecimentos que marcaram nossa política e nossas vidas.

Por que escrever e publicar, cinco anos depois, sobre 2013 e seus consequentes
processos políticos? É preciso compreender o que significou 2013, é preciso entender o que estava em jogo naquele momento, pois é algo que diz muito sobre o que ainda está em jogo. Há uma disputa
discursiva em andamento sobre seu real significado, e muito do que vivemos hoje está sob o efeito das inquietações e possibilidades abertas pelo inesperado levante popular que tomou o
Brasil na época. O processo dos 23 permanece sem sentença, as
perseguições políticas continuam em andamento, e a crise da representação
nunca esteve tão forte. É preciso responder adequadamente ao que 2013 nos
trouxe, e nada melhor do que 2018 --- ano em que a disputa eleitoral
requenta espetacularmente as reações a 2013 --- para retomarmos, através
daqueles que viveram intensamente os acontecimentos, a mensagem que
2013 nos legou.

É uma publicação que se insere, portanto, no âmbito da disputa de
discursos sobre o que houve em 2013. E é fundamental ressaltar a
importância de contarmos nossa própria história. O que se justifica não
apenas como atitude teórica, acadêmica, ou pelo compromisso com o
que será ensinado às gerações seguintes sobre aquilo que vivemos, mas, principalmente, pelos efeitos práticos dessa disputa ideológica.
Pois o que 2013 fez foi apontar para uma outra realidade possível,
mostrar que, de fato, podemos tomar e parar a cidade. Aprendemos a
resistir, aprendemos a não apanharmos calados, fizemos o Estado e seus
agentes recuarem. Isso é extremamente forte, sobretudo em uma sociedade
que jamais viveu uma revolução popular. O governo do Estado teve que
sentar em roda no chão da aldeia Maracanã para negociar com indígenas,
punks e militantes anarquistas. Vimos o 1\% que controla os meios de
produção e os aparelhos ideológicos do Estado tremer de medo das ruas
tomadas, obrigamos as emissoras de \versal{TV} a mudarem seus discursos e sua
programação. Essa possibilidade de uma outra realidade não pode ser
apagada de nossas mentes e corações.

 Trata"-se, portanto, de disputar o legado de 2013, pois seu significado para as
próximas gerações não é algo certo. E o
aprendizado para as próximas lutas, bem como para o
crescimento político geral da nossa sociedade, é algo que
nos cabe conquistar. Não podemos deixar que contem nossa história por
nós, que digam que as ruas foram tomadas pela direita autoritária e pela
classe média alienada para enfraquecer um suposto governo de
esquerda; que depois as ruas esvaziaram espontaneamente, sem nenhuma
repressão ou perseguição política; que as ações diretas contra
símbolos do capitalismo eram pagas por entidades partidárias (ou
fundações internacionais) e realizadas por uma minoria infiltrada nos
atos; que a recusa às representações partidárias tradicionais era um
elemento fascista; ou que a única pessoa a morrer no contexto das
manifestações foi o jornalista Santiago Ilídio Andrade, da \emph{Rede
Bandeirantes de Televisão}.

É importante lembrar e resgatar o que 2013 tinha de inegociável,
invendável, a fenda que abriu no sistema apontando para um
outro modo de vida, pois é isso que ele tem de poderoso. Não podemos
deixar que transformem nossa história em mais um produto, ou em um
espetáculo mal acabado de nós mesmos. Esta não é uma tarefa fácil, pois
não detemos os aparelhos ideológicos do Estado, somos as vozes
marginais, somos o contradiscurso e seguimos insurrectos.

\begin{bottompar}
\raggedleft
\emph{Agradeço a Acácio Augusto pela leitura atenta e sugestões; a
Wallace Moraes pela influência e incentivo e, especialmente, a Jose
Freitas, Duda Castro, Elisa Quadros e Rebeca de Souza, que me permitiram
compartilhar uma parte de nossa história.}

\smallskip

\emph{Este livro é dedicado aos filhos de 2013 e a todas as pessoas que
não se calam diante do intolerável.}
\end{bottompar}

\part{MEMÓRIAS}

\chapter*{Minha prisão\\
\emph{Véspera da final da Copa do Mundo\\
12 de junho de 2014}}

\addcontentsline{toc}{chapter}{Minha prisão}

Você sabe o tempo todo que pode acontecer, mas não acredita quando ocorre.
Um belo dia os agentes armados do Estado invadem a sua casa. Eram seis
horas da manhã e minha porta foi derrubada pela Polícia Civil, pela
\versal{CORE}, dois homens e uma mulher fortemente armados. Eu estava dormindo,
sem roupa. Meu companheiro tentou segurar a porta do quarto para que eu
me vestisse. Ele foi algemado.

``Eu não estou resistindo.''

Mas eu queria resistir\ldots{}

Finalmente vestida, minha prisão é declarada.

Meus gatos fogem, eles podem fugir. Eles queriam meu celular, eu só
fazia questão dos gatos. Não entendia direito que já tinha
perdido ambos.

Me mostraram o mandado de prisão temporária, mas eu ainda não tinha
entendido como aquilo podia ser. Eles não tinham nada contra mim.

``Vocês não podem me levar assim, por nada, fascistas!''

``Melhor você ficar calma.''

Disseram que eu podia ser perigosa e por isso permaneciam me apontando
armas. Eles acreditam no Estado que servem. No banheiro, tento acessar
meu celular escondido, para avisar alguém que estou sendo presa.

``Abre esta porta, senão vamos derrubar.''

Nervosa, não consigo enviar a mensagem. Outra porta arrombada. O policial
entra no banheiro e joga longe meu celular com uma porrada.

``Está vendo como você é perigosa?''

Eu ainda queria os gatos\ldots{} Nenhum vizinho aparecia. Era preciso fechar a
porta da minha casa, mas ela havia sido derrubada. Até hoje ainda me
pergunto: ``por que simplesmente não saímos?''

Mas nesse momento a policial abre a porta do meu escritório e começa a
revista\ldots{}

``Liga para o \versal{DP} e pede outro mandado, pede apreensão.''

Encontram algo, gasolina. Tomam gosto pela coisa, tudo começa a ser
revirado.

Não posso mais simplesmente sair, não posso dizer que eles não têm
permissão para mexer nas minhas coisas, não tenho mais coisas minhas.

Algo é encontrado. Eles comemoram: ``é uma bomba!''

``Esta bomba é sua?''

``Você que está dizendo que é uma bomba.''

``É sua?''

Silêncio.

Não importa mais.

``Vamos levar os escudos?''

``Não, isto é muito pesado, e eles têm direito a se defender.''

Ufa, ainda temos direitos, incrível.

A policial ainda vai para o meu quarto, mas já estava bom. ``Chega, não
tem nada aqui''. Eles tinham o suficiente para tentar me encarcerar por
cinco anos e eu nem sabia disso. Seguimos para a cidade da polícia.

Toda a operação não durou nem trinta minutos e desconstruiu minha vida
inteira, eu nunca mais seria a mesma.

Em poucos minutos eu não era mais nada, não tinha mais direito à minha
casa, nem à minha filha, nem ao meu emprego, nem acesso a qualquer meio
de comunicação. Estava presa, e isso passou a ser repetido
exaustivamente pelos policiais na minha cabeça, como para me
convencer da minha nova situação.

Mas será que era nova? Pois, se eu fosse livre, poderia ter
passado tão facilmente de um estado ao outro? Como eu havia tão
rapidamente deixado de ser uma pessoa com direitos e havia me tornado um
objeto de propriedade do Estado? Não se passa de livre para não"-livre tão
facilmente, por mandado. Livre e não"-livre não são modos de um mesmo
Ser. Quais condições já estavam dadas para que aquele ato de fala
pudesse ser bem sucedido?

A minha liberdade, eu descobriria, não poderia ser tirada pelo
encarceramento. E decretar minha prisão apenas tornou explícito
como eu já era, assim como todos mais nessa sociedade, em grande medida
uma prisioneira.

Entender isso, o quanto nossa liberdade é falsa e, ao mesmo tempo, como há outra noção de liberdade que resiste e que não nos pode
ser arrancada, é talvez a maior vingança de todo prisioneiro contra
aquele que o prende. Prender alguém é dar"-lhe a possibilidade de
descobrir essa verdade. E a verdade liberta. Doce dialética que coloca
no aprisionamento a condição da libertação.

Mas naquele momento eu não conseguia pensar em nada disso. Fomos
colocados em solitárias, mínimas, acho que três metros por dois, todas
brancas. A impressão é de estar sendo enterrado vivo:
a porta não tem grades, apenas uma pequena janelinha dá para o exterior,
o que se vê é uma pequena fresta distante. Mas você pode ser visto lá
dentro, facilmente, por qualquer um que passe pelo corredor. Há apenas uma
fossa no chão para as necessidades. Ela é estrategicamente localizada em
frente à pequena janelinha na porta que dá para o corredor, a uma
distância que não é possível para quem usa a fossa saber se é ou não
observado enquanto faz suas necessidades, dando a impressão de se
estar o tempo todo exposto naquela posição. Por alguns
instantes penso no princípio da vigilância, e no efeito esperado em poder
observar alguém sem ser observado. Naquele momento Foucault e suas
análises das tecnologias prisionais me pareceram particularmente
corretas. Hoje, após passar por alguns procedimentos jurídicos e médicos
adicionais, percebo que os poderes médico e judiciário têm se apropriado
de um modo mais ostensivo dessas tecnologias do que o próprio cárcere em
sua simplicidade branca. Pelo menos a cela era limpa. Na verdade, a
Polinter havia sido reformada recentemente para receber prisioneiros da
classe média, corruptos e ativistas. Acima da fossa havia um cano pelo qual descia água, ou melhor, deveria descer se você solicitasse ao
carcereiro. Seria a única água disponível para beber, tomar banho e
dar descarga. Em breve eu descobriria que, acionada a descarga, a cela
inteira era alagada, de tal modo que era melhor conviver com o cheiro de
suas próprias fezes do que solicitar ao carcereiro para se livrar delas
com a água que descia do teto. Não deixa de ser digna de nota a função
que o Estado assume finalmente em nossas vidas: levar nossa merda. Na
prisão, não se pode cagar sem ter que pedir ao agente do Estado
para dar descarga, tal como pedem as crianças para os pais até certa
idade. ``Acabei, Estado, pode levar''. Mas o Estado não leva, ele é
incompetente em se livrar da merda.

Até aquele momento não nos haviam deixado falar com nossos advogados.
Isso só foi permitido depois que a operação \emph{Firewall} terminou,
para evitar que outros possíveis presos fossem avisados. Quando
finalmente nos chamaram para depor, agiram como se já soubessem que não
falaríamos nada. ``Só vai falar em juízo, né?'' Eu via a quantidade de
companheiros que estavam lá na mesma situação, parecia que todos os
militantes da cidade haviam sido presos. Finalmente tive direito a um
telefonema. Os advogados ainda estavam a caminho. Não podíamos
esperá"-los ali, tínhamos que voltar para a cela. Me deixaram ligar
também para minha mãe, que estava com minha filha.

``Camila, a polícia esteve aqui, reviraram tudo.''

``Mãe, eu estou presa.''

Nada mais é dito. Só depois soube que ela recebeu visita policial antes
de mim e que foi ameaçada para que não me avisasse por telefone que
estavam indo me prender.

Aquela foi a pior noite de todas, pois passamos na solitária da
Polinter. Parece desnecessário dizer que foi algo totalmente arbitrário,
aquela cela não é para se permanecer por mais de algumas horas. Além
disso, não existem carcereiras mulheres na Polinter. Mas, para toda uma operação arbitrária, esse é um detalhe sem importância.

Mais pessoas presas chegavam, ainda não sabíamos quantos eram no
total. Na minha solitária foram colocadas Rebeca Souza e Moa. Uma
felicidade sem nome encontrá"-las. Horrível é ficar sozinha na cela
branca, melhor não ter espaço nenhum e estar com rostos conhecidos, se
solidarizar mutuamente, se dar força, fazer rir da situação. Podíamos
fumar, mas não podíamos acender nosso próprio cigarro, então o
carcereiro tinha que ter boa vontade para vir acender. Ele não tinha
esta boa vontade, fingia que não ouvia quando chamávamos. Aliás, coisa
que não se pode é chamar o carcereiro de carcereiro, ofensa terrível,
tínhamos que chamá"-lo de ``senhor funcionário'', mostrando respeito. Era
necessário manter uma relação cordial, dependíamos dele para tudo.
Naquele momento o local estava cheio de familiares e advogados, mas não
sabíamos como seria a noite.

Subitamente, alguém começa a cantar de dentro de uma das solitárias:
``presos políticos, liberdade já, lutar não é crime, vocês vão nos
pagar''. As demais celas acompanharam, ouvíamos as vozes conhecidas,
assim sabíamos mais ou menos quem estava ali. Em seguida, a
Internacional socialista começou a ser assobiada, primeiro de maneira
tímida, em seguida como em um grande coro, seguida por ``\emph{A las
barricadas}'' e outras palavras de ordem. Não é possível expressar a força
que isso dá em um momento como esse, faz lembrar que há um sentido em
estar ali, que nossa prisão não foi a última nem a primeira, que se
trata de uma luta muito antiga, de todos que se rebelaram em tantas
épocas, em tantas ditaduras e apartheids, declarados ou não.

Somente no dia seguinte fomos transferidos para Bangu.

\chapter{Contramanifesto aberto pela legitimidade das manifestações populares\footnote{Manifesto assinado por vários intelectuais e publicado nos meios digitais dois meses antes da nossa prisão.}}

Desde o ano passado, temos acompanhado uma série de revoltas populares,
protestos e manifestações por todo o país. Aquilo que inicialmente seria
uma insatisfação com os altos preços dos transportes públicos tomou a
forma de exigência generalizada por participação política, demonstrando
uma crise dos fundamentos da democracia representativa, marcada pela
ausência de participação popular efetiva nos rumos da vida pública, e
uma crítica profunda à legitimidade e suposta fatalidade do sistema
capitalista.

As tentativas de direção e controle dos protestos pelas diversas forças
partidárias institucionais, tanto da direita quanto pelos partidos da
esquerda tradicional, falharam sucessivamente. Também foi insuficiente
até agora a estratégia do governo de criminalizar o movimento, assim como
as manipulações midiáticas que procuraram desqualificar a revolta
popular sob um discurso repetitivo de que ``tudo não passaria de uma
minoria de vândalos infiltrados'' e de ``baderneiros inconsequentes''.
Apesar disso, as manifestações continuam, bem como a crise de
representatividade aludida, que agora reverbera para o fato de as bases
dos trabalhadores sustentarem greves em contrário à orientação das suas
direções patronais, dominadas em sua maioria por partidos políticos mais
preocupados com as próximas eleições do que com a luta da categoria. São
exemplos claros as greves com repercussão internacional, tocadas pela
base das categorias, como a dos garis e dos rodoviários.

Diante da proximidade da Copa do Mundo e da perplexidade popular com os
gastos exorbitantes com estádios de futebol, assistimos absurdos
naturalizados pelos grandes oligopólios de comunicação de massa como:

\begin{enumerate}
\item O cerco de comunidades inteiras, inclusive com uso do exército, e a
continuidade da violência policial nas favelas e periferias,
significando na prática a multiplicação de territórios em permanente
\emph{Estado de exceção}, através das chamadas Unidades de Polícia
Pacificadoras;
\item A criminalização de manifestantes, como no caso de
estudantes do Rio de Janeiro, São Paulo, e, recentemente, Goiás que
foram apanhados em casa e postos em prisões, cerceando suas liberdades e
suas vozes críticas.
\end{enumerate}

Por fim, os protestos tomaram a forma de denúncias do extermínio de
negros e pobres nas favelas e adotaram como palavra de ordem sintética a
bandeira: ``se não tiver direitos, não vai ter Copa''. Na medida em que as
manifestações geram prejuízos sistemáticos às grandes corporações e,
principalmente, colocam em questão este modo de vida pelo qual as elites
e a maioria dos políticos se privilegiam e locupletam, alguns
intelectuais vieram à público recentemente questionar sua legitimidade
democrática e seu sentido histórico, evidentemente ajudando a criar um
espaço de criminalização dos protestos.

Nós, estudiosos da sociedade, professores e professoras, nos sentimos
então compelidos pela obrigação de responder este manifesto com um
contramanifesto, de lembrar o inalienável direito dos povos à rebelião,
que, inclusive, está na base do próprio nascimento da democracia. Se o
poder emana do povo, este tem todo direito de reivindicá"-lo e de colocar
em questão os governantes e os sistemas que não correspondam aos seus
anseios e reivindicações. Nenhuma grande revolução ou transformação social,
em toda a história, foi conseguida sem levantes, revoltas ou rebeliões. Nenhuma
ruptura com sistemas vigentes aconteceu com respeito incondicional às
instituições que se pretendia justamente combater. Maiores do que os
prejuízos econômicos gerados pelas manifestações são as vidas perdidas
por este modo de organização societal cujo sistema econômico exclui
milhares de pessoas não apenas da participação política, mas da mera
possibilidade de sobrevivência.

O incômodo de alguns com relação às
manifestações, notadamente quando implicam no fechamento de
ruas/avenidas, não é nada quando comparado às opressões sistemáticas
pelas quais passa grande parte da população; é, também, consequência do
flagrante distanciamento de uma pequena parcela da sociedade, favorecida
e privilegiada, que não consegue (ou não quer) sair da sua zona de
conforto.

Por tudo isso, consideramos completamente legítimas as manifestações
populares e os métodos de autodefesa usados pela população oprimida.
São plenamente justificados os meios de luta que incomodam quando o
objetivo é justamente tornar evidente o incômodo sofrido pelos
excluídos. São necessários e úteis para a melhora da sociedade estes
protestos que tornam evidentes para todos a denúncia de uma Copa do
Mundo construída pelo uso indevido do dinheiro público, sem qualquer
consulta popular, atropelando literalmente diversas moradias populares e
em detrimento do investimento em saúde, educação, moradia, saneamento
básico, etc.

\chapter{Bangu}

Dormimos eu, Moa e Rebeca quase empilhadas na solitária da Polinter.
Durante a madrugada acordei várias vezes, e sempre pensava ``ainda
estou aqui, não foi um pesadelo''. Partimos bem cedo algemados em duplas
para Bangu, seis homens e seis mulheres. Na saída, apesar das ameaças dos
guardas, puxamos palavras de ordem. Ninguém sabia o que ocorreria. Fomos
entulhados na viatura da \versal{SEAP} (Secretaria da Administração
Penitenciária), que corria propositalmente, para enjoarmos. Estávamos
algemados uns nos outros em um espaço que mal cabiam três pessoas.
Apesar de tudo, ninguém vomitou. Polícia gosta de rir da cara de preso,
é uma espécie de vingança pela função que exercem: para se manter em
uma função como esta, sadismo é fundamental, assim como uma grande
dose de ignorância acerca do próprio papel na sociedade em que está inserido.

Os rapazes desceram primeiro, ouvindo ameaças de que sofreriam violência
porque, segundo os Policiais Civis, ``aqui eles agem um pouco diferente
da gente, melhor vocês olharem pra baixo o tempo todo e só falarem se
forem perguntados''. Depois saberíamos que agir um pouco diferente
consiste em espancar e torturar de acordo com a própria vontade.
Seguimos para a penitenciária feminina.

Nosso maior receio era de apanhar e da revista íntima. Sininho parecia
uma professora nos dizendo como tudo funcionava, pois já havia sido
presa ali. Logo na chegada vimos que seríamos tratadas de forma diferente das
outras presas. O procedimento da revista íntima não nos constrangeu mais
do que à própria funcionária que a realizou. A chefe de turno nos
recebeu dizendo que éramos ``outro nível'', e ficou claro que ela se
referia a uma diferença de classe. Quase cem por cento das detentas são
negras, isso não é um exagero. Nós fomos colocadas isoladas. Embora as
outras presas não pudessem falar conosco, faziam questão de nos acenar
de longe e demonstrar apoio. Muitas jogavam bilhetes escondidos, pedindo
ajuda, denunciando torturas, situação que permaneceu até o dia em
que saímos.

A chefe de turno nos explicou as regras: só andar com as mãos para trás
e cabeça baixa nas dependências do presídio, chamar as carcereiras de
``senhora funcionária''\ldots{} Ocorriam dois ``conferes'' ao dia, um às 7h e
outro às 17h, quando, a funcionária dizendo nosso primeiro nome,
deveríamos completar com o sobrenome e dar um passo para frente virando
para o outro lado (isto é, ficando de costas para onde olhávamos antes).
Quando a funcionária chegasse na frente da cela já deveríamos estar formadas
para o confere, e não estar constituía falta grave e podia nos levar
para a solitária (chamada de tranca). Além disso, era nosso dever cuidar
da limpeza e arrumação da cela. Não teríamos contato com nenhuma outra
presa nem banho de sol por pelo menos sete dias. Esperávamos, claro, sair
antes.

O mais assustador no primeiro momento era a quantidade de
mosquitos. Não adiantava cobrir a cabeça, eles entravam por entre as roupas,
pelo meio da coberta, invadiam em nuvens cada espaço. Em bando ao cair
da tarde, pareciam treinados para tornar a permanência na prisão um inferno
ainda maior. Coberta, aliás, é obviamente luxo ali, e, como estávamos no
inverno, um artigo cobiçado por todas e conseguido por poucas: tal
como os mosquitos, o frio também é usado como punição. Estava claro para
nós que apenas recebemos cobertas por sermos ``outro nível'', como havia
dito a carcereira. Afinal, em breve alguma comissão de direito humanos
procuraria saber como estávamos sendo tratadas. Mesmo assim, éramos
oito e recebemos tão somente três cobertores. Com exceção de uma de nós,
todas concordaram em coletivizá-los, e o que ficou faltando seria
coletivizado no dia seguinte pelo grupo em decisão tirada na assembleia
da cela. Quando se tem pouco, não há espaço para individualismo. Ou
melhor, de modo geral, o individualismo só pode ser defendido por quem
tem privilégios. Dormimos mais uma vez juntas, eu, Moa, Rebeca Souza,
Elisa Quadros, Duda Castro e Jose Freitas, aproveitando as cobertas
coletivas e com camisetas na cara para se proteger dos mosquitos. Me
lembro de alguém ter dito que era por ``fidelidade à tática''. Antes de
dormir, fizemos a primeira assembleia da cela e escrevemos nosso
manifesto de prisão.

\chapter{Carta pública das militantes presas na \emph{Operação 12
julho}\footnote{Carta escrita na cela coletivamente pelas ativistas
  presas na operação de 12 de julho de 2014.}}

Fomos presas no dia 12 de julho de 2014 e uma pergunta ainda permanece:
qual a nossa acusação? Somos indignadas, engraçadas, libertárias,
professoras, resistentes, corajosas. Somos produtoras, garçonetes,
trabalhadoras sem carteira assinada, advogadas, mídia"-ativistas,
estudantes. Somos também mães, filhas, tias, irmãs, primas, netas. Somos
amigas, amadas, amantes. Somos mulheres e somos presas políticas.

Numa ação arbitrária, com um processo forjado, provas plantadas, menores
presos, violências, ameaças, fomos jogadas no cárcere com outras
exploradas e excluídas como nós. A ressocialização desse sistema está
presente apenas na estampa de nossos uniformes, o isolamento e
invisibilidade não ressocializam ninguém. E mesmo com todo assédio, com
as quatro a seis horas em transportes precários, com salários
insuficientes para pagar o alto custo de vida na nossa cidade, como
professoras sem condições dignas de trabalho e, muitas vezes, sem
salário, como negras discriminadas, não nos calamos perante o terrorismo
do Estado, pois tudo isso é uma forma de dizer que é melhor ficarmos
caladas e submissas.

É preciso denunciar ainda que a operação do dia 12 de julho foi um
grande conluio do Estado, com mandados expedidos sem nenhuma prova
concreta e executados sem que nenhum crime tivesse sido cometido, apenas
com o intuito de impedir que houvesse manifestação na final da Copa do
Mundo da \versal{FIFA}. Para garantir o espetáculo mundial e o lucro de poucos,
greves foram criminalizadas, alguns perderam seus empregos, muitos
perderam suas casas, nós fomos presas e tantos outros perderam (e ainda
perderão) suas vidas.

O que ocorre no país hoje é uma grande perseguição política. Há anos as
pessoas têm ido às ruas com suas reinvindicações diante da percepção das
contradições desta sociedade doente na qual vivemos, sofrendo sequestros
relâmpagos, infiltração de policiais, quebra de sigilos telefônicos,
processos administrativos, violência policial (inclusive com óbitos),
tiros com armas letais, ameaças diretas e indiretas, cassação de
salários; e agora, como em outros momentos da história, estão sendo
encarceradas por suas posições políticas e pelo crescimento do movimento, ameaças
concretas à ordem dominante. Assim, movimentos sociais e
políticos estão sendo transformados em associações criminosas.

Militamos em vários espaços distintos. Somos companheiras de luta sim. O
que nos une é a luta por uma sociedade mais justa, mesmo que muitas de
nós só tenham se conhecido aqui, atrás das grades. Paralelamente a
isso, a grande mídia cria um espetáculo, fabrica líderes fáceis de matar
e tenta calar as nossas vozes.

Tiraram"-nos a única coisa que nos dizem que temos: nossa liberdade fora
desses muros. Nossa liberdade de ir e vir, mas só nos lugares que nos
permitem. De comprar o que não precisamos ter. Liberdade de ser
exploradas, caladas, submissas, discriminadas, assediadas, liberdade de
ter a chave de nossas próprias celas. Declaramos que a liberdade que
queremos é maior do que esta, é a liberdade de saber que nós não moramos
na rua porque ninguém mora na rua. Liberdade para nos alimentarmos por
sabermos que ninguém mais passa fome. Liberdade de amar a quem quisermos
porque somos livres e só seremos livres quando ninguém mais for escravo.

Evocamos a todas e todos para lutarem nas ruas, para criarem cada vez
mais espaços de resistência e que nossa perseguição, sofrimento e
cárcere não sejam em vão. Chamamos também a todas e todos que ocupam
lugares privilegiados nessa sociedade extremamente desigual --- juristas,
intelectuais, jornalistas, sindicalistas, organizações de direitos
humanos, formadores de opinião, a comunidade da \versal{UERJ}, etc. --- a tomarem
um posicionamento público. É sempre bom lembrar que quem se cala diante
das injustiças contribui para a manutenção desta situação. Gerar medo de
falar é um modo sistêmico de nos tornar todos cúmplices.

A luta segue, voltaremos para as ruas e ninguém ficará pra trás!

\medskip

\begin{flushright}
\emph{Penitenciária feminina de Bangu, Pavilhão 8, Anexo 2}
\end{flushright}

\chapter{Um chinelo do Estado}

Era o nosso primeiro dia inteiro em Bangu. Nosso segundo confere do dia.
A cela era dividida em comarcas, que são como camas beliches totalmente
de pedra. Escura e sem janelas, como se espera de uma cela, úmida e com
mosquitos. A chefe de turno, posteriormente batizada por nós de Macaco
Louco, por sua similaridade física e de conduta com a personagem da
animação ``Meninas superpoderosas'', veio pessoalmente supervisionar nosso
confere.

Existe algo na estética das agentes penitenciárias que precisa
ser salientado. Todas, sem exceção, se arrumam com longos cabelos soltos
com chapinha, unhas grandes pintadas e sapato de salto. A maioria usa
também maquiagem forte e os cabelos pintados em tons claros. Essa
estética objetiva reproduzir a estética dominante e exercer um tipo
de poder pela aparência sobre mulheres negras, uniformizadas com short,
camiseta e chinelo. O efeito visual faz lembrar a época da
escravidão: são mulheres brancas em maioria oprimindo mulheres negras em
maioria; mulheres que mesmo quando negras reproduzem a estética branca o
quanto podem; que ganham muito bem para exercer aquele papel
opressivo e que aprendem a gostar disso. Esta agente penitenciária não
usava apenas salto, mas também uma bota até o meio da perna, um relógio enorme
dourado e vários cordões. Ela parecia bater propositalmente o pé no chão
quando andava, como que para impor algum tipo de intimidação.

Até então, não havíamos tido nenhuma visita de advogados nem nenhuma
notícia de fora, a impressão era que havíamos sido esquecidas. Logo cedo
vieram nos avisar que não havia tido ato na final da Copa, o que somente
depois descobrimos que não fora verdade. Todo o tempo o objetivo das
condutas das agentes penitenciárias parecia ser quebrar a nossa moral. E
devo dizer que estavam sendo relativamente bem sucedidas. Estávamos
enfileiradas para o confere quando a chefe de turno percebe que uma de
nós não está com o chinelo que havia recebido ao dar entrada na cadeia.
Subitamente, é como se Duda tivesse cometido um delito grave, e a
chefe de turno começa a gritar em sua direção.

``Vai procurar o chinelo que você recebeu, agora! O que vocês estão
pensando que é isso aqui, uma colônia de férias?''

Algumas de nós tentam ir ajudar, mas somos
impedidas pelas carcereiras. Duda era a mais nova entre nós.

``Ela procura sozinha, o chinelo foi dado pra ela.''

``Se você não encontrar este chinelo você não sai daqui nunca mais.''

``Este chinelo não é seu, este chinelo é do Estado.''

Nunca um chinelo pareceu tão importante. Duda ficou uns vinte minutos
nervosa, tremendo, procurando o chinelo, enquanto Macaco Louco gritava
ofensas e ameaças a todas nós e principalmente a ela. Ela se retirou
ameaçando voltar em meia hora e levar Duda para a solitária se o chinelo não tivesse
aparecido. Quando ela saiu nós abraçamos
nossa companheira e procuramos o chinelo, que foi encontrado em cima da
última comarca, no canto próximo à parede.

Passamos frio, fome, fomos atacadas por mosquitos, sofremos tortura
psicológica, presenciamos torturas muito piores. Nada do que passamos
pode ser comparado ao que passam as demais presas do Complexo
Penitenciário de Gericinó\footnote{O Complexo Penitenciário de Gericinó
  foi criado em 1987, quando o então governador do Rio de Janeiro,
  Moreira Franco, decidiu criar ali, na região agrícola de Bangu, um
  presídio de segurança máxima. Por se localizar em uma área bastante
  afastada da área urbana, o presídio propicia isolamento e
  invisibilidade dos detentos, contribuindo ainda mais para o exílio e a
  arbitrariedade que já constituem, em geral, nosso sistema
  penitenciário.}, lá esquecidas, isoladas, as verdadeiras presas
políticas de uma sociedade desigual. Estar presa é ser lembrada a todo o
tempo que você não tem direitos, que você não é nada, ``você está presa''
significa ``a partir de agora você não é mais humana'', você é escória,
contra você tudo é permitido, tudo é justificado. ``Mãos pra trás e
cabeça baixa, não encare seu opressor de frente, incorpore a submissão''.
O sistema prisonal leva ao extremo aquilo que o sistema pedagógico já
legitima, uma pedagogia do castigo e da culpa, da punição e da dor, da
exclusão, da segregação e da humilhação. Por isso que, antes de tudo, toda
a sociedade é cúmplice da existência de lugares como aquele. A
única atividade permitida para as detentas no presídio é assistir a cultos
pentecostais, nos quais é ensinado que a única esperança para elas é a
salvação divina; nos quais são mais culpabilizadas por estar ali e
ensinadas a se conformar, diante de um Deus que pune, um Deus que é a
imagem e semelhança do Estado patriarcal.

\chapter{A batalha da \versal{ALERJ}}

Existem muitos episódios que mudam a vida de uma pessoa, e a
batalha da \versal{ALERJ} foi um desses na minha. Eu queria
narrar este dia com detalhes para ver se
mais alguém descobre algo pelo que vale a pena viver. Mas sabemos que a
memória é seletiva, que os detalhes se perdem, e que muito não poderá
ser dito.

Não sei de onde vieram todas aquelas pessoas e toda aquela insatisfação.
Desde cedo, milhares tomaram as ruas aos gritos de ``acabou amor''. Os
atos, que cresciam a cada nova manifestação, estavam agora em sua fase
mais massificada. As pessoas queriam enfrentamento, e não me refiro aos
``militantes combativos'' de sempre, mas das pessoas comuns. O
discurso pacificador propagado pelos meios de comunicação já não
encontrava ouvidos nem ecos. E o mais importante: a favela estava na
rua, a periferia havia entrado em cena. O burburinho de rebelião tomava a
cidade, trabalhadores desciam de seus escritórios, todos os explorados
insatisfeitos formavam uma grande massa de pessoas com camisas na cara.
Não havia o black bloc como um grupo combinado, havia a
aprendizagem prática de uma tática que vinha funcionando até então e que
encantava a população comum como uma ruptura no seu cotidiano de
opressão.

Nenhum líder, todos iguais mascarados nas ruas. Começaram os cochichos,
uma pessoa pra outra, não sei de onde surgiu, o que me lembro é de uma
massa negra que se procurava e dizia ``vamos tomar a \versal{ALERJ}'', ``vamos pra
\versal{ALERJ} hoje, passa adiante''. Os primeiros que avançaram encontraram
alguma resistência, mas os guardinhas abandonados lá não tinham nenhuma
chance, fugiram da população que atirava pedras e paus. O povo havia
ganhado do braço armado do Estado e, como no final de um Saltimbancos, concluiam:
``eles correram, todos juntos somos fortes''.

A partir deste momento o povo não tinha mais medo e avançava em euforia.
Todas as ruas em paralelo à Rio Branco foram tomadas por barricadas e as
lojas foram invadidas e saqueadas. Chocolates caros, joias, roupas
levadas e partilhadas por aqueles que não têm acesso a este tipo de
consumo. Sim, chocolate: a batalha da \versal{ALERJ} teve sabor de chocolate com
cheiro de gás lacrimogênio. As instituições financeiras, que
protagonizam a destruição de tantas vidas com suas políticas de crédito
predatórias e covardes, com seus juros abusivos e sádicos, não foram
poupadas. O povo não é inconsciente, esta é a mensagem que fica; o povo
ataca sem líder, mas de modo extremamente significativo. Bancos
invadidos, cadeiras e títulos de dívida queimados, computadores e sofás,
uma grande fogueira em frente à Assembleia Legislativa. As pessoas
dançavam em êxtase ao redor da fogueira. Diálogo ao pé do ouvido: ``Vingamos a aldeia Maracanã'' --- ``Não, ainda não''\ldots{}

Pedras incessantes em direção à \versal{ALERJ}. Os guardas lá dentro reféns de
uma multidão da qual não se via o fim. O choque cerca a área, mas não
avança. Parados, apenas assistem os morteiros que estouram na
porta da Assembleia. Centros culturais não são atacados nem depredados. Mas
os móveis coloniais voam da janela da \versal{ALERJ} direto para a fogueira.
``Vamos embora, a polícia vai vir e vai prender todo mundo''. ``Eu não
posso ir embora porque vivi todos os dias da minha vida até hoje apenas
para estar aqui e assistir isso''. ``Tomamos a casa do povo!''

Quem estava ali não esqueceria, cada rosto coberto era
um aliado. Daquelas mentes e corações não poderá ser apagado o
significado de enfrentar a polícia e destruir símbolos da opressão
capitalista ao lado de uma massa sem fim de pessoas. Não foi um processo revolucionário,
mas foi como que o prato de entrada da revolução. A possibilidade da
revolução acenou no horizonte naquele dia e tenho certeza que nenhum
usurpador conseguiu dormir em paz.

Sim, vida é coisa que se reconstrói, da destruição pode nascer o novo,
uma relação de forças pode mudar, uma sociedade inteira pode mudar e
naquele dia o processo pelo qual o impossível pode se tornar possível se
manifestava concretamente para aquelas pessoas pelas forças de suas
ações.

Pensava assim quando começaram os tiros de verdade e vi duas pessoas
caírem na minha frente e os socorristas correrem para atendê"-las. Corri
tanto quanto pude na direção oposta e fui embora. Mas nunca esquecerei
este dia, e, quando eu estiver bem velhinha e pronta pra morrer, vou
lembrar da fogueira e dos gritos em frente à \versal{ALERJ}, e de como esse dia
vale uma vida inteira. Assim, eu vou morrer sorrindo.

\chapter{Torturas}

Hoje faz 10 dias. Apenas três dos vinte e três permanecem presos. Toda
noite uma presa é torturada no castigo ao lado da nossa cela. Nós
ouvimos os gritos dela e não podemos fazer nada. Já tentamos intervir
mas as funcionárias nos ignoram, tudo que nos resta é chorar e sofrer
junto. O frio aqui é usado como forma de punição, por isso não podem
entrar casacos. A cela do castigo é fria, úmida, suja e escura. Ouvir
uma pessoa chorar de dor e não poder fazer nada também é tortura. Este é
o sistema carcerário, não há nada de bom aqui, ninguém merece vir pra
cá, este lugar apenas não deveria existir. Ninguém precisa fabricar
demônios, nem infernos, bastam os presídios e seus regimes de terror. As
internas precisam se humilhar o tempo todo, porque as funcionárias precisam
manter a atmosfera de medo. Não andam armadas aqui, e cada plantão deve ter 10
funcionárias para em torno de 400 internas. Se não conseguem persuadir
com castigos e ameaças, poderiam ser mortas até chegar reforço. Eu sei
que elas atacam por medo e sadismo fabricado pelo sistema, eu sei que a
humanidade delas foi vendida barato. A quantidade de presas é muito
maior que a de agentes, e mesmo que estas estivessem armadas, se
houvesse uma revolta coletiva, elas não teriam chance. Prepotência e
arbitrariedade são modos de manter o medo constante, e o medo constante
é a única possibilidade de controle e dominação. Humilhar, baixar a
moral, fazer você se sentir sub"-humana não é apenas e fundamentalmente
punição, mas é sua própria condição de possibilidade.

Pode ser que o \emph{habeas corpus} saia ainda esta semana. Quando os
advogados vêm, nos enchemos de esperanças, mas a esperança frustrada pode
ser pior que a angústia. Sustentar a angústia, sustentar o
insustentável, transvalorar o absurdo pela revolta, é preciso não ter
esperanças nem desespero\ldots{} e o tempo lento domina tudo\ldots{}

Por vezes se pode ouvir o amor resistindo, presas namorando que, de
celas separadas, pedem para sonhar uma com a outra. É coisa que
agrada ouvir, a vida dando sua maneira de
seguir apesar do sistema. Mas neste momento o que escuto é a interna na
tranca pedir desculpas, por incomodar com seus gritos; a funcionária
a ignora. Tenho a impressão de que os pássaros e gatos aqui riem da nossa humanidade.
Os animais, mais livres, circulam na penitenciária, entram e saem de onde
não podemos. Aqui eles se vingam de toda a nossa suposta superioridade.

\asterisc

Mais uma interna na tranca, agora durante o dia. A tranca é a solitária,
que serve como castigo e pode ser limpa ou suja. Quando é suja, as
funcionárias chamam de ``buraco''. A impressão que se tem é que quanto
mais se fica sem visita, pior se é tratado. O sistema aposta no abandono
e as mulheres são as principais abandonadas. Ser preso é como morrer, só
que em vida. No início as pessoas se negam a acreditar, depois se
revoltam e, finalmente, te esquecem e seguem suas vidas. Sempre alguém
não vai esquecer, mas, no geral, somos mortos"-vivos. Toda a sociedade é
cúmplice de que lugares como este existam. Ontem uma interna nos jogou
um bilhete pedindo para denunciar o que elas passam, a invisibilidade é
grande. A frase mais recorrente entre as funcionárias é ``porque eu
quis'': ``você vai pra tranca porque eu quis''; ``você vai ficar com frio
porque eu quis''; ``porque eu quis não tem banho de sol para você'';
``porque eu quis você passa a noite sentada no banco esperando
atendimento médico''. Perguntei para uma carcereira:

  --- Você não se sente mal estando de casaco com outras pessoas passando
frio ao seu lado?

Ela chegou bem perto de mim para me intimidar antes de responder. Existe
toda uma vestimenta usada para expressar poder e intimidação dentro da
penitenciária feminina. As detentas são em sua maioria negras, andam de
camiseta, bermuda e chinelo; as funcionárias são em sua maioria brancas,
cabelos lisos e pintados de loiro, e só andam de salto ou botas. A imagem
salta aos olhos.

--- A família dela é que precisa trazer na custódia.

--- Mas a minha pergunta é bem mais simples, sra. funcionária, eu só
queria saber se você não se sente mal.

--- Não, não me sinto, a família dela é que devia se sentir.

--- Entendi\ldots{}

Mas por que razão ela se sentiria? Vivemos num mundo onde tantos não têm
nada ao lado de outros que têm tudo. A maioria das pessoas também não se
importa de fato com as pessoas sem casa, sem comida, sem escola, sem
hospitais. Existem boas justificativas para tudo isso e, no final das
contas, ``elas merecem estar assim''. Ah, a punição e o prêmio, os dois
lados do merecimento. O que dizer então daqueles que cometeram crimes,
os vagabundos, os que foram apanhados, os matáveis, aqueles em relação
aos quais a violência é socialmente justificável e aceitável e sobre os
quais pode ser exercida desavergonhadamente. Não, o problema não é o
crime que elas cometeram, jamais foi, o problema é quanta violência e
poder se pode agora exercer sobre elas, mesmo se convencendo de que você
é uma boa pessoa, uma cidadã de bem, a mocinha. Aqui se encontram as que
merecem sofrer, é preciso acreditar nisso, por um sistema desigual que
na realidade necessita da manutenção desta exclusão para continuar
existindo.

Assim estabelece"-se um exercício quase absoluto de poder que gera
satisfação sádica ancorada na certeza de que nada sairá daquelas
paredes. Quem está preso é refém do Estado, não pode fazer nada, a
família temerá dizer o que quer que seja para que seu parente não sofra
represárias. Quem sai, se mora em zonas militarizadas, permanece refém,
sempre se pode voltar e, também lá fora, se pode morrer. Ter que se
calar diante do intolerável é uma das expressões mais eficazes da
violência continuada.

\chapter{Impressões parciais do primeiro dia de nosso julgamento}

A audiência tinha um clima espetacular, com fotógrafos e redes de
televisão\ldots{} As mídias alternativas não tinham autorização para filmar
lá dentro, mas a \emph{Rede Globo} podia permanecer. Quando entrávamos no
tribunal, uma câmera disposta no meio do nosso rosto devia registrar
nossa expressão naquele momento, não havia a possibilidade de negarmos o
uso de nossa imagem. Que liberdade de expressão é essa reservada apenas
aos grandes veículos midiáticos que detêm o direito de acossar com
câmeras aqueles e aquelas que estão sendo julgados, sem que lhes seja
possível deliberar sobre o uso de sua própria imagem?

Iniciou"-se a audiência com nossos punhos erguidos. Quando os
companheiros algemados entraram na sala, gritamos: ``Não passarão!'' Ao que se seguiu a primeira manifestação do juiz Itabaiana: ``Aqui quem
manda sou eu, aqui não tem punhos cerrados não, aqui não é a rua''. O
tribunal é um espaço absolutista, o juiz é o rei e os demais são seus
súditos. O réu não é sequer humano, é um objeto da vontade do soberano,
sem direito à voz, sem subjetividade. Nesse relato, entretanto, eu posso
esclarecer. Itabaiana considerou a frase como desacato. Ora, ``Não
passarão!'' é um lema internacional antifascista que expressa
determinação em se defender de um inimigo que avança. Foi usado na
Revolução Espanhola e também em vários outros momentos de resistência na
História. É bastante simbólico seu uso como cumprimento por aqueles
e aquelas que se defendem no momento em que se reconhecem como aliados.
Mas desacato significa ofender um funcionário público, qualquer que
seja, no exercício da sua função, não apenas juízes ou chefes de poder,
mas qualquer servidor. Como um lema defensivo usado como saudação pode
ser considerado uma ofensa?

A sala de audiência tem cinco telões, que nos cercam e refletem o julgamento por toda sala. Cria"-se um clima espetacular no ambiente
já ritualístico do tribunal. Atrás de nós estão as câmeras da \versal{TV}, na
frente, o juiz e a promotoria. Entre nós e eles, os advogados de
defesa, que criam uma aparente distância de segurança. Mas
tudo o que fazemos passa insistentemente nos telões. Acima da cabeça do
juiz um crucifixo, a imagem do poder religioso ao mesmo tempo conivente
e contraditório, uma vez que ele mesmo, Cristo, foi condenado à cruz que, ali disposta, legitima todo aquele ritual. Por outro lado, é o símbolo do poder patriarcal, Deus"-filho,
Deus"-homem, Deus"-condenado. Acima do representante do Estado, Laico,
Onisciente, Onipresente na sala, o Deus"-condenador. É sempre bom lembrar
que o pensamento jurídico ocidental se desenvolveu pelo e para o poder
régio. É sempre bom lembrar que o poder Judiciário é o absolutismo
presente na sociedade disciplinar.

É preciso muita semiótica, muita violência simbólica, para que as
palavras de um ser humano possam ser suficientes para colocar outro ser
humano na cadeia. Na audiência, eu queria dizer"-lhes, se os
ritos medievais assim permitissem, que não reconheço esse poder, que
não participaria desse ritual, que jamais o investi de tal prerrogativa. Mas
estaria ali, no lugar de réu, sentada ao fundo da sala, sem direito à
voz. E ainda que quisesse dizer bem alto que as palavras ``mágicas'' do
juiz não poderiam ter efeito sobre mim, sabemos muito bem que a semântica é
determinada por uma prática social que confere poder a tais palavras.
Não por conta de qualquer metafísica, mas porque toda a metafísica que
somente a gramática pode criar incide diretamente agora sobre nós. As
revoluções modernas não cortaram cabeças suficientes.

\asterisc

A primeira testemunha da acusação foi a Delegada Renata. Ela situou a
formação da \versal{FIP} (Frente Independente Popular)\footnote{A Frente
  Independente Popular do Rio de Janeiro foi formada a partir dos
  protestos de 2013, reunindo militantes e ativistas não organizados
  juntamente com coletivos e organizações políticas não eleitorais. A
  principal unidade entre eles era justamente o repúdio ao processo
  eleitoral e a defesa da legitimidade da resistência popular e da
  combatividade nas ações de rua.}, com objetivos criminosos, em
setembro de 2013, que também seria o início do inquérito. De acordo com
o informante Felipe Braz, a \versal{FIP} teria sido a responsável por transformar
o movimento popular de junho em crimes e atos de vandalismo. É
importante ressaltar que tudo que ela disse se baseia no depoimento de
Felipe, a única fonte de fato. A Delegada não testemunhou nada. A \versal{FIP}
teria saído do fórum do \versal{IFCS} (Instituto de Filosofia e Ciências Sociais
da \versal{UFRJ}). Teria plenárias abertas, nas quais seriam aprovadas passeatas e
panfletagens, e teria uma comissão fechada, interna, na qual seriam
decididas \emph{ações diretas} (em seu próprio termo). Ela cita um
testemunho do Felipe Braz, uma suposta reunião em uma barraca do ``Ocupa
câmara'' na qual teria sido aprovada a queima de um ônibus. Não importa se
tal ônibus foi ou não queimado, a \versal{FIP} havia planejado queimá"-lo,
planejar cometer crimes já é cometer crimes. A tese central é de que
partiríamos de problemas reais na cidade, que são motivos reais para
fazer passeatas para, a partir disso, cometermos crimes. Mas quais são
os crimes? Aqueles que planejamos e nem sequer cometemos. Um ponto
importante: não assumir o caráter político das ações faz parecer que
cometeríamos crimes pelo prazer de cometer crimes, já que não existe
vantagem financeira, ou de qualquer outra espécie, envolvida.

Começa então a falar da Elisa Quadros. Diz que ela era a líder máxima, líder
na \versal{FIP} e nos ``Ocupes'', diz que ela que coordena os pedidos de quentinhas,
diz que ela pede quentinhas para manifestantes. Diz que ela se ausentou
do Rio com medo de ser presa, que ela estava preocupada com Minas Gerais
porque lá já haveria provas da formação de quadrilha da \versal{FIP}"-\versal{MG}.
Entretanto, não havia \versal{FIP}"-\versal{MG}. Começa a falar da questão financeira, com outra informante como fonte, que diz que seriam políticos
partidários. Depois fala sobre pessoas que teriam como função
``jogar molotovs de dentro de veículos em movimento''. Nesse momento,
houve uma gargalhada no tribunal, repreendida pelo juiz. A livre
interpretação de testemunhos confusos, fundada em afetos pessoais e
fragmentos de conversas ou postagens no Facebook segue até o final de
seu testemunho. Quando a livre interpretação se torna fato, não importa
se alguém não lembra ou simplesmente inventa. Mesmo que isso coloque
alguém preso. Mesmo que destrua vidas. Toda inocência ainda tem que ser
provada, invertendo o que o próprio Direito postula sob a rubrica da inocência presumida.

Lembro do momento que ela mencionou o dia 28 de junho, quando a Andressa
foi detida e eu, preocupada com ela, liguei para várias pessoas para
ajudá"-la. Renata (investigadora) diz que isso prova que eu sabia que ela
era menor, pois pergunto se ela não teria sido levada para a \versal{DPCA}
(Delegacia de Proteção à Criança e Adolescente), o que é usado para me acusarem de
corrupção de menores em relação a ela. Ora, quem não se preocuparia com
uma menina de 16 anos sendo levada pela Polícia Militar? Como saber que
ela é menor e me preocupar com ela pode significar que a corrompi?

Fizeram uma megaoperação e gastaram dinheiro público para fabricar
crimes que não existiam. O próprio informante policial tentava incitar as
pessoas a cometerem crimes para ter material para seu relatório, que
provavelmente foi vendido. Os agentes do Estado perseguem, inventam,
incitam\ldots{} qualquer conversa pode ser interpretada contra aqueles que estão
sendo acusados. E existem infinitas
interpretações possíveis, mas somente aquelas favoráveis às
acusações serão levadas em conta. Todo o processo é um grande teatro,
uma farsa que só não é tragicômica porque destrói psíquica e
materialmente a vida de pessoas. Por que a palavra de informantes é tão
inquestionável? O que torna tal palavra critério de verdade, se ele
mesmo não apresenta nenhuma prova material para o que diz? Se ele mesmo
possui razões para mentir? Se ele reproduzia o que queria delatar? Nada
aqui é conclusivo\ldots{}

O segundo depoimento acusatório foi do policial Marcelo Ortiz, seguido
por seu companheiro Marcio. Marcelo começou chamando de absurdo nossa ida à imprensa para dizer que Bakunin seria preso, como modo de desqualificar a
investigação. A sua formulação foi extremamente patética: ``Bakunin
não estava processado, ele apenas aparece citado na investigação''.
Novamente uma gargalhada toma conta da sala, logo contida pelo juiz
Itabaiana. Seus depoimentos eram pontuados por inúmeras contradições,
inclusive com o que havia sido dito pela Renata. Impossível não pensar
na particularidade desses discursos sempre risíveis, sempre no viés
entre o grotesco e o patético, mas que ainda encontram respaldos
suficientes, na autoridade e na produção de verdade jurídica, para
condenarem em tribunais. Os absurdos condenam. É o império do ubuesco!
Eu queria fazer perguntas: onde estava o mandado de busca e apreensão?
Por que nossa prisão não foi documentada (filmada)? Outras, com a da
Elisa Quadros, foram. Por quê? Por que, mesmo sem resistência, Marcio
apontou uma arma para o meu então companheiro e o algemou? Por quê? Por
que a suposta bomba foi destruída, impossibilitando que um segundo laudo
fosse solicitado pela defesa? Não é estranho que uma prova tão
importante seja totalmente destruída? Mas eu não podia falar\ldots{}

\chapter{O infiltrado na \versal{FIP}}

Felipe Braz foi a testemunha mais importante, pois seu depoimento era o critério e
a referência de verdade da delegada, e muitos dos acusados estavam ali
só por causa dele. Esse informante policial se aproximou da Frente
Independente Popular (\versal{FIP}) em 2013, tentou entrar em diversos grupos
políticos, se aproximou muito de várias pessoas que o convidavam para
sair, para reuniões e para participar de ações em atos. Chegou a se
envolver afetivamente com uma militante, que chamarei aqui de
Simone\footnote{De agora em diante, ``Simone'' é o nome atribuído à pessoa
  com quem Felipe Braz se envolveu afetivamente com o intuito de obter
  informações.}. Claramente já estava ali com a intenção de recolher
informações, pois agia como provocador. Mas analisar isso posteriormente
ao ocorrido é fácil, ninguém pode ser culpabilizado pelas técnicas que o
Estado e seus agentes usam para criminalizar os movimentos sociais e
suas ações. Muitas pessoas se enganaram com Felipe Braz e com tantos
outros ao longo da história que cumpriram funções semelhantes. Toda
cautela é pouca, e é preciso aprender com nossa história. Ainda assim,
existirá sempre um âmbito não previsível que torna, em grande medida, a
proteção total impossível (mesmo porque a busca pela segurança total
pode ser paralisante).

A infiltração policial é um tipo de violência do Estado
recorrente, para se precaver é preciso saber as táticas e técnicas que usam.
Em todo caso, não se deve culpar as vítimas pela violência que sofreram
quando não foram capazes de avaliar um informante como tal. O informante
é contraprova definitiva de que o Estado trata todo cidadão como virtual
inimigo, e imprime a paranoia entre os iguais para manter sua dominância
sobre a vida e a conduta de todos.

O próprio Felipe Braz não apresenta nenhuma prova para o que fala, ele é
uma prova forjada, apresenta como critério o ``ouvi dizer'' e a palavra da
Simone. Sua fala se faz verdade apenas porque está investida pelo interesse
dos que julgam com a condenação já pronta. Ora, ele jamais diz que viu
algo, quase nada ele viu realmente, além da tal suposta reunião dentro
de uma barraca no ``Ocupa câmara'', que não se sabe se ocorreu. ``Ouvir
dizer'' não é critério e a Simone desmente todo o resto. Há uma clara
circularidade: Felipe Braz comprova a interpretação das escutas, a
interpretação das escutas comprova Felipe Braz. Esta circularidade
recebe o nome nos tribunais de conjunto probatório.

Felipe Braz é misógino, espancava travestis, pagava meninos de rua para
transportar explosivos, agrediu fisicamente duas mulheres em uma
manifestação. Ele demonstra profundo ódio quando fala de qualquer
mulher, tem uma visão nefasta do feminino como nocivo. Nesse ponto, ele,
o juiz e meu então advogado parecem estar de acordo. Ah, o patriarcado e
seus finais felizes. O melhor argumento contra ele só poderia ter vindo
de uma mulher. Eu gostaria de citar a advogada Fernanda Vieira: ``Se ele
participou de uma reunião que deliberou um crime, ele é criminoso, ou
então nem todos que participavam desta reunião eram criminosos''. No
final, ele teria que desconsiderar que todos ali eram
responsáveis pela ação/criminosos/líderes, ou ele mesmo teria que se
entregar. Como ele poderia ter visto algo, tudo que diz saber, sem tomar
parte em nada? Ou ele não viu e está inventando ou ele viu e participou, tornando"-o tão criminoso quanto os demais acusados no tribunal. Mas o caminho
escolhido pela defesa não deve ser o de levá"-lo ao banco dos réus, o
caminho é considerar que ele não sabe mesmo de nada do que está dizendo, que
inventou, que tentou construir provas para fazer um relatório/dossiê
encomendado.

Felipe Braz disse que fez o dossiê para o Ministério Público (\versal{MP}) após a
morte de Santiago Andrade por ter senso de justiça. Ele responsabilizou
a \versal{FIP} pela morte. Incentivada por isso eu gostaria de dizer
algumas palavras sobre senso de justiça, essa noção que
aparentemente a tudo se presta.

Senso de justiça é não aceitar que apenas alguns têm onde morar sabendo
que tantos moram na rua; é não poder comer em paz sabendo que tantos
comem lixo; é não poder dormir tranquilo sabendo que tantos não dormem,
ouvindo tiros; é não aceitar sem fazer nada o discurso meritocrático
enquanto para tantos é negada até mesmo a educação básica; é saber que
não se está livre enquanto outros estão presos; que a vida de ninguém
pode ter sentido enquanto tantos morrem por nada ou para que outros
fiquem ainda mais ricos. Senso de justiça é não se calar diante das
injustiças com medo de perder seus privilégios.

De fato, quem matou Santiago não foi Caio ou Fabio, foi o Estado, o
mesmo Estado que mata tantos e tantas Amarildxs. O que matou Santiago foi
a violência policial; a desigualdade; as filas dos hospitais precários;
os estupros nas favelas; a gentrificação; o trabalho escravo; e não
aqueles ou aquelas que se levantaram contra essas e tantas outras
injustiças. O Estado pode até não ter matado Santiago diretamente, mas
ele o matou indiretamente e continua o matando e se aproveitando da sua
morte oportunamente para criminalizar e punir exemplarmente a revolta
popular.

Quem tem senso de justiça não se revolta com uma morte acidental apenas,
se revolta contra a morte sistêmica e estrutural, a morte repetida,
diária, consentida pelo monopólio da violência estatal, as mortes
conclamadas e legitimadas pelos meios de comunicação.

Há algo de muito significativo e doloroso quando um agressor, machista,
misógino, mentiroso, violento, judas, traidor, corruptor de menores,
cagoete, homofóbico, delator, desprezível até entre aqueles que cometem
os atos mais vis, é ouvido falando em senso de justiça em um tribunal.
Há algo de muito significativo no fato da nossa sociedade dar voz e
crédito a esta pessoa para condenar e prender trabalhadorxs
precarizadxs, professorxs, jornalistas, profissionais liberais, mães,
filhas, irmãs, estudantes\ldots{} pessoas que sabem pelo que lutam e no que
acreditam.

A justiça não deveria ser uma palavra vazia, a justiça não deveria ser
um tribunal, a única justiça que reconheço é a da revolta, e esta nasce
no coração de pessoas que subitamente dizem ``basta, não mais, não
passarão!'' e se levantam diante de tudo aquilo que não é dito, que é
invisibilizado, ainda que, para isso, tenham que colocar em risco suas
próprias vidas.

\chapter{Não é brincadeira, a \versal{UERJ} apoia a Mangueira}

No dia 24 de maio de 2015 ocorreu um novo processo de remoção na favela
conhecida como ``favelinha do Metrô"-Mangueira'', situada ao lado da
Universidade do Estado do Rio de Janeiro (\versal{UERJ}). Nesse dia, várias
famílias tiveram suas casas derrubadas com tudo dentro, as crianças
chegavam da escola sem ter mais onde morar, uma senhora foi
soterrada dentro da sua própria casa por não conseguir sair, um bebê de
dois meses foi internado por respirar gás lacrimogênio, pessoas já abaixo
da linha da pobreza perderam as poucas coisas que tinham, e nada disso
saiu no jornal.

Mas nesse dia ocorreu também algo excepcional: estudantes da
Universidade do Estado do Rio de Janeiro foram apoiar a resistência da
favela contra a remoção. A Tropa de Choque que esperava massacrar uma
população invisibilizada pelo narcotráfico encontrou também cerca de
duzentos universitários brancos e de classe média engrossando uma
manifestação no local. Novamente, como em 2013, essa potente parceria
gerou um efeito inesperado na repressão, que não sabia como reagir. As
famílias resistentes e demais moradores da favela se uniram em um ato
histórico que retornou à Universidade sob forte repressão policial. No
momento em que os moradores da Mangueira entraram na Universidade, esta
fechou suas portas para impedir que as pessoas acuadas lá pudessem se
refugiar. A segurança da \versal{UERJ} apoiou totalmente a ação policial,
chegando a espancar um estudante que foi trancado dentro de uma sala por
horas. As mangueiras contra incêndio foram usadas pelos seguranças para
conter os manifestantes, e, segundo alguns, foi a força da água que quebrou as vidraças. Sim, as vidraças da \versal{UERJ} foram
quebradas. Isso saiu na primeira página dos principais jornais no dia
seguinte. A reitoria lançou uma nota de repúdio à ``presença de pessoas
estranhas à comunidade acadêmica na \versal{UERJ}'' (sic.).

O que se seguiu foi uma liminar impedindo que as remoções continuassem,
liminar que mantém as famílias resistentes no local até hoje. Mas não
foi só isso. No dia seguinte, a polícia estava no nono andar da \versal{UERJ}, os
telefones do Instituto de Filosofia e Ciências Humanas foram grampeados,
uma investigação foi instaurada para apurar os responsáveis pela
depredação do patrimônio público, além de uma fantasiosa ligação entre o
Centro Acadêmico de Filosofia e o narcotráfico. Dizem que essa última
investigação foi instaurada a partir da denúncia de um professor do
próprio Departamento de Filosofia. Quase todos os professores da
Filosofia, juntamente com alguns da História e das Ciências Sociais,
foram intimados a prestar esclarecimento na delegacia. Eu fui uma entre
estes. O inquérito não foi adiante e ninguém foi processado, mas o
terror pela criminalização foi instaurado entre nós.

\chapter{Algumas palavras sobre o ocorrido ontem na \versal{UERJ} e a nota do reitor\footnote{Nota lançada pelo movimento social Ação Direta em Educação popular relativa ao ocorrido na \versal{UERJ} no dia 28/05/2015.}}

O título da nossa nota poderia bem ser o mesmo da nota lançada pela
reitoria: ``Não há diálogo com a barbárie'', pois barbárie é derrubar a
casa das pessoas com suas coisas dentro, é atirar em criança, é
invadir residências para espancar moradores. Em máximo grau, a exclusão
social gritante na qual nos encontramos é a verdadeira barbárie. A
reitoria sabe disso, mas um dos procedimentos frequentes da guerra dos discursos é a inversão dos efeitos pelas causas das relações estruturais sistêmicas, o que culpabiliza as resistências. Isso não é
novidade. É uma das coisas que a história nos ensina. E sim, a história
nos ensina de fato muitas coisas, sobre momentos em que a polícia
invade Universidades, sobre o fim dos diálogos possíveis à força de
tiros.

Mas achamos que algo deve ser valorizado nos acontecimentos desta última
quinta"-feira, para além da criminalização costumeira. Foi fantástico
que, apesar das imensas diferenças que existem no cerne do movimento
estudantil da \versal{UERJ}, os estudantes decidiram consensualmente apoiar a luta da Mangueira. Isso demonstrou grande unidade e maturidade
entre nós. Não vamos deixar que a reitoria criminalize alguns grupos,
colocando"-os como terroristas responsáveis pelas ações. O grande valor do
ocorrido foi a solidificação do diálogo e o apoio mútuo entre a Mangueira
e a Universidade. Infelizmente, não foi possível evitar a remoção das
casas, mas sem dúvida fortaleceu"-se o vínculo para a
construção de uma Universidade popular e para a unificação das nossas
lutas. Os problemas da Mangueira também são nossos problemas, e temos
certeza que a visibilidade da \versal{UERJ} foi fundamental para a resistência
das famílias e para a aprovação da liminar impedindo a continuidade das remoções. Que este vínculo permaneça, que esta aliança prospere,
que fique cada vez mais claro em que sentido a precarização da educação
e as remoções na favela fazem parte de um mesmo projeto político. Se somos atacados por um mesmo inimigo, faz todo sentido que possamos resistir conjuntamente.

Mas a reitoria debocha de nossa unidade e da sociedade. Primeiro, afirmou que ``não vai ter sexta"-feira sangrenta na Universidade'', como
justificativa para suspender autoritariamente o conselho universitário,
e agora usa a expressão ``não passarão''. Queríamos dizer que tal como a
\versal{UERJ} não é quintal da reitoria, a vida e as lutas concretas das pessoas
não são motivos de piadas. Talvez pareça engraçado para alguém
que nunca imaginou ter a casa destruída pela polícia, ou cujo filho
nunca levou uma bala de borracha na cabeça, como ocorreu ontem com uma
criança de cinco anos. Mas essa reitoria representa e defende a
violência de Estado. Não vamos deixar que nossa força seja usada contra
nós, jamais nos envergonharemos de termos a Mangueira ao nosso lado.
Nossa força está justamente nesse diálogo da Universidade com a favela.
Nossa legitimidade enquanto resistência está na violência policial
anterior e diária instaurada na comunidade. É uma pena que seja dado
mais valor às vidraças do que às vidas e moradias das pessoas. Alguns
acham que a pedra e o vidro quebrado são as únicas causas e fins
de tudo que aconteceu naquela noite. O que é horrível, pois essas
discussões morais tentam mascarar as remoções na Mangueira, as agressões
de seguranças contra mulheres e estudantes, a precarização da Universidade,
as bombas lançadas e os tiros de armas letais disparados pela polícia na
Mangueira e na \versal{UERJ}, além dos motivos que levaram ao ato.

Mas, ao menos agora, o desalojo absurdo foi minimamente notado e
problematizado. Não fosse isso, talvez nem sairia no jornal. Talvez a
elite intelectual em seus seminários e simpósios internacionais nem
ficasse sabendo da violência extrema exercida ali do lado. Mas se a
realidade bate à porta, não significa que alguém foi chamar
``agentes externos à Universidade''. Na sua nota, o \versal{REI}tor faz
referência aos moradores da Mangueira como ``pessoas estranhas à nossa
comunidade''. Claro que é estranho! Pois não se imagina uma universidade
pública e popular na \versal{UERJ}. Não se pensa em meios de furar a bolha
acadêmica que é essa universidade. Os moradores da Mangueira não devem
mais passar pela \versal{UERJ} apenas para cortar caminho até suas casas, enquanto
elas ainda estão lá, é claro. O que nós gostaríamos de dizer é: a
comunidade, a sociedade, os excluídos não mais serão agentes externos a
esta Comunidade.

\chapter{Meu depoimento na Polícia Civil}

Cheguei à delegacia acompanhada de dois advogados para prestar
depoimento. Após as devidas apresentações, o delegado esclareceu o
motivo de minha intimação. Havia duas investigações em curso. A primeira
foi aberta após maio de 2015, para averiguar danos ao patrimônio público
da \versal{UERJ}; a outra foi aberta a partir da denúncia de
um professor do próprio Departamento de Filosofia, para apurar
a existência de tráfico de drogas no nono andar e uma possível
associação entre o Centro Acadêmico de Filosofia e o tráfico de drogas
na Mangueira --- associação que me envolveria diretamente. Deixei
escapar um riso nervoso e comentei o quanto isso era um absurdo sem
tamanho, que não poderia deixar de me surpreender e espantar, mais
do que qualquer outra coisa. De fato, eu não conseguia ficar seriamente
preocupada, mas vivemos em um mundo surrealista. Depois de investigar
Bakunin, a polícia ia me criminalizar por tráfico de drogas.

\asterisc
\begin{Parskip}
\versal{Delegado:} Nós recebemos a denúncia e temos que investigar. Mas
eu mesmo não acredito nisso. Achei desmoralizante quando o processo
contra os 23 ativistas citou Bakunin como procurado. Não quero que algo
semelhante ocorra por aqui. Eu sou amigo do Zaccone, não sou um policial
fascista. De fato, foi um professor de Filosofia muito estranho este que
fez a denúncia, ele disse que bom mesmo era na época da Ditadura
Militar, que vocês já estariam todos presos. Eu não penso assim, sou
policial civil, jamais defenderia a volta de uma ditadura militar no
Brasil.

\forceindent\emph{``Seja como for, certamente aquele parecia ser o `tira bom'", eu pensava.}

\versal{Advogado 1:} Nós podemos ter acesso às investigações?

\versal{Delegado:} Sim, podem, não hoje, mas depois podem marcar e virem
aqui, tirar cópias. O material não está organizado, eu tenho algo aqui
que os senhores podem dar uma olhada já.

\forceindent\emph{O delegado passa umas duas pastas para os meus advogados, com os
depoimentos colhidos até então, eu olho um pouco de rabo de olho, não
sei se eu mesma posso ver aquilo, mas identifico os cartazes do Grupo de
Estudos Anarquistas Maria Lacerda de Moura, do qual faço parte, entre os
papéis. São muitos cartazes, \emph{prints} da nossa página no
Facebook, panfletos do Ação Direta em Educação Popular, que é o
movimento social no qual participo. Muitas das nossas atividades públicas
estavam ali, catalogadas. A gente imagina que isso acontece, sabe que
pode acontecer algo neste sentido, mas ainda assim é surpreendente que
eles percam tanto tempo conosco.}

\forceindent\emph{Outro policial entrou na sala para fazer papel de escrivão. Meus
advogados combinam de ver as pastas com calma um outro dia. A partir de
então tudo que dizemos passa a ser digitado. As perguntas começam:}

\versal{Delegado:} A senhora estava presente na \versal{UERJ} no dia 28/05/2015?

\versal{Camila Jourdan:} Sim, eu estive lá mais cedo, dei aula, depois
tive algumas reuniões de orientação com estudantes. No início da tarde
tentei ir até a sala do \versal{ADEP} (Ação Direta de Educação Popular), que é o
pré"-vestibular comunitário vinculado a um projeto de extensão da \versal{UERJ}
que coordeno, na Mangueira. No caminho, eu me deparei com as remoções,
muita violência policial, senhoras e crianças machucadas, gás por todo
lado, casas sendo derrubadas. Cheguei a parar um pouco, mas como havia
uma manifestação começando, eu não podia ficar no local, tenho uma
medida restritiva que me impede de estar presente em manifestações. Eu
retornei à \versal{UERJ} e vi quando a manifestação chegou lá no início da noite.

\versal{Delegado:} O que aconteceu naquele dia na \versal{UERJ}?

\versal{Camila Jourdan:} Os seguranças da \versal{UERJ} não deixaram entrar as pessoas
que pretendiam se refugiar do Choque dentro da Universidade. As
portas da Universidade foram fechadas e as pessoas ficaram encurraladas
entre o Choque e os seguranças, que usaram mangueiras de água para
dispersar a manifestação. Isso foi o que consegui ver da janela.

\versal{Delegado:} A senhora viu ou sabe quem quebrou as vidraças da
\versal{UERJ}?

\versal{Camila Jourdan:} Por mais que os manifestantes arremessassem
coisas nos seguranças, eles não teriam condições de quebrar uma vidraça
daquelas. As vidraças foram quebradas pela força da água das mangueiras
dos próprios seguranças. Existem vídeos na internet mostrando isso
inclusive. É inútil tentar acusar os estudantes ou os moradores da
favela, a quantidade de imagens da violência por parte da segurança da
\versal{UERJ} é muito grande, inclusive agredindo estudantes, existem várias
imagens desta violência na internet. Eu fico pensando que isso sim deveria
ser investigado, apurado melhor, sabe?

\versal{Delegado:} Alguma vez a senhora defendeu publicamente, em alguma
aula ou palestra, o uso da violência como arma política?

\forceindent\emph{Neste momento minha vontade era dizer ``defina violência''. Mas eu não
podia me arriscar diante de alguém que estava me investigando e que
certamente poderia usar isso contra mim, não poderia parecer que eu
incitava o crime em qualquer sentido.}

\versal{Camila Jourdan:} Não, jamais.

\versal{Delegado:} A senhora nunca defendeu ações violentas em
manifestações, em nenhuma aula, em nenhuma palestra?

\forceindent\emph{Um dos advogados intervém pedindo esclarecimento sobre a pergunta, sobre
o que ela contribui para as investigações em curso. Eu fico em silêncio,
tenho claramente a impressão de que ele tem algum áudio meu que está
interpretando com esta pergunta, por isso parece estar tão certo sobre
alguma coisa. Lembro"-me de que estudamos o texto \emph{Como a não"-violência
protege o Estado} recentemente em nosso Grupo de Estudos. Me ocorre
então dizer alguma coisa.}

\versal{Camila Jourdan:} Olha, eu dou aulas teóricas, nós discutimos
textos, debatemos argumentos, nesse espaço é preciso ter liberdade
para refletir sobre a realidade e considerar vários pontos de vistas,
claro que muita coisa pode ser mal interpretada, mas não é possível que
eu fique em sala de aula pensando no que podem dizer sobre as ideias que
estamos debatendo na polícia porque isso acabaria com o meu trabalho.

\versal{Delegado:} Qual a sua ideologia? A senhora é comunista, é
anarquista?

\versal{Camila Jourdan:} Bom, isso é público, eu sou anarquista.

\versal{Delegado:} O que defende um anarquista?

\versal{Camila Jourdan:} O senhor não acha estranho que em 2016 eu
esteja sendo intimada a prestar esclarecimentos sobre minha ideologia
política na polícia? E falo isso até porque o senhor disse que jamais
seria a favor da volta da ditadura.

\versal{Delegado:} Não, mas a senhora não está prestando esclarecimento
sobre isso, eu estou te perguntando isso porque tenho curiosidade mesmo.
Eu gostaria de entender melhor do assunto, eu me interesso muito por
Filosofia. E eu estudei a teoria do Estado em Hobbes, então eu fico
querendo entender como pode haver sociedade humana sem contrato social?

\versal{Camila Jourdan:} Bom, eu não estou aqui exatamente conversando
com o senhor por livre e espontânea vontade, não é mesmo? Eu fui
intimada para estar aqui, então todas estas perguntas fazem parte de uma
investigação policial e minhas respostas estão sendo digitadas, não é
como se eu estivesse falando livremente em uma conversa ou em uma aula.

Mas, ok, o anarquismo defende exatamente o oposto de autores como Hobbes
e outros teóricos do Estado, no sentido de que defende que pode sim
haver sociedade sem Estado, porque a organização social pode se dar de
baixo pra cima, diretamente entre as pessoas, partindo de células
sociais, que seriam federalizadas. Neste sentido, a existência do Estado
não é condição necessária para haver organização social, ao contrário, o
Estado é que suporia a sociedade e esta poderia então se organizar sem
ele.

\versal{Delegado:} Mas, de qualquer modo, o anarquismo defende então o
fim do Estado, certo? Ora, se vocês querem que o Estado seja destruído,
como vocês esperam que isso vai ocorrer sem o uso da violência?

\forceindent\emph{Os advogados me interrompem e um deles responde por mim.}

\versal{Advogado 2:} Esta é uma visão muito limitada do anarquismo,
precoceituosa e do início do século, achar que todo anarquista defende
e/ou faz uso da violência. É uma visão que equaciona anarquista com
terrorista e por isso criminaliza.

Os anarquistas hoje acreditam na modificação social principalmente por
meio da educação. É isso que a Camila faz, ela participa de um grupo de
educação que tem um trabalho na Mangueira, um trabalho que mostra como
este tipo de organização de baixo pra cima é possível, porque defende e
usa a educação libertária.

\versal{Camila Jourdan:} Sim, a educação pode ser uma arma de
modificação social, nós defendemos isso, e acreditamos também que toda
mudança tem que ser orgânica, a sociedade tem que estar organizada em
células sociais autônomas de baixo pra cima, isso para poder chegar a
pensar em derrubar o Estado. Fomentar este tipo de organização é uma das
principais tarefas de uma militância anarquista hoje.

\versal{Delegado:} A senhora alguma vez viu comércio de drogas no Centro
Acadêmico da Filosofia ou algum outro local do nono andar?

\versal{Camila Jourdan:} Olha, isso é um absurdo sem tamanho\ldots{}

\versal{Delegado:} Mesmo sendo um absurdo eu tenho que te perguntar.

\versal{Camila Jourdan:} Como professora, eu não sou frequentadora do
Centro Acadêmico de Filosofia, mas não, eu nunca vi comércio de drogas
no nono andar.
\end{Parskip}

\part{ENTREVISTAS}

\chapter{Uma líder fabricada\footnote{Matéria publicada em 28/07/2014, na \emph{Folha de São Paulo}.}}

\section*{Acusada de articular atos violentos, professora diz que
inquérito é ficção}

Por 13 dias, a professora universitária Camila Jourdan, 34, permaneceu
em uma cela no complexo penitenciário de Bangu, na zona oeste carioca.
Ela é uma das protagonistas do inquérito com mais de 2.000 páginas,
produzido pela Polícia Civil do Rio, que, sob classificação de
``quadrilha armada'', responsabiliza 23 pessoas pela organização de
ações violentas em protestos.

``Do pouco que li, posso dizer que esse processo é uma obra de
literatura fantástica de má qualidade'', definiu Camila, em entrevista à
Folha, no sábado (26), dois dias após conquistar sua liberdade
provisória.

Ela cita o teórico do anarquismo Mikhail Bakunin, ao falar sobre a
fragilidade do inquérito. Em mensagens interceptadas pelas polícia,
Bakunin era citado por um manifestante e, a partir daí, o filósofo
russo, morto em 1876, passou a figurar nos autos como potencial
suspeito.

Por volta de 6h de 12 de julho, véspera da final da Copa, três policiais
civis invadiram o apartamento da professora, que estava acompanhada pelo
namorado, Igor D'Icarahy, 24, com mandados de prisão contra ambos.

De acordo com o inquérito, os agentes encontraram uma garrafa com
gasolina, uma bomba de fabricação caseira e outra conhecida como
``cabeção de nego''. Em diálogos grampeados, Camila faz referências a
``livros'' e ``canetas'', que, segundo os investigadores, seriam
respectivamente coquetéis molotov e rojões.

Camila se recusou a falar sobre provas contra ela por orientação de
Marino D'Icarahy, seu advogado e pai de Igor, que diz que as provas
foram plantadas pela polícia.

\section*{Líder ``fabricada''}

Às referências constantes a seu nome no inquérito, Camila atribui uma
razão: ``existe uma necessidade de se fabricar líderes para essas
manifestações. E quem se encaixa muito bem no papel da mentora
intelectual? A professora universitária. Cai como uma luva, entende?''

Na Universidade do Estado do Rio de Janeiro (\versal{UERJ}), Camila Jourdan
sempre foi associada à excelência acadêmica. Um currículo ``invejável'',
segundo um diretor da \versal{UERJ}. Formada em filosofia, concluiu o doutorado
pela \versal{PUC}"-\versal{RJ}, com direito a um período de estudos na Universidade de
Sorbonne, em Paris. Sua tese foi sobre a obra do filósofo Ludwig
Wittgenstein.

``É uma excelente pesquisadora que se destacou por um trabalho original
e muito sério'', avalia Luiz Carlos Pereira, seu orientador nas teses de
mestrado e doutorado.

De família da zona norte, Camila é neta de general. Seu pai morreu de
câncer, quando era adolescente. Solteira, conta com o apoio da mãe para
criar a filha, de 12 anos.

Classificada em primeiro lugar na seleção de professores da \versal{UERJ} em
2010, ela atualmente é coordenadora do curso de pós"-graduação em
filosofia. Diz não gostar da burocracia inerente ao cargo. Prefere a
sala de aula.

Ao longo da entrevista, manteve o mesmo tom de voz, sem alterações
dramáticas. Conduz sua narrativa de forma didática, com ironia, e pontua
a argumentação com perguntas ao interlocutor.

A professora recorre ao filósofo francês Michel Foucault para explicar
que sua formação acadêmica está dissociada de sua participação na \versal{OATL}
(Organização Anarquista Terra e Liberdade) e na \versal{FIP} (Frente Independente
Popular), grupos acusados no inquérito de promover ações violentas em
protestos.

``Foucault diz que os intelectuais descobriram que as massas não precisam
deles como interlocutores. Não tenho autoridade para falar sobre a
opressão de ninguém. O movimento não precisa de mim para este papel''.

Camila credita à \versal{FIP} o mérito de tirar das manifestações do Rio a
influência dos militantes de direita e dos partidos de esquerda.

Define"-se como anarquista. Começou a se interessar na adolescência. ``Eu
gostava muito de Raul Seixas e descobri que ele era anarquista. Ali
decidi começar a ler sobre o assunto.'' Aos 14 anos, saía para
distribuir panfletos pregando o voto nulo. Sua estreia em protestos de
rua foi no fim da década de 1990, época das privatizações do governo de
Fernando Henrique Cardoso.

O desempenho do governo Luiz Inácio Lula da Silva reforçou suas
convicções: ``O Lula era visto como a esperança de mudança e fez um
governo à direita. Esfregou na cara das pessoas aquilo que os
anarquistas sempre disseram: não adianta você mudar as peças do jogo se
o problema é o jogo.''

Ela considera o processo eleitoral ``viciado'', incapaz de provocar
alguma modificação social ou política. ``A participação política não
pode se resumir a um objeto de consumo. Mandam o eleitor comprar um
candidato. O ser humano precisa de participação política real e
permanente. Nós fizemos isso nas manifestações e nos trabalhos de base,
com movimentos sociais e assembléias populares'', afirma.

Atribui as ações violentas dos manifestantes a uma resposta à
truculência policial. ``Existe o direito à legítima defesa''. Rechaça a
tese de que a baixa adesão às manifestações recentes se deve à violência
e aponta a maior conquista neste processo.

``Ninguém em sã consciência achou que junho representava um momento
revolucionário. Foi importante no sentido do empoderamento da população.
Isso nem esta tentativa de criminalização pode tirar. Está feito. Neste
aspecto, a gente já ganhou.''

Camila analisa a possibilidade de perder e ser condenada: ``Tenho receio
do que pode acontecer porque sei que não vivemos em uma sociedade justa.
Não acredito neste Estado como um Estado democrático. Se acontecer {[}a
condenação{]}, ao menos, não vou me decepcionar neste sentido''.

\chapter{Censura e perseguição política\footnote{Matéria publicada em 05/09/2014, no jornal virtual \emph{O Progresso}.}}

\section*{Justiça proíbe ativista de vir a Dourados}

\subsection{Camila Jourdan participaria de evento que acontece na \versal{UFGD}, no
    entanto, a viagem da docente foi proibida}

    %publicada aonde?
 
Censura. É assim que a professora Camila Jourdan, doutora em Filosofia e
coordenadora de pós"-graduação na Universidade Estadual do Rio de Janeiro
(\versal{UERJ}), define a decisão judicial que a impediu de viajar ao Mato Grosso
do Sul para participar de um evento da Universidade Federal da Grande
Dourados (\versal{UFGD}), onde seria uma das palestrantes.

Na próxima sexta"-feira, a docente ingressaria no ``\versal{IV} Encontro de
Integração: Dias de História'', palestrando sobre o tema ``Jornadas de
Junho, Perseguições Políticas e Anarquismo Hoje''. Na oportunidade, ela
relataria as experiências que teve ao longo de sua carreira acadêmica e
de suas ações como ativista, incluindo todo o processo que acarretou em
sua prisão.

Durante a Copa do Mundo no Brasil ela acabou detida com outros ativistas
por protestos contra os gastos do Governo Federal com o Mundial. Ela
conseguiu habeas corpus e vai responder pelo processo em liberdade. Em
uma publicação no Facebook, Jourdan relatou que a justiça a proibiu de
viajar por considerar que ``a atividade de dar palestras não é essencial
ao exercício de sua atividade profissional''. ``Como assim?'',
questionou.

Ela alega estar sendo alvo de censura. ``Mais um absurdo sem
precedentes. Acabo de ser censurada. [\ldots{}] O que a sociedade
tem a dizer sobre isso? Esta censura precisa ser denunciada! Peço a
todos que denunciem isso em todos os meios que tiverem acesso. Querem
nos calar!''.

\medskip

\noindent
\subsection{Confira a postagem completa da professora}

\noindent
\emph{Mais um absurdo sem precedentes. Acabo de ser censurada. O
juiz responsável pelo caso indeferiu meu pedido de ir a Dourados"-\versal{MS}, dar
uma palestra no \versal{IV} Encontro de Integração: Dias de História, na \versal{UFGD}. A
justificativa do magistrado é que a atividade de dar palestras não é
essencial ao exercício da minha atividade profissional. Como assim?!? Eu
sou uma professora universitária, o programa de pós"-graduação em
Filosofia da \versal{UERJ} é avaliado inclusive levando em conta minha
produtividade acadêmica\ldots{} Além disso, houve financiamento público
para possibilitar minha ida, a decisão do juíz, dois dias antes da
palestra, com hospedagem paga, passagem, refeições, gera prejuízo ao
dinheiro público. O que a sociedade tem a dizer sobre isso? Esta censura
precisa ser denunciada! Peço a todos que denunciem isso em todos os
meios que tiverem acesso.}

\chapter{Anarquismo, sistema prisional e crise da representação\footnote{Entrevista concedida ao \emph{Le monde diplomatique}, em 24/10/2014.}}

\versal{LE MONDE}: \emph{Junho e os meses na sequência deram visibilidade a uma série
de movimentos de lutas diretas organizadas por movimentos
autogestionários, horizontais e, portanto, desvinculados de projetos
partidários. Em contrapartida essas organizações foram duramente
criticadas pela esquerda tradicional. Por que isso acontece? E em que
medida isso colabora com a criminalização destes movimentos?}

\noindent\versal {CAMILA JOURDAN}: Creio que isso
acontece porque a esquerda tradicional viu que a grande
crise de representatividade que junho [2013] trouxe à tona na nossa
sociedade, a grande recusa da população aos métodos institucionais e ao
burocratismo da esquerda partidária, bem como a demanda por participação
política direta, a levaram a perder espaço, a perder o que para eles é
um nicho de mercado eleitoral, que canaliza a luta política da sociedade
para os fins da disputa nas urnas. Partidos de esquerda lançaram notas
se desvinculando e criticando a ação dos anarquistas, alguns ajudaram
mesmo a criminalizar e entregar pessoas para a polícia.

De fato, toda
luta não cooptável, não assimilável institucionalmente é transformada
sistematicamente em crime, em quadrilha. Há uma disputa por terreno,
para canalizar a luta concreta, direta, da população para fins
eleitorais, para os fins que são controláveis, e nisso esquerda, direita
e agentes do Estado revelam seu ponto comum. Desde o policial e do
político, passando por intelectuais acadêmicos, pela mídia oficial, pelo
inquérito, todos devem compartilhar as mesmas conclusões: ``irracionais,
bandidos, anarquistas, baderneiros, vândalos, arruaceiros, desprovidos
de ideologia\ldots{}'' palavras que têm efeitos muito concretos, pois privam
a ação coletiva de toda credibilidade política ao mesmo tempo que
legitimam e estimulam a repressão policial e a perseguição política. De
fato, eu tenho cada vez mais visto que este processo de criminalização e
perseguição se coloca contra a auto"-organização da sociedade, é um
processo do Estado, suas instituições e seus agentes, contra a
sociedade.

\medskip

\noindent\emph{Como você observa a construção da sua imagem e destes
movimentos como inimigos da sociedade?}

\noindent Então, há um esforço muito grande para transformar toda auto"-organização
da sociedade em quadrilha e, nesta medida, os anarquistas são tomados
como os maiores inimigos, justamente pelo seu caráter não cooptável pela
máquina do Estado. E nós somos sim inimigos do Estado, mas não inimigos
da sociedade. Esta diferenciação é muito importante, a sociedade não é o
Estado. De fato, o Estado se coloca no direito de dar uma unidade
abstrata à sociedade, nem que, para isso, se volte contra a sua
auto"-organização concreta. Há um grande esforço no inquérito para
transformar organizações políticas não"-eleitorais em quadrilhas.
Trata"-se de um esforço discursivo, interpretativo das nossas práticas e
ideologias. E o que se pretende recusar com isso é o direito à
auto"-organização da sociedade sem ser para fins eleitorais. O inquérito
chega a dizer que ``a organização não eleitoral se afasta do viés
político"-ideológico legítimo em nosso sistema democrático''. Então, há
um viés político"-ideológico que não é legítimo, e este é,
principalmente, o viés anarquista. Não é exagero quando dizemos que o
anarquismo e, mais ainda, toda organização direta da sociedade civil é o
que se pretende impedir, é o que será transformado em crime, se formos
de fato condenados. Tanto é assim que um grupo de educação popular, que
organiza alfabetização para jovens e adultos e pré"-vestibulares, é
identificado como ``uma iniciativa louvável dos anarquistas, mas que é
na verdade uma fachada para treinar jovens para a luta armada''. Existe
um construto hermenêutico capaz de fazer organizações sociais e
políticas serem lidas como quadrilhas armadas. E nos é fundamental
pensar quais os passos que permitem esta passagem.

Sobre os anarquistas, o inquérito é textual ao dizer que ``a
delinquência política de viés anarquista é a mais insidiosa e a que
precisa ser mais fortemente combatida. Ela é ideológica, age de modo
dissimulado e sorrateiro, instrumentaliza os demais agentes violentos,
infiltra"-se e coopta movimentos sociais, apodera"-se dos focos de
insatisfação difusos na sociedade para manipulá"-los segundo as
conveniências de seu interesse político''.

O anarquismo é tomado como pior do que a violência como fim em si, que por sua vez é
tomado como pior e mais perigoso do que o uso da violência para
conseguir algo pontual, que pode até ser atingido, contemplado,
negociável. Por que devemos ser fortemente combatidos? A insatisfação
sistêmica, a recusa dos meios tradicionais, a impossibilidade de se
negociar, tudo isso contribui para que o anarquismo, enquanto proposta
de organização direta da sociedade, seja fortemente criminalizado. Para
mim está claro que o que amedronta é nossa horizontalidade e o nosso
caráter não"-institucional. Afinal, é todo um modo de vida que se recusa,
não há um reformismo, uma reivindicação pontual que pudesse ser
simplesmente atendida, mantendo"-se toda a estrutura como está. Então não
há como deter tal insatisfação sistêmica. Isso, que foi em tantos
momentos razão para que o movimento fosse acusado de falta de foco, de
utopia, é sua força e sua identidade. O que nos leva a ser
criminalizados é estarmos demandando modificações reais, radicais, que
não podem ser concedidas por uma reforma, sem uma mudança de baixo para
cima igualmente concreta.

Então o que preocupa o sistema não é a
delinquência vazia, não é não sabermos pelo que lutamos, como se repete
no discurso midiático, é justamente o que temos de mais consciente, mais
ideológico e que não é passível de ser atingido ou corrompido. E se
somos nós, os anarquistas, aqueles que devem ser mais fortemente
combatidos, como dizer que isso não é político"-ideológico, ou mesmo que
demandamos mudanças impossíveis? Não se persegue mais fortemente um
grupo político do que supostos criminosos comuns se não se toma tal
posição política como uma clara ameaça em si mesma; não para a
sociedade, que é impedida de se auto"-organizar, mas pela estrutura
estatal que parasita atualmente esta sociedade. Fato é que podemos criar
uma nova sociedade sob fundamentos distintos da atual. É a sociedade
organizada que tem legitimidade para isso, acima de qualquer coisa,
mesmo que a repressão estatal insista em nos criminalizar, é da
sociedade que emana todo poder e isso é fundamento inclusive do que se
entende por democracia, embora não vivamos em uma sociedade realmente
democrática, já que o poder do povo é contraditório com o poder do
capital.

\medskip

\noindent\emph{De todas as prisões durante os protestos, apenas Rafael
Braga Vieira foi condenado e cumpre pena de prisão. Como você observa
esta situação?}

\noindent A condenação de Rafael Braga foi mais um capítulo da reação ao que
significou junho do ano passado, um capítulo com uma mensagem clara:
poder para o povo não haverá. Não há dúvida de que jamais houve
democracia real, de que a maior parte da população, negra e pobre, vive
sob a opressão da violência do Estado, da intervenção militar, dos
grupos de extermínio, da milícia, da arbitrariedade da instituição
absolutista que é a polícia (que se autolegisla, julga e executa). No
período da Ditadura Militar, a tortura e a violência policial foram
``democratizadas'' com a chamada ``classe média'', isso deu visibilidade à
violência do Estado. Mas quando a supressão de direitos volta a ficar
restrita às camadas excluídas, finge"-se que ela não existe.

Hoje, quando
a violência do Estado atinge as elites ou a chamada classe média,
chama"-se ``Estado de exceção''. Para falarmos em Estado de exceção seria
necessário dizer que vivemos em uma democracia, mas este não é o caso.
Esta não é, nem jamais foi, a exceção nas favelas. Nós não lutamos
contra a suposta exceção, nós pretendemos combater a regra, isto é, este
sistema falsamente democrático. O Estado se recrudesce, mostra
sua cara e se volta mesmo contra as camadas incluídas da população
sempre que estas apoiam a luta dos excluídos, lutam ao lado destes e
denunciam a violência contra eles. O Estado sempre estará pronto para
punir exemplarmente quando este for o caso. Mas o importante é que
permaneça um silêncio conivente da maior parte da sociedade ao
terrorismo imposto pelos agentes estatais, e do poder econômico que estes
representam, às camadas excluídas.

É extremamente sintomático desta
situação e expressivo desta sociedade em que vivemos que, quando o povo foi
às ruas exigindo poder ao povo, quem tenha permanecido preso e
condenado seja negro, pobre e morador de rua. É a resposta punitiva que
o Estado quer dar, ressaltando que, aconteça o que acontecer, o Estado e
o poder permanecerão nas mãos das elites e não do povo. Nós estamos
sendo criminalizados por defender o poder do povo. Sim, nós fomos
perseguidos, presos, estamos impedidos de nos manifestar e talvez
sejamos condenados, esta é a exceção. Toda sorte de arbitrariedade e
manipulação foi usada contra nós para que isso fosse possível. Mas
existem muitos a quem a voz é permanentemente negada de muitos modos. Que
quem permaneça encarcerado seja hoje um morador de rua é a regra, quando
o que se pretende evitar antes de tudo é o poder do povo. Nós lutamos
pela igualdade, em princípio, sabendo da desigualdade que há, de fato.
Para que nossa luta não signifique tão somente nossa criminalização,
nosso próprio silenciamento e banimento, é preciso que ela continue
sendo a luta da sociedade em geral.

\medskip

\noindent\emph{Você acredita que todo preso é um preso político ou existe
diferença entre a criminalização de uma atividade política e das camadas
mais pobres da sociedade?}

\noindent Debatemos isso ontem na Faculdade Nacional de Direito. Existe, claro,
uma distinção jurídica relativa à motivação, se a motivação do crime foi
ou não política. Mas eu creio que temos que perguntar antes de qualquer
coisa ``o que é política?'', até para saber se podemos distinguir tão
claramente o que são motivações políticas. Em uma sociedade desigual e
excludente como a nossa, é evidente que toda prisão tem uma irredutível
dimensão política. Mesmo que ao sujeito preso não possa ser atribuído
motivações conscientemente políticas, o que o leva a ser preso e assim
permanecer são motivações relativas às exclusões inerentes à nossa
sociedade. E, em grande medida, o que o leva a cometer o crime, também,
já que somos seres sociais. Quem é preso de fato? Quem atenta contra a
propriedade privada, mais de 80\% de negros e mulheres das favelas. Como
dizer que isso não é político? A prisão é uma arma da exclusão social e
é também um grande comércio. O criminoso padrão precisa ser fabricado
pelo sistema carcerário como tal, até para manter a população com medo e
justificar a existência dos agentes armados do Estado. Eu me pergunto,
em que sentido do termo ``político'' isso não seria político?

O contraexemplo de plantão em uma discussão como esta é o do estupro. As
pessoas perguntam: e o estupro, também é político? Primeiro, que a maior
parte da população carcerária não é composta por estupradores, mas sim,
como já disse, por pessoas que atentam em algum sentido contra a
propriedade privada. Depois, existem muitos modos de se lidar com o
estupro. O estupro não é algo natural às sociedades humanas, capaz então
de, para ser banido, justificar nosso sistema carcerário. Nem é nosso
sistema carcerário a única maneira de se lidar com condutas com estas,
tampouco a mais eficaz. O problema é que nós temos uma cultura do
estupro, que naturaliza a objetificação da mulher, que culpabiliza a
vítima. É uma cultura permissiva ao estupro, permissiva a toda violência
contra a mulher, das quais o estupro é a forma mais aguda, mas nós
vivemos em uma sociedade na qual as mulheres são abusadas diariamente
nos transportes públicos, expostas nos jornais, humilhadas na \versal{TV}, por
isso somos uma sociedade na qual o estupro faz sentido relativo, como
ação aceitável. A partir disso, usar o estupro como um crime inevitável
e que, não sendo político, legitima o sistema carcerário, é má intenção
argumentativa. O problema é que o sistema carcerário foi naturalizado
como única maneira de se lidar com o crime, qualquer que seja, quando,
de fato, ele mesmo é o grande responsável por fabricar os crimes que
supostamente deveria combater e também por instituir uma série de
condutas outras, tanto ou mais execráveis do que aquelas condutas que
supostamente deveria punir. Fato é, em todo caso, que o encarceramento
não é desde sempre o modo padrão de se lidar com o crime, isso é algo
que surge historicamente, na modernidade, e que também pode ser
modificado, o problema é que atualmente as pessoas não veem outras
alternativas. Quem analisa isso muito bem é Foucault, no \emph{Vigiar e
Punir}. Lugares como Bangu não deveriam existir, nada justifica que um
ser humano possa dispor da existência de outro ser humano daquela
maneira, apenas um grande terrorismo psicológico e muita manipulação de
informação permitem que a sociedade continue legitimando a existência de
espaços como aquele. E isso é totalmente político.

\chapter{Não existe governo de esquerda\footnote{Entrevista concedida ao \emph{Diário do Centro do Mundo}, em
31/10/2016.}}

\versal{DIÁRIO DO CENTRO DO MUNDO}: \emph{Há quantas eleições você não vota e por quê?}

\noindent\versal{CAMILA JOURDAN}: Como sou anarquista
desde muito jovem, meu posicionamento sempre foi o
voto nulo. Admito que na primeira eleição que o Lula venceu, eu também
votei nele, não que tivesse deixado de acreditar no mesmo que ainda
acredito hoje, mas como não estava muito ativa politicamente naquele
momento, achei na época que era o máximo que podia fazer. Mas não me
decepcionei totalmente, pois já sabia das limitações da via
institucional. Mais importante do que o ``não vote'' é, sem dúvida alguma,
o lute, o organize"-se. Os anarquistas defendem o ``não voto'' ou o voto
nulo como ação política refletida, é uma consideração sobre a
impossibilidade da via institucional trazer as mudanças que buscamos e,
ao mesmo tempo, sobre o equívoco envolvido no peso que se coloca nesta
disputa. Porque a eleição canaliza as vias de ações políticas concretas
e faz parecer que a participação política democrática se resume a votar.
Esta canalização é extremamente nociva. Se pensarmos o que houve nas
últimas décadas no país, veremos que a chegada de um partido de esquerda
ao poder não fortaleceu a esquerda, mas a fez recuar nos espaços de luta
concreta e organização. Foi isso que ocorreu com o \versal{MST}, por exemplo, que
recuou a luta no campo com o \versal{PT} ocupando a presidência. Foi isso que
ocorreu também recentemente com as greves da educação em 2016, que foram
entregues para que os partidos que aparelham os sindicatos pudessem se
dedicar melhor à campanha eleitoral. E estou dizendo isso para citar
dois exemplos apenas.

O que ocorre no geral com as eleições é uma
inversão dos meios pelos fins, ganhar a disputa se torna um fim em si,
e, com isso, se perde aquilo que é de fato importante. Tenta"-se alcançar
o poder, quase sem se notar que este mesmo poder, nos moldes em que se
encontra, é incapaz de gerar as mudanças estruturais que desejamos e que
só poderá ser usado em favor das classes e elites dominantes. Então,
para alcançar o poder, pela disputa, o partido, o candidato de esquerda
se transforma naquilo mesmo que pretendia combater, não por um problema
de princípios particularmente deste ou daquele, mas por uma questão
estrutural. O que aconteceu com o \versal{PT} não é próprio ao \versal{PT}, é inevitável,
é meio fatalista dizer isso, mas basta fazer as contas, o \versal{PT} de hoje é
o \versal{PSOL} de amanhã. E esta história eleitoral se repetirá assim
indefinidamente porque é preciso ter um partido para canalizar um
``público alvo'' que de outro modo poderia realmente se tornar perigoso
ao sistema.

Daí que esvaziar o sentido da via eleitoral, e da via
institucional no geral, é uma ação política extremamente importante. Não
há superação do paradigma da representação sem esvaziamento deste
paradigma. Estes partidos canalizam um ``nicho de mercado'', eleição é
mercado, é sociedade de consumo dominando a atuação política e
tornando"-a controlável, vendível. Vence quem é vendível, e o que é
vendível já está dentro da lógica dominante. Vende"-se um produto, uma
imagem, pois isto que é a representação, nada além de uma imagem. A
política concreta é outra coisa. Obviamente é uma imagem que não
representa nada, pois não existem mecanismos de consulta e participação
direta, só existiria representação de algo se houvesse um âmbito de
apresentação direta da sociedade. Mas as pessoas em geral simplesmente
não atuam politicamente, então eleição é espetáculo, no sentido de Guy
Debord mesmo, é uma representação sem representado. O espetáculo é
diametralmente oposto à ação direta, é a doença da representação porque
é a representação sem representado, mas não existe paradigma da
representação sem a sua doença, isto é, sem o espetáculo, sem as imagens
valendo mais que a realidade concreta. É preciso notar ainda que o
discurso eleitoral tem que ser um discurso de apaziguamento de classes
porque se trata de ganhar a opinião pública tal como aí está, com todo o
senso comum manipulado pela discurso dominante, e eleição não é
formadora, não é educativa, não é ``trabalho de base'', o político não
educa o eleitorado, ele quer ganhá"-lo com todos os seus preconceitos,
quer convencê"-lo, quer se vender como um produto no mercado. Para isso,
ele vai necessariamente recuar. O medo de perder voto faz com que os
candidatos sejam nivelados com poucas diferenças, o que difere é só uma
imagem superficial, jamais a prática concreta que é determinada por
outros fatores. O próprio discurso vai sendo esvaziado, até que os
candidatos todos se parecem, porque eles querem agradar o mesmo público.
Devem, portanto, parecer inócuos e, acima de tudo, para governar,
precisam fazer alianças e responder aos que realmente controlam as
instituições, não ao povo.

\medskip

\noindent\emph{Em algum momento pode ser necessário votar para tentar evitar o
pior? Digamos Donald Trump, fanáticos religiosos\ldots{}}

\noindent Eu realmente entendo quem tem este medo, é um equívoco fácil de se
cometer. Creio que esta ideia deriva em máximo grau ainda do peso que
as pessoas colocam no processo eleitoral e na via institucional. Mas
existem lutas concretas acontecendo todos os dias, a troca de políticos
ocupando cargos e o parlamento não é toda a vida política de uma
sociedade, e não é o mais importante, fundamentalmente não é o que faz a
diferença. Não foi por meio disso que algum direito foi conquistado,
nenhum salvador deu algum direito de presente ao povo, esta é uma ilusão
muito nociva. Ao lado disso, há o discurso do medo: temos que votar em
tal candidato porque de outro modo algo terrível vai acontecer. Este
discurso é feito pelos dois lados, sempre. ``Tenham medo, votem em alguém
para evitar uma catástrofe. Tenham medo, escolham um senhor para
proteção''. Ora, as coisas já estão péssimas, muito terríveis mesmo. Não é
o voto que vai evitar uma catástrofe maior, a catástrofe está posta, ela
é conjuntural, ela é a fase atual do capitalismo, ninguém vai nos
salvar, é preciso colocar peso nas lutas que acreditamos, e é preciso
não ter medo também para focar no que pode mudar as coisas realmente.

\medskip

\noindent\emph{A descrença na democracia representativa por parte da esquerda
não contribui para que a direita vença nas urnas com mais facilidade?}

\noindent É a acusação que nós mais sofremos. ``Se vocês votassem, nós ganharíamos
as eleições''. Nós quem?! E aqui eu gostaria de dizer: ``a César o que é
de César! Quando vocês ganham as eleições, já não são mais de esquerda''.
Sabe aquela famosa citação do Deleuze? ``Não existe governo de esquerda
porque a esquerda não tem nada a ver com ser governo''. Eles acham que,
se votássemos, a esquerda ganhava. Nós achamos que se a esquerda
institucional deixasse de gastar tanto tempo, energia e dinheiro
tentando ganhar e legitimar este processo, se não entregasse todas as lutas
de base, todas as greves e seus próprios princípios tentando ganhar isso
(que já está perdido de saída), e se investisse este tempo, esta energia
e este dinheiro na luta concreta e na organização popular, na criação de
comunas autônomas, não haveria direita que conseguisse nos governar.
Pois é claro que se pode ganhar e não levar, pois as lutas são diárias,
são concretas, são nos espaços de base de construção da sociedade. E
tanto mais fortes quanto menos institucionais.

\medskip

\noindent\emph{Um governo mais à esquerda não pode ajudar a fortalecer a luta
social?}

\noindent Eu acho que de certa forma já respondi isso. Pode inclusive enfraquecer,
como já ocorreu com a chegada do \versal{PT} ao poder em vários aspectos, porque a
função do \versal{PT} para as elites era justamente conter as lutas sociais por
dentro, levando a luta para algo palatável, aceitável, negociável. Se se quer
parar um partido de esquerda com profunda inserção social, torne"-o
governo. Foi esta fórmula que foi usada no caso \versal{PT}, porque a partir de
então ele saiu da oposição e teve que responder aos esquemas postos
dentro da máquina estatal. E em face disso, há um limite muito grande
para o que se pode fazer, pois não se pode desestabilizar o governo.
Manter a máquina funcionando e a satisfação equilibrada das diversas
forças políticas que ocupam lugares de poder na sociedade exige cautela.

Assim, o funcionamento da máquina e a manutenção das alianças para o
próximo processo eleitoral engessam e se tornam um fim em si. Entendemos
o governo como um parasita: você não se mobiliza para ocupar o lugar de
um parasita, você se mobiliza para acabar com ele. Isso não significa
esperar que piore e achar que ``quanto pior, melhor'' porque,
supostamente, isso poderia levar as pessoas a se revoltarem mais. Isso é
um discurso privilegiado, quanto pior, pior mesmo, nossa questão aqui é
sobre o que realmente pode fazer melhorar. Tivemos 14 anos de um governo
dito de esquerda e isso não fortaleceu a luta social. O \versal{PT} foi terrível
para a luta no campo, não fez sequer a reforma agrária prometida, foi
terrível também para os indígenas, para quilombolas, engessou os
sindicatos nos quais tinha inserção, apoiou as \versal{UPP}s nas favelas,
paralisou o \versal{MST}, criminalizou os movimentos sociais, inclusive assinando
a lei anti"-terrorismo\ldots{} Serviu sim para calar os movimentos sociais e
colocá"-los à serviço de um projeto de manutenção do poder como um fim em
si mesmo.

\medskip

\noindent\emph{Seria impossível governar contra os interesses do chamado 1\%,
a favor de 99\%?}

\noindent Vamos pensar como seria isso. Primeiramente esta pessoa teria que se
eleger e, portanto, sua companha teria que ser financiada por quem tem
dinheiro, no geral, grandes corporações que investem no processo
eleitoral como modo de manterem"-se exercendo o poder. Mas, digamos que
houvesse um candidato que não fosse assim financiado não sendo engessado
pelos mantenedores do sistema. Ainda assim, ele teria que agradar a
opinião pública manipulada pelo monopólio dos meios de comunicação que
servem à classe dominante. Candidatos de esquerda ``paz e amor'' jurando
respeito à sacrossanta propriedade privada e dizendo que vão ``governar
para todos'' não são acasos. Mas digamos que ele simplesmente não
dissesse a verdade e pretendesse, mesmo, após se eleger, realmente
colocar pouco a pouco em curso uma política contrária aos interesses da
classe dominante. Bom, ainda haveriam as alianças, os esquemas, toda a
estrutura corrompida na qual ele estaria inserido e em relação a qual
precisaria responder e ficaria amarrado. Por outro lado, digamos que
ainda assim ele representasse em algum momento uma perda real para os
banqueiros e aqueles que detém o grande capital. Houve um outro momento
histórico no qual isso realmente ocorreu, podemos lembrar aqui de
Salvador Allende. Você acredita que eles diriam o que? ``Ah, ok, vamos
aceitar nosso prejuízo porque afinal ele foi eleito por um sistema
democrático''? Disseram isso para Allende? Obvio que não, tal personagem
imaginário seria deposto ou morto. Não existe real democracia, os donos
reais do poder não têm qualquer problema em usar a força e suspender a
aparência de Estado democrático sempre que é necessário, usando todos os
meios necessários para isso, a exceção é regra na nossa sociedade, a
aparência de democracia serve apenas para manter os 99\% acreditando que
têm real participação política. Votar é legitimar isso, é assinar
embaixo desta farsa. A tragédia do \versal{PT} encena, do particular para o
geral, a tragédia da via institucional, ser vendido, corrompido,
esvaziado e depois jogado fora por não servir mais aos interesses
dominantes. Não precisamos encenar esta tragédia novamente.

\medskip

\noindent\emph{Quais ações políticas considera mais importantes que o voto
nesse momento?}

\noindent As ações políticas concretas que considero importantes, mais importantes
que o não"-voto, são as ações de auto"-organização coletivas nas células
da sociedade e as ações de mobilização. Isso inclui as ocupações de
escolas, as greves levadas pelas bases das categorias, as manifestações
de rua, as assembleias de bairro, a criação de espaços autônomos e a
criação de redes de apoio mútuo entre estes espaços. Tratam"-se de ações
que carregam os princípios da sociedade que defendemos, que não esperam
que alguém faça por nós; pressionam o governo também, mas pressionam pela
ação direta, pelo já fazer e mostrar que outro modo de vida é possível.
Não se trata de esperar a sociedade perfeita, mas pela auto"-organização
coletiva trazer melhoras para a vida das pessoas aqui e agora, ocupando
um prédio e gerando moradia popular, por exemplo, impedindo um aumento
das passagens através de manifestações de rua. Quando eu digo que existe
luta todo dia, não estou exagerando, todo dia estão removendo famílias,
e existem resistências, todo dia a guarda está proibindo camelô de
trabalhar e existem resistências, as favelas estão aí resistindo também,
existe muita luta acontecendo na sociedade, no dia a dia, no micro, as
pessoas podem atuar a partir dos espaços nos quais estão inseridas,
podem ser agentes das resistências, podem ser fomentadores das
resistências a partir de baixo, podem ajudar a construir um outro modo
de vida sem precisar reproduzir de novo e de novo o espetáculo dos de
cima.

\medskip

\noindent\emph{E a longo prazo?}

\noindent Acredito na educação libertária como arma na modificação da sociedade.
Claro que não sem a construção de espaços verdadeiramente autônomos e a
possibilidade de autogestão na produção e reprodução da vida. Eu diria
que a educação libertária é necessária, mas não suficiente. Sem ela não
temos sobre o que basear outros modos de relação, e este processo é contínuo
e longo, é ele que forma as bases de um outro modo de vida. Pode começar
onde você estiver porque a educação não se dá apenas nas instituições de
ensino, se dá para muito além delas, e não é passível de ser destruída
facilmente. Temos muitas dificuldades neste sentido porque a educação é
um aparelho ideológico do capitalismo hoje em dia, os meios de
comunicação e as instituições são verdadeiros monopólios neste sentido,
mas importantes experiências de resistência existem também. Não é por
acaso que a luta da educação é tão forte no mundo todo atualmente, e que
a educação e nós professores sofremos tantos golpes do Estado. Mas acho
que este seria o tema para outra entrevista.

\medskip

\noindent\emph{Há tentativas de mandatos representativos coletivos pelo país,
como um coletivo anarquista eleito para uma vaga na câmara de
vereadores. Acha uma boa ideia?}

\noindent Eu não sei detalhes sobre isso, mas toda a ideia soa muito \emph{fake},
me lembra aquelas mercadorias industriais com um selo de ``feito à mão''.
O que estou querendo dizer é o seguinte: me parece outra tentativa de
tragar o que está fora do sistema para mais um objeto de consumo no
processo eleitoral. O capitalismo é muito bom nisso, ele mata e depois
vende. O que você quiser, o capitalismo pode te vender, menos o que não
é vendível. Então, é como se ele procurasse o que pode negá"-lo,
construisse um \emph{fake} palatável e vendesse no mercado. E isso é um
modo de esvaziar o sentido do que realmente poderia negá"-lo. O mercado
pode vender tudo, menos o que nega o mercado, menos a igualdade social,
por exemplo. O processo eleitoral, como uma instância do mercado, também
é assim, portanto tenta tragar de modo espetacular para dentro de si o
que o nega e se coloca como fora dele. Mas só pode fazê"-lo, claro, por
meio de uma \emph{fake}, por meio de uma imagem espetacular daquilo que
o recusa.

O que eu gostaria de dizer sobre isso é: nós anarquistas não
somos mais uma opção vendível no sistema representativo, não existe
gestão coletiva dentro da câmara, isso não é capaz de tornar este
sistema menos ilegítimo, somos o \emph{totalmente outro} deste sistema,
não estamos aqui para disputá"-lo, estamos aqui para tensionar as suas
estruturas e fazê"-las ruir.

\medskip

\noindent\emph{Acredita que a maioria das pessoas não foi votar por estar
engajada politicamente de outras maneiras, ou por comodismo e
passividade?}

\noindent Não estão engajadas politicamente, mas a maioria das pessoas não vota por
uma descrença generalizada nos políticos. Esta descrença, embora não
seja teoricamente fundamentada, tem um significado político, vem
aumentando e não é apenas um fenômeno brasileiro. Não se trata de uma
maioria reacionária manipulada pela televisão. As pessoas em questão têm
posições misturadas, não são completamente coerentes, não estão no geral
acostumadas a atuar politicamente, mas isso não significa que a
insatisfação que possuem seja menos legítima. Também não creio que seja
comodidade, votar é mais cômodo do que ter que justificar ou pagar
multa. Ocorre que há uma crescente descrença e insatisfação com o
sistema representativo, decorrente da sua impossibilidade de promover
mudanças reais, o que é facilmente constatado pelas pessoas,
principalmente depois da chegada da social"-democracia ao poder. Acredito
que a outra campanha, a campanha não"-eleitoral, tenha como principal
tarefa aprofundar esta reflexão, propagar outras formas de atuação, não
podemos ficar só no ``não vote''.

\part{ANÁLISES}

\chapter{O que houve afinal em 2013?}

Em 2013 vivemos no Brasil um levante popular, uma insurreição, como
tantas que ocorreram nos anos anteriores em vários lugares do mundo:
Wall Street, Grécia em 2008, Seattle. As características destas
insurreições populares são, em geral, a horizontalidade organizacional;
a recusa à via institucional e ao reformismo da esquerda partidária; uma
revalorização explícita do anarquismo e dos valores historicamente
associados à tradição libertária, dentre os quais destacam"-se a busca
pela participação política direta; a recusa às hierarquias e a recusa ao
paradigma representacional.

Entender 2013 é entender o nosso tempo e é fundamental que possamos
contar nossa própria história. Atualmente podemos identificar ainda
algumas linhas de leitura em disputas discursivas sobre o que significou
2013\footnote{A identificação destas linhas de leitura sobre 2013 se
  coaduna à análise avançada pelo prof. Wallace Moraes sobre o tema,
  ainda não publicada.}. A primeira delas é a defendida pelo \versal{PT} e seus
aliados, que consideram 2013 um movimento
fundamentalmente manipulado pela direita, que serviu para preparar o
golpe contra o \versal{PT} e, por isso, o avalia negativamente. A segunda é a
defendida pelos ultraliberais, que procuram esvaziar o
significado político dos atos de rua associando"-os com ``puro
vandalismo vazio'' ou mesmo com ``ações terroristas'', alguns dizendo mesmo que esses vândalos eram pagos pelo próprio \versal{PT}. Ao lado das duas
leituras desqualificantes existem duas avaliações possíveis ligadas à
esquerda partidária. Uma delas é positiva, porque considera que a
crítica ao governo do \versal{PT} aumenta as chances de vitória eleitoral ou
crescimento por partidos de esquerda, mas se torna majoritariamente
negativa na medida em que estes partidos não conseguiram dirigir o
processo, tornando claro o afastamento deles da população. Há ainda
grupelhos fascistas que, criticando também os ataques das ruas às
instituições e ao capital, advogam via 2013 para pedir intervenção
militar. A leitura que aqui avançamos se afasta e se contrapõe a todas
estas avaliações e se aproxima de uma leitura insurrecionária de 2013.

Em 2013 milhares foram às ruas, em todo o país, exigindo reais
transformações sociais. Vimos se espalhar a rebeldia, a indignação, a
revolta, o ódio ao Estado opressor, a luta por saúde, moradia e
educação, o confronto direto com o capitalismo, com o monopólio dos
transportes públicos, o ataque a bancos e a resistência aos agentes
do Estado e demais órgãos da repressão. Esse também foi, e talvez
fundamentalmente, o ano do surgimento da tática black bloc no
Brasil. A tática ajudou a dar voz aos protestos nas ruas, expressando uma
crítica radical ao sistema e fortalecendo sua capacidade de resistir aos
ataques da polícia à população. Já famosa em vários
lugares do mundo, a tática que surge aqui em meio aos protestos de
junho possibilitou que os corpos, diariamente jogados uns contra os
outros pela máquina do mercado, se encontrassem igualados nas ruas,
unidos para responder à violência inerente ao cotidiano das cidades e
fundamental para a manutenção dessa sociedade desigual. Aprender a
resistir, desafiar o monopólio da força destrutiva estatal e lembrar ao
próprio povo e ao Estado que o oprime de onde deriva o poder. E foram
milhares de jovens (ou nem tão jovens assim), usando escudos
improvisados, máscaras, ou o que encontravam pela frente para
resistir à violência policial. O levante que se deu em junho contou com
a presença popular maciça; pessoas que jamais haviam ido em
manifestações; moradores de rua; negros das periferias das grandes
cidades; feministas; gays; lésbicas\ldots{} Não foi um movimento da classe
média branca, como se pretendeu estabelecer na grande mídia. Também não
foi um movimento composto em sua maioria por ``pessoas alienadas'' que
não sabiam pelo que lutavam, como também foi afirmado. Junho de
2013 não foi ainda ``o início do golpe'', como quer fazer crer os
aparelhos de reprodução de hegemonia da esquerda partidária. As pessoas
sabem muito bem o que as oprime e é sempre bom ressaltar que nenhum
intelectual esclarecido precisa contar isso para elas.

O alvo da revolta popular eram os agentes da sua opressão diária:
ônibus; agências de bancos; palácios dos poderes; assembleias
legislativas; veículos do monopólio da mídia manipuladora; viaturas
policiais. Faz muito tempo que a favela desce quando a polícia mata uma
criança, e que o povo queima ônibus e trens quando o transporte, já
precarizado no geral, quebra justamente na hora da volta pra casa.
Ninguém precisa ensinar a revolta pra ninguém.

Mas o que encontrou"-se desta vez foi a visibilidade do asfalto. Não era
possível dizer que aquilo estava sendo orquestrado por traficantes, não
era possível negar a dimensão política da revolta, uma grande camada da
população estava lá, estava vendo. A potência de junho foi a do encontro
da visibilidade do asfalto --- onde as balas são, na maioria das vezes, de
borracha --- com uma certa democratização da violência de Estado para
setores da população que não estavam acostumados a sofrê"-la. Não se
tratava de uma ação orquestrada, e toda tentativa de gerar uma pauta
única convergente, esvaziada politicamente e que apaziguasse a
luta de classes, foi recusada pelas ruas. Nesse sentido, a multiplicidade
de pautas e o caráter difuso foi mais uma força do movimento, era todo
um modo de vida que se recusava. Não que os inimigos não fossem
concretos e identificáveis, mas não havia um reformismo, uma
reivindicação pontual que pudesse ser simplesmente atendida, mantendo"-se
toda a estrutura como estava e, assim, parando o levante. Isso,
que foi razão em tantos momentos para que o movimento fosse acusado de
utopista, sem foco, era sua força e sua identidade. Talvez pela
primeira vez estávamos demandando modificações reais e radicais, que não
ocorreriam sem mudanças estruturais e concretas. E a quebra do monopólio da informação possibilitada pela internet
permitiu que as imagens fossem mostradas diretamente. O povo, a
sociedade, assumiu o papel de sujeito histórico, a população participou
da edição da história, não foi somente espectadora.

A internet, obviamente, não faz movimento social. Ao contrário, ela surge
como mais um modo de controle e de comércio, mas ela pode ser
apropriada, pode ser também um instrumento de luta. A comunicação
foi estabelecida em rede e se espalhou exponencialmente. As mídias
digitais e as redes sociais serviram também para desmentir a constante
desinformação da imprensa burguesa. Não podemos deixar de notar a
importância do desenvolvimento das tecnologias ainda não totalmente
controladas pelo Estado, que permitem a criação de territórios livres, de
zonas virtuais autônomas. E estas brechas abertas permitiram a passagem
da insatisfação popular generalizada pelo país.

Além disso, não podemos deixar de lembrar a participação dos
trabalhadores, das greves não institucionalizadas e tocadas pela base
independentemente das representações sindicais que marcaram 2013/2014.
Greves radicalizadas, tocadas pelos professores, pelos rodoviários e
pelos garis, pararam a cidade, unificaram demandas e foram fortemente
reprimidas e criminalizadas pelo Estado.

Mas a violência diária já estava instaurada antes. As
corporações capitalistas e os organismos financeiros, bem como o Estado
que representa tais corporações e serve para calar o povo,
impõem uma situação de guerra permanente. Recentemente, com os projetos
de cidades requeridos pelos megaeventos, a ofensiva ficou ainda mais
evidente: remoções; desalojos; fechamento de escolas; projetos de
pacificação nas favelas; chacinas; ``democratização'' de um pouco da
violência já permanente nas favelas e periferias para os centros urbanos
de classe média no asfalto; megaempreendimentos como Belo Monte; aumento
da violência também no campo; avanço sobre Terras Indígenas;
gentrificação em geral; aumento do custo de vida com incentivo à
manutenção do consumo, gerando um endividamento grandioso da população com os
juros gigantescos; tribunais de exceção; suspensão do direito à
manifestação; suspensão do direito de ir e vir\ldots{}

O Estado não tem nenhum problema com o uso da violência, ao contrário,
ele se arroga o direito ao seu monopólio. Se não fosse o caso, o que dizer de
Pinheirinho, das bases militares nas favelas, dos incêndios criminosos,
de Belo Monte, das prisões lotadas? Um episódio fundamental no Rio de
Janeiro foi a desocupação da Aldeia Maracanã. Naquele dia, após o uso
desmedido da violência policial, as pessoas atacaram com cocos os carros
da polícia em frente à \versal{ALERJ} --- a mesma \versal{ALERJ} que, um mês depois, seria
tomada pela população com pedras e paus. No contexto, um aumento de
passagem serviu como gota d'água para transbordar a insatisfação
popular geral, mas, anteriormente, a principal fagulha foi a desocupação da Aldeia
Maracanã e a luta de resistência que se seguiu. No Rio de
Janeiro não foi o Movimento Passe Livre que colocou milhões nas ruas,
de fato, nem mesmo em São Paulo foi (embora tenha convocado atos que
depois massificaram). Foi todo um contexto convergente, com razões que
já existiam antes, mas que se acirraram nessa ebulição social sem
precedentes na nossa sociedade. Pela primeira vez foi quebrada a
manipulação das oligarquias dominantes, diferentemente do que houve no
``fora Collor''. Pela primeira vez, talvez, estávamos demandando modificações reais, que não seriam possíveis com a estrutura atual.

Outros aspectos precisam ainda ser ressaltados: o que condicionou junho
e o que junho permitiu. Sobre o que condicionou junho, muito foi dito
sobre a conjuntura imediata do projeto de cidade excludente que, acirrada
pela proximidade dos megaeventos, disparou uma grande insatisfação
popular. Mas um outro aspecto fundamental é a chamada crise no sistema
de representação, diretamente relacionada com a chegada da esquerda
tradicional ao poder. A ascensão do \versal{PT} ao poder mostrou como é indiferente o resultado eleitoral, pois a via da eleição, comprometida com os interesses das grandes corporações, que financiam os
candidatos, encontra"-se arruinada de saída: para ganhar o jogo é
preciso estar vinculado aos interesses daqueles que financiam o jogo,
por isso não adianta trocar as peças, é preciso acabar com o jogo. A
perda de interesse na via indireta da representatividade se relaciona
diretamente com a constatação da sua incapacidade de modificar a
realidade. A grande desilusão que significou o \versal{PT} passa a servir como
paradigma para a busca de uma ação política que não vá repetir o trajeto
desse partido, isto é, que para chegar ao poder não se transforme, ao
menos no essencial, em uma instância daquilo que combatia. Lula defende
os interesses da classe dominante porque, para chegar ao poder, foi
financiado por esta classe no jogo eleitoral e, caso não o fosse, não chegaria ao poder. A
ideia de que agora ele pode governar para todos, apesar dos
interesses serem contraditórios, é obviamente falsa e serve ao
apaziguamento, à conciliação de classes. Quando o \versal{PT} diz que vai
governar para todos, em uma sociedade dominada pelo capital e na qual os
interesses do capital são contrários aos interesses do povo, ele (e
qualquer outro partido que faça o mesmo discurso para ganhar eleições)
já está escolhendo o lado do capital. Isso ficou imediatamente evidente
quando o \versal{PT}, um partido vindo de movimentos sociais, se voltou
completamente à disputa eleitoral e institucional, deixando claro os
limites da organização partidária e, também, deixando os movimentos
sociais órfãos ou destruídos por dentro, uma vez que, com a eleição do \versal{PT} ao
governo federal, abriu"-se uma nova possibilidade de desenvolvimento, pela base.

Tratou"-se, antes de tudo, de uma perda de confiança no sistema
representativo, que é própria do contexto histórico, e uma
insatisfação com os meios da esquerda institucional e o burocratismo dos
partidos eleitorais. É assim que se abre espaço para as propostas
anarquistas e a \emph{ação direta} nas ruas, mesmo que a maior parte da
população não pense exatamente nestes termos.

A partir disso, as posições defendidas pelos anarquistas, enquanto uma
alternativa real à esquerda institucional, obtiveram grande crescimento
e repercussão. A crise no modelo da representação, diretamente
relacionada com a chegada da esquerda partidária ao poder e à
constatação de que com isso nada mudou, mostra que aquilo que os
anarquistas sempre disseram estava correto: não adianta mudar as
peças se você não mudar o jogo, o próprio sistema impede qualquer
mudança, qualquer transformação, substancial, pela via institucional,
porque os meios que são usados transformam os fins aos quais se
pretendia chegar.

Esse não foi um acaso do \versal{PT}, os partidos reformistas,
necessariamente, evoluem no sentido de deixarem de ser partidos de
organização de massa, que pretendem organizar a luta dos trabalhadores,
para serem partidos eleitorais, de conciliação de classes, servindo,
portanto, à classe dominante pelo e para o próprio processo no qual
tomam parte. Eles se tornam, portanto, anti"-revolucionários, no sentido
que impedem que a luta se desenvolva para um conflito que constituiria
uma situação de transformação revolucionária real. A ação parlamentar
exige financiamento, daqueles que detém o capital, e alianças, com
aqueles que defendem os interesses da classe dominante. O processo
eleitoral é hoje uma grande briga de corporações empresariais, é mais um
objeto de consumo da nossa sociedade do espetáculo, e as pessoas sabem
que isso não pode trazer transformações reais. Donde o crescente número de
pessoas que não pretendem votar e as campanhas desesperadas do
governo para o evitar. O que o sistema eleitoral faz é tentar
canalizar toda a participação política de um indivíduo na sociedade a
qual pertence para uma votação em uma sigla. Só isso: você vai lá e vota
de dois em dois anos e isso, dizem, faz com que vivamos em uma
democracia. Ora, isso é obviamente falso, você transforma o cidadão em
um consumidor de candidatos, você cria uma sociedade alienada das
próprias decisões que constituem o seu modo de vida. Mesmo para aqueles
que não se dizem anarquistas, nunca se disseram, não defendem essa
posição, esta é uma verdade concreta do momento histórico que vivemos.
Por isso o anarquismo encontra condições para se desenvolver hoje, por
isso temos tantos ouvidos atentos.

Quando falamos de boicote eleitoral, é preciso entender qual a
perspectiva anarquista desse boicote. Por que não votar? Por que votar é
contraditório com a luta? Por que esta é uma ação concreta e não
meramente simbólica? A defesa do boicote eleitoral já era defendida por
Proudhon e Bakunin no século \versal{XIX}. A ideia básica envolvida também era a de
que um processo eleitoral dentro de um sistema dominado pela classe
burguesa necessariamente estaria comprometido com os interesses desta
classe e afastaria o proletariado da sua organização direta e,
consequentemente, da luta revolucionária. Em Bakunin, a recusa às
eleições, ao processo institucional é também uma recusa ao funcionamento
destas instituições e do Estado. Para os anarquistas, fazer revolução
não é conquistar o Estado, nem pela via eleitoral, nem pela violência. A
única luta revolucionária de fato seria aquela que permite aprofundar a
auto"-organização da sociedade que, combinada com lutas insurrecionais,
desembocaria na destruição do Estado. O boicote às eleições parte de uma
reflexão crítica em relação ao sentido da participação política indireta
no pleito representativo do parlamento burguês, tendo em vista a
conquista do Estado. Por isso, o boicote anarquista não está subordinado
à conquista do Estado por outros meios, ainda que revolucionários, mas é
uma crítica ao próprio Estado enquanto instituição a ser conquistada.
Defende"-se que a luta revolucionária é aquela que incide sobre o modelo
de organização e não meramente sobre o controle dos modos de produção.
Além disso, tratam"-se mesmo de alternativas excludentes: a oposição entre a
luta direta e o sistema representativo indireto, entre a urna e as ruas,
é uma oposição em princípio, uma oposição entre duas estratégias de
organização social, uma subordinada às instituições existentes
atualmente e outra que rompe com a subordinação. Por isso não votar
não é, para os anarquistas, uma ação negativa apenas, é uma ação
positiva direta de desobediência civil que se insurge contra o próprio
Estado, que afirma a capacidade de auto"-organização da sociedade. Não votar tem um aspecto educativo, pois nega a representatividade da instituição partido e traz consequências desorganizadoras, e um aspecto organizativo, pois combate a desorganização induzida pelos
partidos eleitorais para pulverizar e minar a luta direta da sociedade. Subordinando essa luta às eleições e à estrutura estatal, ela se torna um instrumento da dominação que supostamente pretendia combater.

Se não houvesse contradição entre lutar e votar, por que será que o
calendário das lutas e das greves seriam pautados, modificados e, muitas
vezes, esvaziados em virtude do calendário eleitoral? Se não houvesse
contradição entre lutar e votar, por que se esforçariam tanto em colocar
a luta nas ruas em função do blá blá blá eleitoral, como se vencer tal
disputa e conseguir um cargo qualquer fosse um fim em si mesmo?
Se não houvesse contradição entre lutar e votar, quais as razões dos
acordos a portas fechadas que vendem sempre a luta do povo? Em quais
gabinetes o empoderamento popular é esvaziado, cooptado, domesticado,
burocratizado? Por que será que as bases das mais diversas categorias
tocaram em 2013 e 2014 suas greves independentemente (e muitas vezes
contra) suas direções partidárias?

A disputa eleitoral, manipulável e controlável, posto que vendível,
passa a importar mais do que a luta do povo, invisibilizando"-a naquilo que ela possui de potencialmente revolucionário.
E isso não é um acaso. A contradição que há é sistêmica, é uma
contradição entre fins e meios, e ainda entre o que se toma ou não como
um fim em si. A contradição se mostra quando pensamos pelo que lutamos
exatamente, e eu digo que lutamos por uma sociedade que precisa ser
construída de baixo para cima, pelas próprias mãos daqueles que são
agora excluídos e oprimidos. Ou será isso, ou não será a sociedade pela
qual lutamos. Nenhum processo eleitoral pode nos aproximar de tal
sociedade, ele é feito para nos afastar dela, para canalizar a real
participação política para o âmbito de mais uma mercadoria negociável. E
não venham dizer que aqueles que lutam diretamente no dia a dia para a
construção desta nova sociedade estão ajudando a direita porque deixam
de apertar botões. Quem contribui com o fortalecimento da direita é quem
aproxima sua prática das mesmas práticas historicamente associadas a ela
(até que a diferença entre tais práticas seja menor que qualquer
quantidade dada e não possa sequer mais ser notada), apenas para
disputar cargos em um sistema desigual e corrompido.

Por tudo isso, boicotar as eleições é uma luta política concreta, não é
mero simbolismo oportunista ou propaganda do anarquismo, trata"-se de uma
arma ideológica e organizativa, uma estratégia concreta de luta.

Voltando a 2013, eu gostaria de dizer que é claro que os movimentos
sociais de base estavam ali desde muito antes, mas houve um momento no
qual a disposição em se encontrar, se associar e responder aos chamados
quase diários que eram feitos para manifestações poderia pender para
vários lados. Houve a tentativa de disputar as ruas, a tentativa de
levar o movimento para uma pauta esvaziada politicamente, como a pauta da
corrupção. O monopólio da informação sempre serviu para manipular a população, e
apostou na ressignificação daquilo que não conseguia impedir para encaminhar
a luta para o esvaziamento político e introduzir demandas da
direita. O mais interessante é que não funcionou, o que demonstrou a grande potência popular e desse momento
histórico. Muitas pessoas até foram às ruas respondendo ao chamado
midiático, manipulador e espetacular. Mas interessa o que elas
viram nas ruas: a concretude da revolta popular e a violência
policial. O contato com uma realidade desconhecida, que lhes é escondida, fez parte da educação política destas pessoas. Ir para as ruas não é controlável como ir para as urnas,
aqueles que manipulam podem perder. Nesse momento eles realmente perderam, aprenderam que ir às ruas é sempre perigoso e, por isso,
agora nos chamam às urnas claramente como uma oposição ao chamado à
participação política direta feito naquele momento. Quando a população
foi às ruas, ``muitos entraram pela direita e ficaram pela esquerda''.
Como as pessoas permaneciam sem controle nas ruas, houve também a ameaça
de um golpe de direita, a tentativa feia, descontextualizada, de levar o
povo a temer seu próprio poder, dizendo que poderia ainda
ficar pior. Mas isso também não funcionou. É nesse contexto que surgiu a
Frente Independente Popular no Rio de Janeiro (\versal{FIP}), uma frente que faz
ressurgir a aliança histórica entre anarquistas e comunistas (maoístas),
com um número grande também de pessoas independentes, que não fazem
parte de movimentos organizados. Os movimentos sociais já estavam lá, o
que surge do levante, da revolta popular nas ruas, é a \versal{FIP}.

E não é por acaso que a \versal{FIP} foi o principal foco da criminalização
orquestrada pelo Estado. A Frente foi fundamental para evitar que o
movimento fosse cooptado por partidos eleitoreiros ou mesmo por grupos
da extrema direita, é por isso que esses grupos, eleitoreiros e
oportunistas, grupos que queriam bater na então Presidente Dilma Rousseff
pela direita, têm tanta raiva da \versal{FIP} e contribuem para a nossa
criminalização. A \versal{FIP} não deixou que a luta fosse usada pelo espetáculo
eleitoral, e o que não pode ser canalizado pela via institucional, que
não é cooptável por um carguinho, que não é freado pelo burocratismo,
que não é assimilável, facilmente vendível, é, então, criminoso, deve
ser combatido e preso.

Há um ganho político nestas experiências, independente do que venha a
ocorrer daqui para frente. É a grande vitória social das jornadas de
junho de 2013: existe um ganho político para a nossa sociedade, um
amadurecimento que não nos pode mais ser retirado. Este ganho, inclusive,
não pode ser tomado por nenhuma criminalização, embora seja o que
se tenta fazer nesse momento. Nós vivemos em uma sociedade que nunca
viveu uma revolução popular, as mudanças que tivemos sempre foram
``herdadas'', mantendo uma mesma oligarquia. O que houve no ano passado
e o que ainda está havendo foi educativo, todo este processo que estamos
vivendo. E é certo que estamos no olho do furacão, é histórico, nossa
sociedade não é mais a mesma, aprendemos muito. Quando pensamos que as
ruas esvaziaram quase que naturalmente também é preciso dizer que não
foi bem assim. Sabemos que as críticas à Copa do Mundo da \versal{FIFA} de 2014 e
ao modelo excludente de cidade que acompanha o megaevento em questão não
se misturam, e são até opostas, enquanto o futebol é parte da cultura
popular. Não é fácil mudar anos de manipulação por meio do futebol em
uma sociedade forjada a partir disso. Mesmo assim, a repercussão dos
atos e da campanha ``Não vai ter Copa!'' foram enormes. Há um fluxo e
refluxo constitutivo da revolta e da organização popular. As pessoas
lotaram as ruas, depois houve o anseio por organização horizontal
e participação política. Ninguém pensou que a revolução seria feita em
junho de 2013, então era necessário aproveitar o empoderamento e a
educação política pela \emph{ação direta} para se voltar para a base,
para formar a base da transformação social, de modo a se voltar para as
ruas ainda mais fortes.

Este é o processo tão temido pelas oligarquias dominantes, um processo
de empoderamente e organização direta descentralizada, fora da
institucionalidade e não cooptável por partidos. Não foram as ações de
resistência nos atos que afastaram as pessoas das ruas, houve um
movimento natural de organização social e uma intensa repressão, que não
foi de modo algum uma contrapartida das ações nas ruas, que já respondiam a um intensa repressão. De todo modo, a partir
disso, um novo aumento dos atos de rua era previsto, e era isso que se
precisava evitar, e se precisava evitar de modo exemplar, não apenas
durante a Copa do Mundo \versal{FIFA} de 2014, mas também em um futuro próximo.

\chapter{Juventude periférica e 2013}

Em todas as épocas, foram os jovens que protagonizaram as grandes
mudanças sociais. Talvez porque ainda não foram
suficientemente acomodados à ordem vigente, eles têm tempo e disposição
para vislumbrar, visualizar uma outra realidade possível, e para começar
tudo de novo, do zero, agora mesmo. No início da vida não temos
tanto a perder, é mais fácil arriscar, e ainda não
fomos moldados pelo hábito a ver determinadas relações como fixas.
Depois todos são mais ou menos forçados a criar relações com as quais não
queremos nos apegar facilmente. Mas a mente hábil em conhecer o novo, em
aprender, é também a mente mais criativa. Não é por acaso que os grande
matemáticos e os grandes revolucionários tiveram em todas as épocas a
mesma faixa etária. Ambos precisam ver o impossível como possível, ambos
precisam criar um novo horizonte de significação e necessidade.

Pelo seu potencial de modificação social talvez seja o jovem que
precise ser moldado mais rapidamente, controlado mais fortemente,
aquele sobre quem as políticas repressivas do Estado de uma sociedade
excludente devam se abater de modo mais contundente. Diferentemente das
sociedades europeias, a população brasileira é extremamente
jovem. O mercado de trabalho e a formação profissional, entretanto, não são capazes de
inserir, em uma sociedade extremamente desigual, essa juventude. Forma"-se um enorme grupo insatisfeito, não inserido na sociedade e com pouco a perder.

A criminalização precoce dessa camada social, com sua entrada no tráfico e
no sistema carcerário, torna"-se então desejável como modo de
controle, como modo de inserção perversa na ordem vigente. É mais fácil
inseri"-los como marginais, justificar assim a militarização e a
repressão de espaços precarizados, o que garante o controle social, do que
gerar escolarização e consequente mão de obra qualificada, que o mercado
não será capaz de integrar.

Precisamos parar de ver a criminalização como um incidente não planejado
e não desejado pela lógica vigente. Ao contrário, a entrada do jovem
negro e pobre no mundo do crime é produzida, desejada, necessária.
Justifica o extermínio daqueles que o sistema precisa necessariamente
excluir, dado que não se propõe a combater a desigualdade. Justifica o
controle social, a militarização e a grande indústria que se tornou o
sistema carcerário. É o que o Estado chama de ressocialização, um
aparente deboche escrito nos uniformes das insituições penitenciárias de
nosso país. Mas a real maneira de inserir a juventude pobre,
potencialmente ameaçadora ao sistema, potencialmente revolucionária, é na
grande prisão que é a nossa sociedade. A alguns só resta o cárcere ou a
morte como inserção possível, uma inclusão diferenciada no sistema. O que não significa que todos nós, dentro dessa lógica, não estejamos já
presos ou mortos indiretamente. É preciso compreender a criminalização
da juventude como uma das faces de seu extermínio sistêmico.

Tomemos o índice de jovens negros de periferia mortos pela polícia. Em
2012, 56 mil pessoas foram assassinadas no Brasil. Destes, 30 mil eram
jovens (com idade entre 16 e 29 anos) e, entre eles, 77\% eram negros.
No ano seguinte, o assassinato de jovens brancos diminuiu 32\%, enquanto
o índice de assassinato de jovens negros aumentou na mesma proporção. É uma tendência geral nos últimos anos: crescimento do número de assassinatos de jovens, com diminuição constante da morte de jovens
brancos e aumento na mesma proporção de jovens negros.

O que temos, diante disso, é que o aumento geral de assassinatos no país
está sustentado no assassinato de jovens negros pobres. E este é um
retrato da pena de morte por execução no país, da enorme criminalização
da pobreza e do racismo silenciado, invisível, interiorizado na política
de segurança pública. Racismo de Estado. O jovem negro morador da
periferia é o matável na nossa sociedade, aquele que pode e deve morrer
para que nós, os humanos, continuemos vivos.

Digo que é invisível porque é a notícia naturalizada nos telejornais:
mais um jovem negro morto pela polícia na favela, mais um grupo de
adolescentes assassinados pela milícia na esquina de casa nas
periferias. É como se as pessoas escutassem e suspirassem transigentes com a ideia de que eles têm mesmo que morrer. E por que eles
devem morrer? Para seguirmos com nosso modo de vida, para que
continuemos fazendo compras no shopping aos domingos, para que
continuemos trocando de carro todos os anos, para que tenhamos Copa do
Mundo e Jogos Olímpicos.

As pessoas estranham, horrorizadas, quando escutam em documentários
que as jovens nas civilizações antigas, quase sempre meninas virgens, eram sacrificadas anualmente aos deuses da colheita em rituais sagrados. Pois os jovens negros das periferias são sacrifícios diários, os
tributos que nossa sociedade paga ao deus"-capital, ao deus"-consumo, ao
deus"-crescimento econômico. No entanto, trata"-se de um sacrifício não
regulado pelo ritual e, portanto, rotinizado como finalidade de
certas instituições estatais destinadas para isso.

De fato, temos uma longa tradição de criminalização da pobreza, de
guerra ao jovem negro e pobre. Por dia morrem em torno de 82 jovens
entre 16 e 29 anos. Isso obviamente não é noticiado pela grande mídia.
Entre eles, 93\% são do sexo masculino e 77\% são negros.

E é falso que não há protesto, que, diferente do que acontece em
Ferguson, as pessoas não se revoltam, não saem às ruas, não enfrentam a
polícia. Ocorre que a mídia oficial constrói a notícia como se o protesto
fosse orquestrado por traficantes, como se a criança assassinada talvez
pudesse mesmo ser culpada e como se, sendo ela negra e pobre e
favelada, já tivesse um destino traçado: mesmo que não estivesse envolvida
no crime, iria se envolver mais cedo ou mais tarde e, assim,
morrer em confronto. Além disso, a repressão dentro
da favela com balas de verdade não tem repercussão
internacional.

É preciso sublinhar a relação entre o potencial de insatisfação do jovem
da periferia, que é perigoso, e sua entrada para o tráfico com
morte precoce. A existência do tráfico nas favelas cumpre uma função
importante ao Estado, justificar a entrada da polícia, justificar a
militarização, justificar o extermínio do pobre com a bela desculpa do
combate às drogas.

Grande parte do potencial revolucionário de 2013 encontra"-se no fato,
que tantos tentaram esconder, de que aqueles que saíram às ruas eram majoritariamente jovens da periferia, moradores de rua. É necessário
desconstruir a ideia de que a classe média era a protagonista
nas ruas. O mais perigoso para o sistema naquele momento
era a presença da favela, daqueles que têm insatisfações concretas,
revoltas concretas, que perderam familiares mortos pela polícia e que
têm razões para odiá"-la. Daqueles que na invisibilidade da favela levam
tiros de verdade e que, naquele momento, diante das câmeras dos
mídia"-ativistas, podiam apenas levar balas de borracha. O menino de rua
diariamente espancado podia ter agora sua vingança materializada
em uma pedra portuguesa. Sua força, vinda do recorte de classe, foi logo
percebida, e são eles, não há dúvidas, que foram chamados de vândalos, desqualificando"-os ao lado dos supostos ``verdadeiros manifestantes''.

O levante de junho de 2013 contou com a presença popular
maciça; pessoas que jamais haviam ido em manifestações; moradores de
rua; negros das periferias das grandes cidades. Não foi um movimento da
classe média branca, como a grande mídia pretendeu estabelecer.
``Classe média'' é uma noção propositalmente difusa, sem identidade
política, que inclui condições sociais bastante distintas. Além disso,
os que ocuparam as ruas em peso eram, claramente, das camadas excluídas,
camelôs, moradores de ruas, favelados, trabalhadores precarizados e,
claro, boa parte da assim chamada classe média --- que, justamente pela
classificação difusa, engloba setores díspares da sociedade.
Devido à quantidade de pessoas que tomou as ruas, os atores
políticos consideraram que a insatisfação popular poderia ser canalizada
ou manipulada discursiva e midiaticamente. Foi isso que a guerra
da informação tentou fazer, inicialmente sem sucesso e, após o processo
de criminalização pesada, com mais êxito.

A presença de jovens negros da periferia foi, justamente, o que conferiu à
tática black bloc um perfil único no Brasil. Embora já famosa em
vários lugares do mundo, a tática que surge aqui em meio aos protestos
que ultrapassaram junho teve como diferencial o recorte e apoio popular,
crescente sobretudo entre a juventude excluída. É importante ressaltar
que a hipocrisia pacifista não funciona tão bem entre a camada mais
excluída da população, pois ela é o alvo do monopólio da violência
estatal. E, em nenhum local do mundo onde surgiu a tática, ela foi tão
admirada pela população excluída como aqui. O que ocorreu mesmo com o discurso
pesado e diário do monopólio midiático contra. Quem não se lembra dos
catracaços na central? Quem não se lembra da pesquisa frustrada no
programa de José Luiz Datena da \emph{Rede Bandeirantes}?\footnote{Referência
  a um programa de \versal{TV} ao vivo, da \emph{Rede Bandeirantes}, do dia 13/06/2013,
  no qual o apresentador José Luiz Datena, no auge das manifestações de
  2013, demanda, em uma pesquisa relâmpago, a seguinte pergunta aos seus
  telespectadores: ``você é a favor de protesto com baderna?'' Após
  resultado majoritariamente favorável, Datena resolve explicar o que
  está perguntando e reformular a pergunta, ressaltando que ele é
  contra. Ainda assim o resultado continua sendo que a ampla maioria é a
  favor. O programa está disponível
  em: \emph{<https://www.youtube.com/watch?v=7cxOK7SOI2k>. Acessado em: 20/01/2018.}}

O potencial de modificação social inerente a esse processo era muito grande,
porque aquele menino da favela que odeia o policial não precisava entrar
para o tráfico para expressar sua revolta, ele podia resistir, enfrentar
o Caveirão do \versal{BOPE} (Batalhão de Operações Especiais da Polícia Militar
do Rio de Janeiro), não abaixar a cabeça e empunhar escudos enquanto
ajudava trabalhadores a não pagarem passagem simplesmente sendo
anarquista. E quem não iria querer ser anarquista nessas condições? Com
isso, toda uma juventude excluída começou a crer no ``herói que veste
preto e máscara'', e a querer saber o que significava o \versal{A} na bola pichado
pelos muros da cidade. Eu sei que muitos opressores não dormiram
tranquilos naqueles dias.

\chapter{Perseguições políticas e criminalizações: a reação a 2013}

Era preciso parar a revolta e garantir as eleições. Fazer com que as
pessoas, tão pacatas antes de junho de 2013, simplesmente consumindo e
votando de dois em dois anos, e que agora estavam nas ruas, exigindo diariamente
real participação política, servissem de exemplo. Como quebrar o apoio de mais de 85\% da população
às manifestações? Como minar o empoderamento popular? Tal efervescência
social colocava em risco o lucro das grandes corporações capitalistas e
a falsa paz entre as classes, sustentáculos do poder de opressão exercido livremente pela classe dominante. A Copa do Mundo \versal{FIFA} de 2014 se aproximava e as manifestações voltavam a crescer. O espetáculo, então, precisava ganhar peso, pois as armas usuais não estavam
surtindo o efeito esperado. A primeira tática da reação, como se
sabe, foi a manipulação midiática, o discurso repetido à exaustação
sobre os vândalos infiltrados que, surpreendentemente, não recebeu completa aprovação dos espectadores. Depois da tentativa de cooptar o
movimento pelas pautas esvaziadas e/ou da direita, da ameaça de golpe e
do massacre discursivo, foi necessário criminalizar quem
resistia nas ruas e nos movimentos de base.

Em todos os lugares do mundo nos quais a tática Black Bloc surgiu
tais reações foram empregadas. Cito aqui uma parte do
recendo estudo de Francis Dupuis"-Déri, professor de Ciências Políticas
na Universidade de Quebec, que pesquisa o fenômeno mundialmente:

\begin{quote}
Quando um black bloc entra em ação, a resposta da mídia costuma seguir
um padrão típico. Na mesma tarde ou na manhã seguinte, os editores,
colunistas e repórteres falam mal dos arruaceiros dos black blocs
chamando"-os de vândalos. No dia seguinte, porém, o tom costuma ser mais
neutro. Os leitores são informados de que os anarquistas estão por trás
de táticas envolvendo armas como coquetéis molotovs, assim como os uso
de escudos e capacetes para se defender. Esses artigos às vezes fazem
referência a grandes black blocs do passado. Em seguida, citam alguns
acadêmicos, assim como representantes da polícia e porta"-vozes de
movimentos sociais institucionalizados que se desassociam dos
vândalos\footnote{\versal{DEPUIS-DÉRI}, Francis. \emph{Black Blocs}. Trad. Guilherme Miranda. São Paulo: Veneta, 2014, p. 20.}.
\end{quote}

Aliás, é importante ressaltar, a reação padrão de desqualificação
não é apenas do discurso midiático, mas da própria esquerda partidária e
institucional. Ainda citando Dupuis"-Déris:

\begin{quote}
Embora alguns porta"-vozes de instituições sociais"-democratas, partidos
socialistas e sindicatos critiquem tanto a violência policial como a
brutalidade do capitalismo, seus ataques padrões aos black blocs não
diferem dos perpetrados pelos policiais e políticos de centro e direita.
Yvette Cooper, membro do parlamento do Partido Trabalhista britânico, ao
comentar os eventos em Londres, denunciou as ``centenas de idiotas
irracionais envolvidos em comportamentos criminosos da pior espécie''.
Chris Hedges, intelectual e escritor de esquerda, falou o seguinte sobre
o chamado do Movimento Occupy a manifestações em novembro de 2011: ``Os
anarquistas do Black Bloc, que atuaram nas ruas de Oakland e outras
cidades, são o câncer do movimento Occupy, eles confundem atos de
vandalismo e ceticismo repulsivo com revolução. Existe apenas uma
palavra para isso: crime''\footnote{\emph{Idem}, p. 29.}.
\end{quote}

E a reação no Brasil não foi distinta. Partidos de esquerda lançaram notas se
desvinculando e criticando a ação dos anarquistas, alguns ajudaram mesmo
a criminalizar e entregar pessoas à polícia. De fato, toda luta não
cooptável, não assimilável institucionalmente, é transformada
sistematicamente em crime, em quadrilha. Há uma disputa por terreno, por
nicho de mercado, para canalizar a luta concreta, direta, da população,
para fins eleitorais --- no que esquerda e direita revelam seu ponto comum.
E os meios para deter a revolta popular são sempre os mesmos:
infiltração policial; prisões preventivas em bloco; foco espetacular
midiático em alguns indivíduos; tentativa de cooptação das imagens do
protesto radical para fins comerciais. Desde o policial e o político,
passando pelos intelectuais acadêmicos, pelo ``bom manifestante'', editor, mídia oficial, inquérito, todos devem compartilhar as
mesmas conclusões, ``irracionais, bandidos, anarquistas, vândalos,
arruaceiros, desprovidos de ideologia, usados por alguém\ldots{}'' Palavras
que têm efeitos muito concretos, pois privam a ação coletiva de
credibilidade política ao mesmo tempo que legitimam e estimulam a
repressão policial.

E o pânico precisava ser instaurado, alguém precisava morrer.
Alguém sempre tem que morrer, embora muitos sempre morram de fato. Não
estou dizendo que o Estado matou o jornalista Santiago Andrade, da \emph{Rede
Bandeirantes}, embora certamente a culpa da violência, em máximo grau,
seja do Estado e de seus séculos de opressão, e não daqueles que
resistem a ela. Estou dizendo que foi preciso matá"-lo muitas vezes,
criar comoção nacional, transformar sua morte em um fato político que
amedrontasse a população com o exercício do seu próprio poder,
transformar um acidente terrível --- pelo qual a própria empresa para a
qual ele trabalhava teve sua responsabilidade, já que o enviou sem material de proteção adequado para uma
área sabidamente de conflito --- em um
homicídio planejado com intenção de matar. Estou dizendo, portanto, que
a morte do cinegrafista Santiago Andrade caiu como uma
luva nesse contexto. É importante lembrar que o mesmo processo ocorreu em 2018 na
França, quando três pessoas morreram dentro de um prédio bancário, após
um protesto, e isso foi usado como modo de esvaziar as ruas e criminalizar os protestantes, trazendo pânico à população. Não foram as resistências
nas ruas e a tática black bloc que tirou o povo das ruas, foi a
violência policial, o uso político que se fez da morte do jornalista
Santiago Andrade e a criminalização que sucedeu. E muitos foram
aqueles e aquelas que morreram no contexto das manifestações, pessoas
que levaram tiros de verdade sem que isso fosse noticiado. Uma
professora intoxicada com o gás lacrimogênio; vários jornalistas que
ficaram cegos; trabalhadores atropelados fugindo da polícia; uma
manifestante que ``caiu'' de uma ponte em Belo Horizonte\ldots{} E
mais numerosos ainda são aqueles e aquelas que morreram pelos motivos que
levaram às manifestações, os que morrem todos os dias nas
favelas, nas periferias, nas filas de hospitais, vítimas da violência e
da tortura policial. Mas, subitamente, parecia que aquilo jamais havia
ocorrido, que teria sido a única morte desumana da nossa sociedade, pelo menos a única que a mídia já tinha ouvido falar. Pedia"-se finalmente as cabeças dos culpados. O circo estava montado. Esse foi o
pano de fundo necessário à nossa criminalização e consequente prisão no
dia 12. É um capítulo que não acabou ainda, que não se esgotou na
nossa prisão, que não termina nesse inquérito. Esta é a história
presente, é a história que estamos fazendo agora.

Nossa prisão preventiva, em bloco, sem absolutamente nenhuma
consistência, nenhum crime concreto, e a divulgação do inquérito,
ocorreu sob este pano de fundo. Há um grande esforço na investigação para
transformar as organizações políticas não"-eleitorais em quadrilhas.
Trata"-se de um esforço discursivo, que interpreta nossas práticas e
ideologias. E o que se pretende com isso é recusar o direito à
auto"-organização da sociedade sem fins eleitorais. O inquérito
chega a dizer que ``a organização não eleitoral se afasta do viés
político"-ideológico legítimo em nosso sistema democrático''. Então, há
um viés político"-ideológico que não é legítimo, e este é,
principalmente, o viés anarquista. Não é exagero quando dizemos que o
anarquismo e, mais ainda, toda organização direta da sociedade civil, é o
que se pretende impedir, é o que será transformado em crime se formos
de fato condenados. Tanto é assim que um grupo de educação popular, que
organiza alfabetização para jovens e adultos e pré"-vestibulares, é
identificado como ``uma iniciativa louvável dos anarquistas, mas que é
na verdade uma fachada para treinar jovens para a luta armada''.

Existe um construto hermenêutico capaz de fazer organizações sociais e
políticas serem lidas como quadrilhas armadas. E é fundamental
pensar quais são os passos que permitem esta passagem. Cito uma parte do inquérito policial que explicita isso. Em um relatório de análise da tática black bloc, da
investigação que levou à nossa prisão, encontra"-se o seguinte o trecho:
``Convergindo na tática black bloc há a delinquência comum
qualificada como exercício da violência como fim em si, há a
delinquência pragmática, qualificada como uso da violência como meio de
impor pautas (que podem até ser justas e defensáveis nelas mesmas) e há,
por último, a delinquência política de viés anarquista, caracterizada
pelo uso da violência para desgastar e solapar as instituições do Estado
para, num fim último, criar uma nova sociedade sob fundamentos diversos
da atual. É desnecessário dizer que estas três formas de delinquência
são igualmente repudiáveis e merecedoras de coerção estatal. No entanto,
a delinquência política de viés anarquista é a mais insidiosa e a que
precisa ser mais fortemente combatida. Ela é ideológica, age de modo
dissimulado e sorrateiro, instrumentaliza os demais agentes violentos,
infiltra"-se e coopta movimentos sociais, apodera"-se dos focos de
insatisfação difusos na sociedade para manipulá"-los segundo a
conveniências de seu interesse político''. Percebe"-se então que o
anarquismo é tomado como pior e mais perigoso do que a violência como fim em si, do que o uso da violência para
conseguir algo pontual, que pode até ser atingido, contemplado,
negociável. Por que devemos ser fortemente combatidos?

A insatisfação sistêmica, a recusa dos meios tradicionais, a
impossibilidade de se negociar, tudo isso contribui para que o
anarquismo, enquanto proposta de organização direta da sociedade, seja
fortemente criminalizado. Nesse sentido, o que amedronta, e que fica claro no próprio discurso investigativo do inquérito, é nossa multiplicidade de pautas e nosso caráter difuso. Afinal, é todo um
modo de vida que se recusa. Não há um reformismo, uma reivindicação
pontual que pudesse ser simplesmente atendida, mantendo"-se toda a
estrutura como está. Então, não há como deter tal insatisfação
sistêmica. E o que preocupa o sistema não é a delinquência vazia, não é ignorarmos pelo que lutamos, como se
repete no discurso midiático. É justamente o que temos de mais
consciente, mais ideológico e que não é passível de ser atingido ou
corrompido. E se somos nós, os anarquistas, aqueles que devem ser mais
fortemente combatidos, como dizer que isso não é político"-ideológico, ou que não demandamos mudanças possíveis? Um grupo político não é mais perseguido do que criminosos comuns se tal posição política não é vista como uma ameaça em si mesma. Não para a sociedade, que é impedida de se auto"-organizar, mas para a
estrutura estatal que parasita a sociedade. Fato é que
podemos criar uma nova sociedade sob fundamentos distintos da atual. É a
sociedade organizada, acima de qualquer coisa, que tem legitimidade para isso, mesmo que a repressão estatal insista em nos criminalizar. É da sociedade que emana todo poder, o fundamento, inclusive, do que se
entende por democracia, embora não vivamos em uma sociedade realmente
democrática, já que o poder do povo é contraditório com o poder do
capital.

Todo poder emana do povo, para o povo e pelo povo, devendo sempre ao
povo retornar. Se o Estado se volta contra seu povo, esse povo tem o
direito inalienável de destituir este Estado, usando para isso todos os
meios que estiverem à sua disposição. Assim, as sociedades democráticas
nascem legitimando a possibilidade de revolução popular sempre que o
``pacto social'', pelo qual supostamente o povo transfere seu poder ao
Estado, for rompido e o Estado não corresponder aos anseios da vontade
coletiva. As sociedades democráticas, desta maneira, legitimaram sua própria origem, pois surgiram de uma
revolução --- lembrando sempre que foi sobre o sangue das cabeças decepadas dos nobres que se
pôde impor, senão como realidade, ao menos como valores, ``a igualdade,
a fraternidade e a liberdade''. Que depois uma elite tenha
continuado detendo os meios do Estado para a manutenção de seus
privilégios, opondo assim claramente Estado e sociedade, não pode
esconder a origem histórica revolucionária das chamadas ``democracias
modernas''. Não que se queira dizer que os ares dessa revolução
tenham realmente algum dia chegado por aqui. Mas, se ainda hoje, menos
de dez por cento da população retém os meios de produção, as
propriedades e o lucro sobre o que é produzido, como falar em democracia
sem sentir vergonha? Como fingir não ver que é sobretudo o poder do povo que o Estado, representante dos interesses da elite econômica, pretende evitar?

O Estado se recrudesce, mostra sua cara e se volta mesmo contra
as camadas incluídas da população quando apoiam a luta dos
excluídos, lutam ao seu lado e denunciam a violência contra eles. O
Estado sempre estará pronto para punir exemplarmente esses
casos. Mas o importante é a permanência de um silêncio conivente, da maior
parte da sociedade, com o terrorismo imposto pelos agentes estatais e o
poder econômico que estes representam às camadas excluídas. É
extremamente sintomático dessa situação e expressivo da sociedade na qual
vivemos que, quando o povo foi às ruas exigindo poder ao povo, o primeiro preso e condenado tenha sido negro, pobre e morador de rua. É a resposta punitiva que o Estado quer dar, ressaltando que,
independente do que acontecer, o Estado e o poder permanecerão nas mãos das
elites e não do povo. Nós lutamos pela igualdade, em princípio, sabendo
da desigualdade que há, de fato. Para que nossa luta não signifique tão
somente nossa criminalização, nosso próprio silenciamento e banimento, é
preciso que ela continue sendo a luta da sociedade em geral.

A criminalização dos movimentos sociais é uma linha política
internacional atual, que tem em vista manter a guerra
funcionando. É preciso temer o terrorismo, garantir o domínio das
riquezas com a exclusão crescente, é preciso justificar o ataque aos
pobres e manter a população sob controle, fazendo parecer que não há
guerra nenhuma em curso. Os megaeventos caem como uma
luva nesse contexto. Da mesma maneira que o tráfico de drogas justifica a
criminalização da pobreza, o medo do terrorismo, a segurança pública nos
megaeventos justificam as leis antiterrorismos, os tribunais de exceção,
toda a gentrificação, e também toda a repressão e criminalização que a
acompanha. Não é possível ser uma cidade cosmopolita, entrar para o
ranque de grandes polos comerciais do mundo, sem sediar megaeventos, e
não é possível sediar megaeventos sem criminalização.

E há um modo padrão como essa criminalização opera hoje. Ela deve proceder
antes de tudo pelo medo, pela sensação de se estar todo o tempo vigiado.
As escutas, os mecanismos midiáticos de exposição, criam uma rede que
prende antes de qualquer julgamento. Não é preciso sequer que o
processo chegue a condenar para que sejamos punidos. Somos punidos nas
medidas restritivas, na suspensão de nosso direito de ir e vir, na ameaça
constante de novos processos, na retirada de nossos direitos políticos,
direta e indiretamente. A presença do Estado é um controle invisível, um
Grande Irmão que nos acompanha como na obra \emph{1984}, de George Orwell.
Mas como ocorre com Josef K., protagonista de \emph{O processo}, de Franz Kafka, não
sabemos sequer quando uma nova acusação será feita ou quando será
executada nossa punição.

Não sabemos pelo que somos investigados porque os inquéritos são
sigilosos. Nossas vidas podem ser, de um dia para outro,
expostas no programa televisivo da \emph{Rede Globo} ``Fantástico'' ou na revista
\emph{Veja}. Assim, com medo de agir, nossa participação política é
suspensa, não atuamos. As liberdades democráticas não precisam ser
retiradas com um golpe militar, a criminalização sistêmica é mais do que
suficiente e eficiente para isso.

\chapter{Como terminará 2013?}

\begin{flushright}
\footnotesize{\emph{E nossa história\\
Não estará\\
Pelo avesso assim\\
Sem final feliz\\
Teremos coisas bonitas pra contar\\
E até lá vamos viver\\
Temos muito ainda por fazer\\
Não olhe pra trás\\
Apenas começamos\ldots{}}

(Legião Urbana -- Metal contra as Nuvens)}
\end{flushright}

\bigskip

Após a ratificação do \emph{habeas corpus} concedido a Elisa Quadros,
Karlayne Moraes e Igor Mendes pelo \versal{STJ}, os chamados ``23 processados da
Copa'' seguem aguardando a decisão judicial da primeira instância.
Tratam"-se de ativistas que foram presos preventivamente na véspera da
final da Copa do Mundo, acusados de formação de quadrilha armada e
corrupção de menores. A prisão, criminalização e perseguição política
ocorreram como uma reação a junho de 2013, às vésperas da final da Copa do
Mundo de Futebol e antes do processo eleitoral, para impedir a revolta
popular, garantir o lucro da \versal{FIFA} e as eleições sequentes sem protestos.

Atualmente, nós vivemos um contexto de criminalização crescente. Há um
grande esforço em curso para transformar as organizações políticas
não"-eleitorais em quadrilhas. A criminalização inside principalmente
sobre os movimentos independentes de partidos, anarquistas, com modos de
mobilização e organização com os quais o Estado não sabe lidar
imediatamente e que possuem algum potencial de não"-cooptação
institucional e, portanto, de modificação social concreta. O processo
dos 23 é uma das sucessivas medidas que visam barrar este modo de
organização popular. É também o primeiro grande processo após junho de
2013, que servirá como modelo e punição exemplar. Não são
apenas os 23 ativistas processados que estão sendo punidos, é toda uma
sociedade que se rebelou, que fez a tarifa baixar, que lembrou aos
governantes que o poder emana do povo, e que, portanto, ``não teria arrego
porque o amor havia acabado''. Todos precisam ficar atentos ao resultado
desse processo, porque trata"-se de punir uma sociedade inteira que se
levantou. Esperamos agora para saber como terminará 2013, como será o
seu desfecho, qual a resposta por parte do Estado e seu significado
para a continuação das lutas e resistências em curso.
Esta sentença é um pouco para todos nós, é um recado, é uma resposta. E
o que se pretende recusar com isso é o direito à auto"-organização da
sociedade. É para dizer que nunca mais haverá um junho de 2013, para dizer
que jamais será primavera novamente. Cumpre notar que vinte dos 23
processados, que respondem em liberdade, continuam submetidos às medidas
restritivas, totalmente inconstitucionais, desde aquela época, com direitos políticos suspensos, sem poder participar de
qualquer manifestação. A pena que podem pegar é de até 8 anos de prisão.
Mas se a \versal{PL} 2.016-\versal{F}, que tipifica o crime de terrorismo, estivesse já em
vigor, a pena poderia ser de até 24 anos, sem direito à regressão de
pena. A aprovação a toque de caixa dessa lei, apresentada e apressada pelo governo Dilma, responde à pressão
internacional e ocorre paralelamente à reação a 2013, tentando evitar protestos similares
durante as Olimpíadas. Pois aproveitar a atenção mundial focada no Brasil para expôr reivindicações deve agora ser considerado terrorismo, isto é, deve ser punido de modo mais severo do que
assassinato. O projeto do governo federal também deixa claro a
contradição inerente ao projeto da social"-democracia, já que a
criminalização é defendida por um partido que nasceu nos
movimentos sociais. Vemos, assim, quão superficial é a diferença entre os
projetos do \versal{PSDB} e do \versal{PT}, diferença que não afeta a política econômica
em seu cerne, nem a política de segurança. A superficial contraposição
serve apenas para parecer que se escolhe um produto, mas são suficientes para gerar nixos de mercado
eleitoral, pois o cerne da agenda econômica e dos interesses das grandes
corporações deve ser mantido.

A linha política que criminaliza movimentos sociais pretende barrar qualquer processo
histórico de modificação social real, mantendo a exclusão crescente e o funcionamento da guerra. Sabemos que a
definição de terrorismo jamais incluirá a violência de Estado, pois
terrorista adjetiva apenas quem é eliminável, designa o outro em uma guerra desigual. Não se trata de saber quem gera
pânico e quem espalha destruição e morte, não se trata de saber quem
dispõe sobre o direito de vida, quem controla os bens necessários para
produção e reprodução básicos --- se fosse assim, verdadeiramente, o
Estado e seus agentes seriam os principais terroristas em todas as
épocas. Mas a \versal{UPP} (Unidades de Polícia Pacificadora) na favela não será
chamada de terrorista, não importa quantos jovens negros e pobres sejam
assassinados. Pois o que a violência simbólica toma como execrável é
justamente o que se opõe ao monopólio da violência naturalizada. Assim,
não importa quantas crianças morram nas favelas bem ao nosso lado, não
importa quantos não tenham acesso a atendimento médico, quantos durmam
na rua sem ter o que comer. Se o capital não nega acesso às necessidades
básicas, não saberíamos dizer quem nega, pois interromper a via pública
por duas, três horas, não significa nada em relação ao direito de
ir e vir, por exemplo, em comparação com o monopólio dos transportes
públicos em mau funcionamento e com preços inacessíveis a tantos. E o
mesmo poderíamos dizer quanto aos hospitais e escolas, moradia e
alimentação. Por que será que saquear um supermercado é mais
terrorista do que manter pessoas sem comer? Ou por que será que
interromper o recolhimento de lixo é mais terrorista do que precarizar
esse serviço até tornar inviável que ele permaneça sem ser
interrompido? Manter pessoas trabalhando como escravas não deveria ser
considerado terrorismo? Mas a violência contra aqueles que não são
ideologicamente tomados como matáveis parecerá sempre mais assustadora e
será retratada sempre como justificativa às respostas
repressivas mais violentas. Pois o terror do outro é sempre mais
palatável, resta saber o que será tratado como legítimo para manter o lado que escreve os discursos e propaga as notícias como ganhador da guerra.

A lei aqui é uma arma de guerra, é um mecanismo para manter justificado
o monopólio do terror. A lei é um instrumento dos poderosos para manter
seu poder, para, fingindo que evitam a guerra, manterem"-se vencedores com as medidas mais violentas e desumanas. Tudo pela preservação das estruturas atuais de poder que se
auto"-definem como estatais. As leis são, portanto, atos violentos
quando justificam as mortes e a miséria. A lei não é o ato do civilizado
para evitar a guerra, a lei estatal é um instrumento para reter o poder,
é um ato na guerra. Um ato que mantém certo monopólio do uso irrestrito
da força. Garantir o monopólio do fomento ao terrorismo no mundo é
garantir a vitória perpétua nas guerras. Sabemos que os Estados criam os
inimigos que precisam justificar o combate. Quem pode ser o inimigo
pós"-guerra fria? O que justifica a intervenção internacional e o
controle imperialista? O terrorista será, a partir de agora, todo aquele
que precisa ser eliminado.

O que ocorre na \versal{UERJ} atualmente é particularmente exemplar. Em um
contexto de precarização da educação pública, com terceirizações e
estudantes bolsistas sem receber, a Universidade vem sendo palco de duas
investigações policiais que se relacionam e complementam. As
investigações foram instauradas na 18ª \versal{DP}, após o ato de solidariedade dos
estudantes da \versal{UERJ} aos moradores da favela do Metrô"-Mangueira, que
sofriam violenta remoção por parte do Estado. Mas, como se
o diálogo entre a favela e a Universidade fosse intolerável, a reação
se voltou diretamente contra coletivos, professores e funcionários
politicamente ativos, que estão sendo chamados para depor em duas
investigações: uma sobre danos ao patrimônio público no dia do ato, e
outra sobre, impressionantemente, tráfico de drogas no nono andar, o
andar da Filosofia, História e Ciências Sociais, tradicionalmente
ativo politicamente, reunindo inúmeros centros acadêmicos e coletivos. A
estratégia aqui da criminalização é a mesma, pois a acusação de tráfico
de drogas justifica a introdução de policiais no nono andar e o
monitoramento das atividades acadêmicas e políticas neste andar. Além
disso, a acusação de tráfico desqualifica, torna o outro matável,
eliminável. Paralelamente a isso, o pânico, o medo do monitoramento
constante é instaurado, e as pautas das mobilizações, os problemas da
Universidade só se agravam. Mas a construção do sujeito terrorista
permanece a mesma, aqueles que são chamados para depor devem responder
sobre consumo de drogas, suas próprias posições ideológicas e sobre a
existência de grupos anarquistas no nono andar, preferencialmente
relacionando os três âmbitos.

Então, da próxima vez que alguém te disser que o povo brasileiro aceitou
tudo quieto, que não faz nada e segue pagando bancos e votando em
políticos corruptos sem se revoltar, lembre"-se do enorme trabalho que os
governos vêm tendo, não só aqui, mas em todo o mundo, para continuar nos
empurrando goela abaixo este projeto, este modo de vida indigente e para
nos manter aceitando, sob a mira das armas, que tão poucos submetam
tantos à miséria e à morte.

Não serão as leis que nos livrarão do terrorismo.

\chapter*{O que resta de 2013?\\
\emph{O momento histórico do anarquismo}}

\addcontentsline{toc}{chapter}{O que resta de 2013?}

O que resta, passados já alguns anos? Se estamos corretos, 2013 tratou"-se, antes de
tudo, de uma perda de confiança no sistema representativo, que é própria
de um contexto histórico, e uma insatisfação com os meios da esquerda
institucional e o burocratismo dos partidos eleitorais. Mantém"-se aberto, pois,
o espaço para as propostas anarquistas e para a ação direta nas ruas, mesmo
que a maior parte da população não pense exatamente nesses termos. E, a
partir desse espaço, as posições defendidas pelos anarquistas, enquanto uma
alternativa real à esquerda institucional, mantêm"-se aptas a obter grande
crescimento e repercussão. Devemos ressaltar a falência do sistema
representativo, a qual tenta"-se esconder com a ideia de que a
troca de governantes poderia resolver a situação, sustentando um
modelo falido que diante de sua tragédia cada vez mais se
alia ao grande capital para se sustentar por meio de um discurso do medo
e do ``menos pior''. Se isso é correto, a campanha antieleitoral é
fundamental à alternativa anarquista para que a população não tenha sua
insatisfação com as medidas de austeridade que vêm sendo postas canalizadas para o
espetáculo eleitoral. Fomentar a revolta e as organizações e lutas de
resistência autônomas não canalizáveis pelas eleições.

Sabemos que a democracia representativa financiada pelos detentores do
grande capital não pode operar internamente a ruptura com seus próprios
pressupostos. Certo é que o governo não poderia atender de fato às
reivindicações das ruas em 2013, pois elas apontavam para algo totalmente
fora deste sistema, para o inegociável, para a possibilidade de um outro
modo de vida. Nenhuma reforma no sistema pode dar conta da sua
própria destruição, o que não significa que tal destruição não seja
possível. A autogestão popular só poderá ser fruto da própria
organização popular. O problema é que os atores, no jogo da política
representativa, só postulam discursivamente uns aos outros como inimigos
possíveis, enquanto 2013 apontou para a possibilidade de ruptura no
jogo. Assim, o \versal{PT} culpará sempre setores da direita partidária mais ou
menos vinculados, ainda que espetacularmente, aos valores de seus
inimigos nas urnas. E vice"-versa: a direita fará o mesmo,
culpando sempre setores da esquerda partidária. Mas a
vida política de uma sociedade não se reduz a comprar candidatos em um
supermercado eleitoral. E, de fato, eles sabem disso. Por isso, enquanto
criam inimigos espetaculares e simulacros de si mesmos nos discursos,
combatem realmente, materialmente, aqueles que são ameaças concretas, por
meio da criminalização geral e punição exemplar. É preciso criar a ilusão, nos
discursos, de que essas ameaças não existem, e na prática fazer o possível para
que deixem de existir, pois suas existências apontam para a morte do
sistema e daqueles que vivem, de um modo ou de outro, da sua manutenção.

Com o levante, presenciamos um grande ressurgimento do anarquismo.
Bandeiras negras tomaram as ruas, o ``\versal{A}'' circulado foi pichado pelas ruas
das principais cidades do país, a grande mídia teve que nos colocar em
programas de \versal{TV} com destaque. Não apenas aqui, mas em todo o mundo, o
anarquismo vem ganhando espaço. E por que vivemos um contexto histórico
tão propenso ao crescimento do anarquismo, não apenas no Brasil, mas no
mundo inteiro? Ora, vivemos o período posterior ao fim da \versal{URSS} e à queda
do muro de Berlim, quando vimos fracassar
os modelos estatais de socialismo. Vimos o socialismo de Estado se
transformar na força opressiva que os anarquistas sempre disseram que se
transformaria. Antes, havia um modelo de revolução exportado e propagado
por um programa mais ou menos único dos Partidos Comunistas no mundo, e
não ser capitalista significava antes de tudo aderir a esse programa.
Não que as vozes anarquistas não estivessem ali, elas sempre estiveram e
foram particularmente fortes em 68, mas mesmo ali constituiam discurso
desviante dentro da esquerda, as notas dissonantes, a minoria sufocada
e, quando possível, massacrada.

Com o fracasso do modelo soviético, sem que as desigualdades
insuportáveis do capitalismo deixassem de crescer, abriu"-se espaço para a
alternativa libertária talvez como nunca antes na história mundial. Se
não temos mais o grande projeto político único do \versal{PCB}, se não temos mais
a União Soviética, a falência deste projeto abriu espaço para a retomada
e o fortalecimento de outros modos de organização, tanto mais fortes
quanto menos centralizados. Somos uma geração que conhece os limites da
suposta democracia fundada no capital ao mesmo tempo que não vê mais a
ditadura do proletariado como uma alternativa real e concreta. A
esquerda institucional deixou claro seus limites tanto tomando o
aparelho de Estado quanto tentando atuar reformistamente dentro da
democracia liberal. Após presenciarmos o fiasco das tentativas de
concretizar a teoria socialista do Estado, a realidade prática falou mais
alto quando em 1994 ocorreu o levante zapatista no México; como
realidade concreta, para além de qualquer modelo anteriormente proposto.
Este foi sem dúvida um novo fôlego para os anarquistas, pois a
resistência zapatista demonstra, de uma vez por todas e a cada novo dia,
a existência desnecessária do Estado. Hoje são inegáveis os
elementos anarquistas dos occupys, levantes e táticas de resistência
recentes: organização em redes, horizontalidade, descentralização,
autogestão e ação direta. O anarquismo ocupa hoje o espaço que nos
movimentos e lutas sociais da década de sessenta era ocupado pelo
marxismo.

Ao lado disso, outro aspecto fundamental é a crise no modelo da
representação. Este modelo cognitivo e ontológico ideológico é o
fundamento das sociedades modernas e funda"-se na separação rígida entre
um âmbito da realidade tomado como abstrato e outro tomado como
concreto. Tal separação, que apareceu em Descartes com a divisão rígida
entre uma substância pensante e uma material, tem como
premissa básica, no âmbito do conhecimento, a posição pela qual conhecer,
significar, organizar a multiplicidade empírica é, antes de tudo,
representar. Estabelecer conhecimentos é estabelecer poder, e, neste
paradigma, também o exercício do poder é visto como um correspondente
abstrato da realidade concreta. A verdade se diz, portanto, não de um acontecimento, mas sempre de um respresentante, que não pode colapsar jamais com aquilo que pretende representar, embora seja dito
verdadeiro precisamente por corresponder a ele. Esta separação rígida
tem obviamente um limite, que aponta para a necessidade de diálogo
fulcral entre abstrato e concreto. Se não fosse assim, inclusive,
nenhuma mudança estrutural seria possível. O próprio paradigma da
representação funda"-se em uma verdade vivida, que não é do âmbito da
representação, mas da apresentação, e da ação direta.

A perda de interesse na via indireta da representatividade se relaciona
diretamente com a constatação da sua incapacidade de modificar,
transformar, fundar a realidade. Nosso tempo talvez seja o tempo que
mais claramente se encontra diante dos limites da representação, e da
certeza da sua superficialidade inerente. É uma decepção vívida, e
talvez por isso este seja o momento histórico que mais fortemente parece
abrir espaço para a possibilidade da auto"-organização.

Entender a ruptura com o paradigma representacional é compreender também
um pouco da ontologia anarquista, enquanto uma ontologia dialética. Mas
é claro para nós que não existe apenas um anarquismo, mas vários; uma
pluralidade que nos faz demandar o que haveria de comum
a ponto de podermos falar de uma ontologia anarquista.

Pensemos no contexto Moderno, do Iluminismo, no qual o anarquismo clássico surge,
como resistência à tendência de
despolitização da sociedade. O anarquismo não nasce se contrapondo a um
regime político particular, mas visando inaugurar uma nova forma de vida
associativa enquanto constitutiva da sociedade. A ontologia da dialética
materialista anarquista é precisamente aquela que não separa rigidamente um
âmbito abstrato de um concreto e que, por isso, também não se
compromete com uma vanguarda intelectual, com uma distinção igualmente
rígida entre teoria e prática e com a manutenção de lideranças ou
qualquer hierarquia fundada em acessos ou méritos especiais de uma
determinada classe. A primeira entidade abstrata, supostamente
representante da sociedade concreta, à qual o anarquismo nega
a necessidade da existência, seja como fundamento da sua unidade,
seja como fundamentado por ela, é o Estado, isto é, a falsa
\emph{estabilidade}, imobilidade, do real, mas que, como a raiz grega do
termo --- \emph{stasis} --- revela, se funda na guerra, revelando na palavra o
princípio da violência do Estado inerente à sua fundação e manutenção. O
Estado, supostamente \emph{estático}, funda"-se e mantém"-se pela
violência, pela guerra constante contra o povo. E não apenas o Estado
enquanto entidade"-forma"-abstrata, mas também o estatismo diluído,
disperso nos modos de organização e nas microrrelações sociais. Nenhuma
cristalização de substancialização pode tornar estático o fluir perpétuo
da passagem do abstrato ao concreto e do concreto ao abstrato. Portanto,
e ao mesmo tempo, toda concretude humana é perpassada por
relações de significação, por isso o anarquismo é a verdadeira ordem,
nos dirá Proudhon, a única ordem capaz de conciliar a mais perfeita
liberdade com a vontade coletiva, mostrando que o indivíduo não é um
pontinho separado, dado no vazio das relações sociais, mas afirmando o
caráter primário da uno"-multiplicidade. Só assim é possível afirmar a
possibilidade de ultrapassar a finitude humana nela mesma sem jamais
negá"-la em prol de qualquer consolo metafísico. Não nos parece um acaso
que o pensador existencialista Albert Camus tenha flertado com o
comunismo, mas tenha se afirmado em sua maturidade um anarquista. O
fundamento desta posição é algo que não nos cabe aqui desenvolver, pois
seria o objeto de um trabalho que ultrapassa as pretensões deste texto,
entretanto cumpre notar que sua superação do absurdo, enquanto exílio do
sentido do mundo, na afirmação deste próprio absurdo reúne bastante da
perspectiva ontológica anarquista. Um ser e mundo que não se reconciliam
pela dialética finalizando a história, uma união dada na própria
separação, uma unidade afirmada na própria multiplicidade, uma dialética
sem síntese. Não haverá sociedade acabada, não haverá final da história.
O anarquismo que é ontologia, já defendeu Vaccaro, não pode ser uma
teoria abstrata, mas é, nos termos de Foucault, uma prática discursiva,
isto é, ao mesmo tempo teórica, ressignificante do real, e prática. Por
isso não pode haver pensador teórico anarquista que não seja ao mesmo
tempo militante, protagonista da história do movimento político da sua
época.

Um exemplo fundamental da continuidade pós"-2013 das novas formas de
resistência foram as ocupações secundaristas. Na medida em que a
criminalização aumenta, as resistências inventam novas formas de luta.
Em São Paulo, as ocupações estudantis levaram o governo a recuar no
projeto de fechamento das escolas. Novamente, como em 2013, vimos as
estruturas de poder serem facilmente ameaçadas, demonstrando a
fraqueza real por trás da aparente indestrutibilidade.

Novamente, a ação direta horizontal não orquestrada pelas instituições
deixou o Estado sem saber como agir. Novamente, a manipulação midiática
não funcionou e o apoio populacional foi intenso.

A tentativa de desqualificação como terroristas ou vândalos não se encaixou
com a realidade concreta da juventude, em sua maioria adolescente, que
protagonizou as ocupações, e o discurso da manipulação por partidos também
foi contradito pela realidade. As ocupações de escolas têm a força de
serem uma experiência da vivência autogestionária, onde cada ocupação é um
modelo de outra organização social: gerindo suas próprias escolas, os
estudantes experimentam a possibilidade de um outro modo de vida. Cada
ocupação é uma protorrevolução. Por isso, as ocupações não são apenas
meios táticos para pautas externas, não são pressões para se conseguir
algo, não ocorrem completamente em função da greve da educação.
As ocupações escolares foram um fim em si mesmo, cada qual como uma
célula de poder popular de baixo para cima. Nesta forma de luta,
os meios são já o fim do que acreditamos ser potencialmente revolucionário.

\asterisc

É quando se tenta tirar tudo de alguém que este alguém fica diante do
que não lhe pode ser tirado ou destruído. Isso é o que sobrevive à
própria morte. Situações limites nos mostram o que é indestrutível, isto
é, algo que nos ultrapassa e pelo qual vale a pena morrer, pois alça
nossa existência mortal ao valor semântico da imortalidade. A busca por
algo assim é o que permite dar sentido à vida.

Em toda a tradição
ocidental, desde Platão, procura"-se situar este modo de compreender a
condição humana enquanto ser mortal, finito, na separação entre teoria e
prática, colocando"-se o modo próprio de alcançar o valor semântico
aludido em algo com o caráter teórico. A teoria aproximaria o ser humano
de Deus, ou melhor, seria o divino no humano. Isso retira muito (como
ressaltou em sua obra Hannah Arendt) da dimensão ontológica da política
para o ser humano. Ora, quando se coloca o modo humano de se eternizar
como estritamente teórico, se exila o ser humano --- este ser no qual
toda prática é política, já que ele age socialmente, entre outros seres
humanos, e, portanto, age sempre com o valor geral, em princípio, que se
espera encontrar na teoria --- justamente do aspecto político da sua
ação. Assim, exila"-se o ser humano de algo muito fundamental para ele, o
fato de que sua prática é sempre teórica. A separação da teoria e da
prática desempenha um exílio na condição humana. Mas ela é fundamental
quando se quer exilar tantos do seu direito à participação política.
Este exílio é fundamental para legitimar o sistema no qual vivemos, no
qual alguns são colocados como vanguarda intelectual, representantes
passíveis de ter poder de decisão e compreender a realidade, e outros
são tomados como não pensantes.

Mas é preciso dizer mais do que isso, é preciso perguntar ainda o que é
uma revolução. E revolução é o que transforma o impossível em possível.
Não podemos esquecer que toda mudança em uma forma de vida é uma mudança
também na linguagem, em suas normas, em seus fundamentos. E como na
linguagem se pode mudar aquilo que é normativo desconsiderando sua
relação com o concreto? Como se muda o necessário? Somente incidindo
diretamente nesta relação entre o abstrato e o concreto. Só através do
diálogo entre concreto e abstrato, ou, dito de outro modo, só com um
significado que é tomado como concreto é que o concreto pode ganhar
novo significado, e é preciso que isso seja dado como um alargamento de
possibilidades. Trata"-se, portanto, de uma transformação na semântica. É
preciso que algo que tinha um valor de necessidade seja tomado como
apenas uma possibilidade, na medida em que algo inconcebível se torna
real: esta não é uma modificação cotidiana, ordinária, mas uma modificação
nos limites do mundo. Não se trata de mudar uma imagem, trata"-se de
mudar o arranjo entre o que é pano de fundo e o que é figura neste pano
de fundo. Uma vez realizada tal transformação, também deixa de haver
caminhos para o que existia antes. Portanto, não há um caminho dado no
mundo tal como está que nos possa levar à revolução, a mudança que
queremos é uma mudança nos próprios critérios de significação, é um
alargamento dos limites do que é um fato, é uma modificação no que se
toma como necessário. Também por isso a possibilidade da revolução não
pode ser vislumbrada de dentro da nossa sociedade tal como ela se
encontra hoje, somente em alguns momentos, como quando ela se revira em
um levante popular e vemos o que antes parecia impossível acenar no
horizonte. Por isso, alguns são capazes de olhar o mundo de outro modo,
de mudar o seu arranjo estrutural. E esta mudança no modo de ver não é
uma idealidade, mas é antes de tudo uma prática.

\chapter{Crise e guerra permanentes}

Uma das maneiras de compreender a Modernidade é através da
noção de crise, estabelecida como
\emph{modus operandi} da sociedade. Longe de uma ruptura com este modo
de vida, o significado da crise é o próprio coração do homem moderno,
uma sociedade que coloca seus próprios fundamentos em questão e cujo
aprofundamento das suas contradições não pode senão ser visto como um maior
enraizamento de um certo projeto de mundo que se alimenta das suas
crises sucessivas e, em grande medida, programadas. No \emph{Dicionário de
filosofia} de Nicola Abbagnano encontramos:

\begin{quote}
Para St. Simon, assim como para Comte e muitos positivistas, toda a
época moderna é de crise, no sentido de não ter ainda atingido sua
organização definitiva em torno de um princípio único, que deveria ser
dado pela ciência moderna, mas, inevitavelmente, encaminha"-se para a
realização dessa organização. Esse diagnóstico depois foi compartilhado
por todos os filósofos e políticos que se portaram como profetas de
nosso tempo. Tanto os que acham que a nova e indefectível era orgânica
será o comunismo quanto os que acham que essa época será caracterizada
pelo misticismo estão de acordo em diagnosticar a crise da época
presente e em indicar sua falta de organicidade, ou seja, de
uniformidade nos valores e nos modos de vida\footnote{\versal{ABBAGNANO},
  Nicola. \emph{Dicionário de filosofia}. Trad. Alfredo Bosi. São Paulo:
  Martins Fontes, 2000, p. 222.}.
\end{quote}

Se é correto dizer isso da Modernidade como um todo, o que dizer do
momento histórico no qual nos encontramos? Vivemos no
Brasil a transição de um projeto político de conciliação de classes, com
um verniz espetacular nacional"-desenvolvimentista e social"-democrata, que
pelo menos na propaganda, apesar de neoliberal, parecia pretender amenizar
os efeitos das desigualdades através de programas sociais, para um
projeto político explicitamente neoliberal de Estado mínimo, no qual a
propaganda é de que a necessidade de um suposto equilíbrio das contas
públicas deve necessariamente se traduzir em cortes de direitos básicos
e a adoção de pacotes econômicos de austeridade.

Ao lado disso, o efeito colateral acarretado com a transição forçada pelas elites
econômicas de um modelo para o outro foi a clarificação do
papel secundário da administração política do Estado em relação aos
interesses econômicos desta elite, deixando evidente ainda como a
máquina pública funciona e gerando o aumento da rejeição da
política partidária por parte da população. Se, para forçar a transição
de um modelo a outro, foi necessário deixar evidentes os esquemas de
corrupção, tal evidência não se aplica apenas a um modelo ou outro, mas
incide diretamente sobre as instituições como um todo.

Com o acirramento das tensões e desigualdades sociais lado a lado com a
crescente recusa e descrença nas vias institucionais democráticas, não
podemos deixar de presenciar também o aumento de elementos conservadores e, com
isso, o surgimento de uma nova e raivosa direita, que de
nova não tem tanto assim, já que requenta nada mais nada menos do que
elementos fascistas. Sobre isso, Eric Alliez e Maurizio Lazzarato
afirmam:

\begin{quote}
Do lado do poder, o neoliberalismo, para melhor acender o fogo das suas
políticas econômicas predatórias, introduziu uma pós"-democracia
autoritária e policial gerida por técnicos do mercado, enquanto a nova
direita desdiabolizada declara guerra ao estrangeiro, ao imigrante, ao
muçulmano. É esta nova direita que se instala abertamente no terreno da
guerra civil e que relança uma guerra racial de classe\footnote{\versal{ALLIEZ}, Eric;
  \versal{LAZZARATO}, Maurizio. \emph{Guerres et capital}. Paris: Ed. Amsterdam,
  2016.}.
\end{quote}

O discurso que mais uma vez acompanha estes acontecimentos, e que em
grande medida justifica a adoção das aludidas medidas impopulares é o da
crise, crise econômica, crise das instituições, crise da representação.
Paralelamente, e complementarmente, vemos o aumento do Estado policial,
o controle do território nas favelas e periferias, o genocídio dos
matáveis, a gestão dos que podem viver e a exclusão dos que devem
morrer. Afirmou Foucault sobre o racismo de Estado:

\begin{quote}
Vocês compreendem, em consequência, a importância do racismo no
exercício do poder assim: é a condição para que se possa exercer o
direito de matar. Se o poder de normalização quer exercer o velho
direito soberano de matar, ele tem que passar pelo racismo. E se,
inversamente, um poder de soberania, ou seja, um poder que tem direito
de vida e de morte, quer funcionar com intrumentos, mecanismos e com
tecnologias da normalização, ele também tem de passar pelo racismo. É
claro, por tirar a vida não entendo simplesmente o assassínio direto,
mas também o indireto: o fato de expor à morte, de multiplicar o risco
de morte, a morte política, a exclusão, a rejeição\footnote{\versal{FOUCAULT},
  Michel. \emph{Em defesa da sociedade: curso no Collége de France
  (1975-1976)}. Trad. Maria Ermantina Galvão. São Paulo: Martins
  Fontes, 1999, p. 306.}.
\end{quote}

A situação não é nova. Nem se trata de fenômeno fundamentalmente
brasileiro, mais uma vez o Brasil parece herdar tardiamente os ares da
Europa. O cenário é similar ao vivido na Grécia desde 2008, uma crise do
sistema financeiro que levou ao arrocho fiscal e adoção do pacote de
medidas neoliberais. E em grande medida é o mesmo discurso que justifica
ainda as reformas na França: reforma trabalhista, reforma da
previdência, medidas crescentes para a exclusão dos imigrantes, controle
do território nas periferias etc.

A principal posição que eu gostaria de defender aqui é aquela segundo a
qual a crise hoje, a crise na contemporaneidade, não é um acidente
superável, mas uma fatalidade programada. Não se trata de uma crise do
capitalismo, um momento de oportunidade para seu aperfeiçoamento; nem da
crise como prevista por uma certa leitura do determinismo histórico,
isto é, a crise que anunciaria, finalmente, o fim do capitalismo e, com
isso, a revolução e o início de um novo tempo. A reflexão que eu
gostaria de trazer é um pouco mais desesperada e mais pessimista, mas eu
espero que possamos tirar da constatação do absurdo algo para
transvalorá"-lo, sem escondê"-lo, como nos propôs Albert Camus.

\begin{quote}
A absurdidade essencial dessa catástrofe não muda nada do que ela é. Ela
generaliza a absurdidade um pouco mais essencial da vida. Ela a torna
mais imediata e mais pertinente. Se esta guerra pode ter um efeito sobre
o homem, é o de fortificá"-lo na ideia que ele faz de sua existência e no
julgamento que tem sobre ela. A partir do instante em que esta guerra é,
todo julgamento que não pode integrá"-la é falso. Um homem que reflete
passa geralmente seu tempo a adaptar a ideia que formou das coisas aos
novos fatos que a desmentem. É nessa inclinação, nessa deformação do
pensamento, nessa correção consciente, que reside a verdade, ou seja, o
ensinamento de uma vida. É porque mesmo sendo esta guerra tão ignóbil,
não é permitido estar fora dela. Para mim, que posso arriscar minha vida
apostando em uma morte sem um medo sequer. E para todos aqueles,
anônimos, e resignados, que vão para essa matança imperdoável --- e pelos
quais eu sinto toda a fraternidade\footnote{\versal{CAMUS}, Albert. \emph{A Guerra começou, onde está a guerra?} Trad. Raphael Araújo e Samara
  Geske. São Paulo: Hedra, 2014, pp. 22-23.}.
\end{quote}

De fato, a crise neste momento é, como tudo mais, um
espetáculo\footnote{Entendemos ``espetáculo'' aqui no sentido cunhado por
  Guy Debord, isto é, como representação sem representado, ou seja,
  ruptura do paradigma representacional elevado a enésima potência. \versal{DEBORD}, Guy. \emph{Sociedade do espetáculo}. Rio de Janeiro:
  Contraponto, 2000.} programado. E que estejamos vivendo em espetáculos
programados é talvez o sentido verdadeiro da crise. Mas trata"-se também de
um espetáculo que serve como arma, como máquina de guerra e como
justificativa para a guerra continuada na qual nos encontramos. Já em
1984, Guattari previa uma crise semiótica e, portanto, uma crise no
paradigma representacional da modernidade, diretamente relacionada com
uma crise dos modos de vida e subjetivações no capitalismo pós"-Guerra
Fria.

\begin{quote}
Dá para estimar que o essencial dessa crise mundial (que é, ao mesmo
tempo, uma espécie de guerra social mundial) é a expressão da gigantesca
ascensão de toda uma série de camadas marginalizadas, por toda a
superfície do planeta. (\ldots{}) Não se trata mais daquilo que se
chamava tradicionalmente de ``crises cíclicas do capitalismo''. É uma
crise de modos de relação entre, de um lado, os novos dados da produção,
os novos dados de distribuição, as novas revoluções dos meios de
comunicação de massa e, de outro lado, as estruturas sociais, que
permaneceram totalmente cristalizadas, esclerosadas, em suas antigas
formas. Os poderes de Estado são tanto mais reacionários quanto mais
aguda é sua consciência de que estão sentados em cima de uma verdadeira
panela de pressão que eles não conseguem mais controlar\footnote{\versal{GUATTARI},
  Félix. \emph{Essa crise que não é só econômica.} Disponível em:
  <https://machinedeleuze.wordpress.com/2017/05/03/essa-crise-que-nao-e-so-economica-por-felix-guattari/>.}.
\end{quote}

O que há de profético no texto de Guattari
é compreender os modelos institucionais da social"-democracia e do
neoliberalismo enquanto fadados ao fracasso justamente por não
reconhecerem o caráter semiótico"-existencial envolvido na adoção da
crise como modo de governo. Dito de outro modo, por não compreenderem
que a verdadeira crise envolvida na adoção da crise como espetáculo é
uma crise da representação, o que tem um aspecto ontológico e um aspecto
semântico, estando inserida no contexto de uma guerra social permanente
e crescente.

\begin{quote}
Mas essa crise explodiu mesmo, incontestavelmente, a partir de 1974, e
desde então a saída do túnel nos é anunciada a cada ano. No entanto, ao
contrário, tudo leva a crer que se trata de um desafio, em escala
internacional, e para todo um período da História. Crise que poderíamos
chamar também de guerra --- uma guerra mundial --- com a diferença de que
não está sendo uma guerra atômica (apesar de essa possibilidade não
estar excluída), mas uma sucessão de guerras locais sempre em torno
desse eixo Norte"-Sul. E, por fim, o terceiro tipo de atitude {[}diante
da crise{]}. Ao contrário das duas atitudes precedentes {[}do
neoliberalismo e da socialdemocracia{]}, nesse caso considera"-se para
valer as mutações subjetivas, tanto do ângulo de seu caráter específico,
quanto de seu traço comum --- trata"-se de diferentes formas de
resistência molecular, que atravessam as sociedades e os grupos sociais,
contra as quais se choca essa tentativa de controle social em escala
planetária\footnote{\emph{Idem, Ibidem.}}.
\end{quote}

O capitalismo não está em crise, como nós talvez gostaríamos de pensar,
mas vivemos o triunfo do capitalismo de crises. As instituições não
estão em crise, as instituições instituem as crises e se alimentam dela.
São crises programadas, crises continuadas, crises para gerar
governamentalidade. A noção de governamentalidade surge nos cursos de
Foucault do início de 1978. Trata"-se de analisar as diversas maneiras de
estabelecer condições para controlar os outros e a si mesmo, um estudo
das técnicas que permitiram, desde o século \versal{XVI}, governar e ser
governado. Neste momento, Foucault entende que há uma continuidade entre
o governo moral de si, o governo econômico da família e o governo
político do Estado, passando a defender, portanto, o Estado de modo
imanente à micropolítica. Trata"-se de gerir as populações através da
economia, nos discursos, e do controle policial, nas práticas:

\begin{quote}
Por ``governamentalidade'' entendo o conjunto constituído pelas
instituições, procedimentos, análises e reflexões, os cálculos e as
táticas que permitem exercer essa forma bem específica, ainda que
complexa, de poder que tem por alvo principal a população, por forma
maior de saber a economia política e por instrumento técnico essencial
os dispositivos de segurança. Segundo, por ``governamentalidade'' entendo
a tendência, a linha de força que, em todo o Ocidente, não cessou de
conduzir, e desde muito tempo, à preeminência desse tipo de poder que
podemos chamar de ``governo'' sobre todos os outros: soberania,
disciplina, e que, por uma parte, levou ao desenvolvimento de toda uma
série de aparelhos específicos de governo {[}e, de outra parte{]}, ao
desenvolvimento de toda uma série de saberes\footnote{\versal{FOUCAULT},
  Michel. \emph{Sécurité, territoire, population: cours au Collège de
  France. 1977-1978}. Paris: Gallimard/Seuil, pp. 111-112.}.
\end{quote}

Mas o que esta noção de \emph{governamentalidade} introduzida por
Foucault tem com a crise aludida e a guerra continuada? Ora, a crise
anunciada é uma tática para a manutenção do Estado, para que este nos
pareça necessário, para que implemente suas medidas como que desejadas
pela população, para que se busque organicidade, para que se procure
``univocidade nos valores e modos de vida''. O que no âmbito dos saberes
aparece como discurso econômico, mas que se introduz ainda mais
fortemente como Estado Policial, a política de segurança requerida numa
guerra de todos contra todos, o mal necessário diante do temor pelo
aumento da tensão social.

\begin{quote}
A cada momento, são as táticas de governo que permitem definir o que é
da competência do Estado e o que permanece fora dela, o que é público e
o que é privado, o que é estático e o que é não"-estático. Assim
{[}\ldots{}{]}, só é possível compreender a sobrevivência e os limites do
Estado levando em conta as táticas gerais da
governamentalidade\footnote{\emph{Idem, Ibidem.}}.
\end{quote}

A crise se torna estratégia de governamentalidade na medida em que
incide nas subjetividades, instaurando o desejo pelas medidas que se
pretende adotar, então anunciadas como remédios para a crise tão temida.
Gere-se, então, mais endividamentos na população e, com isso,
gerações e gerações de escravos pacatos no porvir. A escassez e o medo
da piora como modo de governo.

É fácil notar que sem o discurso da crise, não seria possível aprovar a
reforma da previdência, nem privatizar a \versal{CEDAE}, acabar com os
direitos trabalhistas, reformar a educação, suspender direitos
básicos, fechar a \versal{UERJ}, cortar os salários dos funcionários
públicos\ldots{} Mas é preciso notar também que esse contexto,
não nos sendo particular, é um projeto internacional. As mesmas medidas foram e estão
sendo adotadas em vários lugares do mundo. Tomemos os comentários do
Comitê Invisível sobre a recente situação grega:

\begin{quote}
``Onde e quando'' é uma questão de oportunidade ou de necessidade
tática. É de conhecimento público que, em 2010, o recém"-nomeado diretor
do Instituto Grego de Estatísticas (Elsat) falsificou continuamente as
contas da dívida do país, tornando"-as mais graves e dando, assim,
justificativas para a intervenção da Troika. \emph{É fato, portanto,
que a crise das dívidas soberanas foi iniciada por um homem que, à
época, ainda era uma agente remunerado oficial do \versal{FMI}, instituição que
supostamente iria ajudar os países a saírem da dívida. Tratava"-se ali de
experimentar, em escala real, num país europeu, o projeto neoliberal de
reformulação completa de uma sociedade, os efeitos de uma boa política
de ``ajustamento estrutural''}\footnote{\versal{COMITÊ INVISÍVEL}. \emph{Aos
  nossos amigos: crise e insurreição}. Trad. Edições Antipáticas.
  São Paulo: \versal{N}-1 edições, 2016, p. 24.}.
\end{quote}

A crise toma o papel de um inimigo interno temido contra o qual devemos
nos unir e, juntamente com a violência urbana, cria o cidadão temeroso
que aceita e deseja o governo como um mal necessário, mantendo
pacatas as populações. Assim, a crise é permanente, e sem fim. Ela não
é, portanto, fundamentalmente econômica, o fim dos mercados dos
liberais, como alguns têm defendido. Ela é antes de tudo uma escolha de
governo pautada no controle populacional que serve como modalidade discursiva na
produção de saberes.

\begin{quote}
Quando se corta pela metade o vencimento dos funcionários públicos
gregos, isso é feito sob o argumento de que seria possível nunca mais
lhes pagar. A cada vez que se aumenta o tempo de contribuição dos
assalariados franceses para a seguridade social, isso é feito sob
pretexto de ``salvar o sistema de aposentadorias''. A crise presente,
permanente e omnilateral, já não é a crise clássica, o momento decisivo,
pelo contrário, ela é um final sem fim, apocalipse sustentável,
suspensão indefinida, diferimento eficaz de afundamento coletivo e, por
tudo isso, Estado de exceção permanente\footnote{\emph{Idem}, p. 28.}.
\end{quote}

Não se trata de um desgoverno, o que se espera de fato é a implicação da
crise em mais governo, como modo de manter o controle da sociedade. Mas
não que não se tenha que produzir inimigos, é fundamental culpar alguém
pela crise e também aqui a culpa será sempre do ``outro'', do já
excluído, do imigrante, do desempregado\ldots{} Assim a crise instaura uma
guerra de todos contra todos, guerra civil em todas as instâncias da
sociedade, ódio generalizado aos possíveis culpados. É a crise fabricada
como tática para gerar governamentalidade e a crise planejada como arma
em uma guerra continuada e não declarada.

Talvez Foucault tenha sido quem melhor analisou na contemporaneidade o
esquema contratualista de poder moderno surgido com as revoluções
burguesas a partir do século \versal{XVIII}. Se arrogando oposto ao exercício de
poder pela guerra, ainda que sendo ele mesmo uma tática de guerra,
separaria usos legítimos de usos ilegítimos da violência apenas como
modo estratégico de combate, para manter o exercício continuado da
violência nas mãos de quem está vencendo a guerra. Nesse sentido,
Foucault desmascara e inverte a famosa ideia de Clausewitz segundo a qual ``a
guerra seria política continuada por outros meios'' --- uma política
degenerada --- e a inverte, mostrando que ``a política é que seria a
guerra continuada por outros meios'', uma parte do jogo de forças em uma
guerra contínua de fundo.

\begin{quote}
Se o poder é mesmo, em si, emprego e manifestação de uma relação de
força, em vez de analisá"-lo em termos de cessão, contrato, alienação, em
vez mesmo de analisá"-lo em termos funcionais de recondução das relações
de produção, não se deve analisá"-lo antes e acima de tudo em termos de
combate, de enfrentamento ou de guerra? Teríamos, pois, diante da
primeira hipótese --- que é: o mecanismo do poder é, fundamental e
essencialmente, a repressão ---, uma segunda hipótese que seria: o poder é
a guerra, é a guerra continuada por outros meios. E, neste momento,
inverteríamos a proposição de Clausewitz e diríamos que a política é a
guerra continuada por outros meios. O que significaria três coisas.
Primeiro isto: que as relações de poder, tais como funcionam numa
sociedade como a nossa, têm essencialmente como ponto de ancoragem uma
certa relação de força estabelecida em dado momento, historicamente
precisável, na guerra e pela guerra. E, se é verdade, que o poder
político para a guerra, faz reinar ou tenta fazer reinar uma paz na
sociedade civil não é de modo algum para suspender os efeitos da guerra
ou para neutralizar o desequilibrio que se manifestou no final da
batalha. O poder político, nessa hipótese, teria como função reinserir
perpetuamente essa relação de forças, mediante uma espécie de guerra
silenciosa e de reinseri"-la nas instituições, nas desigualdades
econômicas, na linguagem, até nos corpos de uns e de outros. Seria,
pois, o primeiro sentido a dar a esta inversão do aforismo de Clasewitz
(\ldots{}). E a inversão dessa proposição significaria outra coisa,
também, a saber: no interior dessa ``paz civil'', as lutas políticas, os
enfrentamentos a propósito do poder, com o poder, pelo poder, as
modificações nas relações de poder (\ldots{}) deveriam ser interpretadas
como as continuações da guerra''\footnote{\versal{FOUCAULT}, Michel, 1999, p. 23.}.
\end{quote}

Recentemente Eric Alliez e Maurizio Lazzaratto retomaram esta noção de
guerra permanente para analisar o momento presente. O livro \emph{Guerres
et capital} (Guerra e capital), recém"-publicado na França e sem tradução no
Brasil, defende que vivemos uma guerra econômica e política constante, de
intensidade variável, e que, mesmo quando não é
evidentemente sangrenta, incide diretamente sobre as populações. Os
autores pensam a economia como prolongamento da guerra e defendem que a
política das múltiplas guerras é a única forma de gerar governamentalidade hoje.

\begin{quote}
Nós vivemos no tempo da subjetivação das guerras civis. Não saímos do
período do triunfo do mercado, dos automatismos da governamentalidade e
da despolitização da economia da dívida para reencontrar a época das
concepções de mundo e seus afrontamentos, mas para entrar na era da
construção das novas máquinas de guerra.
(\ldots{}) ``é como uma guerra'' --- entendemos em Atenas durante o
final de semana do 11 e 12 de julho de 2015. Com a razão. A população
foi confrontada com uma estratégia em grande escala de continuação da
guerra pelos meios da dívida: esta guerra concluiu a destruição da
Grécia e, de um mesmo golpe, disparou a autodestruição da União
Europeia. (\ldots{}) O anúncio ``é como uma guerra'' é uma metáfora que é
preciso retificar: é uma guerra. A reversibilidade da guerra e da
economia está no fundamento mesmo do capitalismo: a economia persegue os
objetivos da guerra por outros meios (o bloqueio do crédito, o embargo
das matérias"-primas, a degradação da moeda estrangeira)\footnote{\versal{ALLIEZ}, Eric;
  \versal{LAZZARATO}, Maurizio, 2016.}.
\end{quote}

Complementarmente, uma das maiores armas da fase atual do capitalismo é
esconder a guerra, ocultá"-la para que se
possa continuar com o monopólio da ofensiva. Em conflagrações declaradas
existem leis de guerra e aqueles que são atacados têm direito à
autodefesa. Já a guerra não declarada é parcial:
para os atacados (as populações) estamos no Estado Democrático de
Direito, a lei vale, e qualquer ação de resistência será julgada
criminosa; porém para os que atacam (o Estado) a lei é suspensa, e com
isso mantém"-se o monopólio do extermínio. É exatamente neste sentido
que a própria lei não é mais do que uma arma ou uma
estratégia que visa manter eterna a vitória conseguida por meio das
batalhas, e não uma ruptura com a lógica do enfrentamento:

\begin{quote}
Contrariamente ao que diz a teoria filosófico"-jurídica, o poder político
não começa quando cessa a guerra (\ldots{}) a lei não nasce da natureza,
junto das fontes frequentadas pelos primeiros pastores, a lei nasce das
batalhas reais, das vitórias, dos massacres, das conquistas que têm suas
datas e seus heróis de horror, a lei nasce das cidades incendiadas, das
terras devastadas, ela nasce com os famosos inocentes que agonizam no
dia que está amanhecendo. (\ldots{}) A lei não é pacificação, pois sob a
lei, a guerra continua a fazer estragos no interior de todos os
mecanismos de poder, mesmo os mais regulares\footnote{\versal{FOUCAULT}, Michel, 1999, pp. 58"-59.}.
\end{quote}

Sendo assim, quanto mais nossa ``política'' puder esconder a guerra e
naturalizá"-la, mais ela continua mantendo o exercício continuado da
violência nas mãos de quem está vencendo. Por isso, iniciar uma
autodefesa passa pelo campo dos saberes, pela construção dos discursos, e
por se evidenciar que estamos de fato em guerra.

\begin{quote}
A sociedade em sua estrutura política é organizada de maneira que alguns
possam se defender contra os outros, ou defender sua dominação contra a
revolta dos outros ou simplesmente ainda defender sua vitória e
perenizá"-la na sujeição\footnote{\emph{Idem}, p. 26.}.
Temos de redescobrir a guerra, por quê? Pois bem, porque essa guerra
antiga é uma guerra permanente. Temos de fato de ser os eruditos das
batalhas, porque a guerra não terminou, as batalhas decisivas ainda
estão se preparando (\ldots{}) Isto quer dizer que os inimigos que estão
a nossa frente ainda continuam a ameaçar"-nos, e não poderemos chegar ao
termo da guerra por algo como uma reconciliação (\ldots{})\footnote{\emph{Idem},
  p. 60.}.
\end{quote}

Vivendo no Rio de Janeiro, onde pessoas são executadas diariamente nas
favelas, onde os autos de resistência são largamente utilizados, onde as
milícias dominam regiões inteiras ditando quem deve morrer e quem pode
viver, não é difícil ver como isso opera. Nossa estatística de jovens
negros mortos nas favelas é hoje maior do que em muitas áreas de
conflito declarado, o esforço discursivo para manter essa guerra velada
é sobretudo o esforço discursivo para mantê"-la parcial. Não existe
nenhum critério coerente estabelecido para o uso da violência
considerado legítimo e ilegítimo senão o princípio básico de qualquer
guerra: ``para nossos aliados, tudo; para nossos inimigos, nada''.

Como lidar então com esta crise que se apresenta como arma em uma guerra
permanente e como resistir nesta guerra na qual a crise é expediente ao
mesmo tempo bélico e discursivo? Os meios tradicionais de luta e a via
institucional não são capazes de responder a esta questão. E se a esquerda
não souber respondê"-la, restará como alternativa de ruptura apenas o
advento crescente do fascismo espetacular com o qual agora lidamos. É
neste sentido que a crise, como nos dizia já Guattari, é muito mais
profunda do que pode dar conta a dicotomia neoliberalismo e
social"-democracia, pois coloca todo o nosso modo de vida em questão.

\begin{quote}
É justamente porque os movimentos de esquerda sindicais tradicionais
viveram essa situação unicamente em termos de crise econômica, que o
conjunto dos movimentos de resistência social ficaram totalmente
desarmados. E, na ausência de respostas, foram as formações mais
reacionárias que tomaram conta da situação\footnote{\versal{GUATTARI}, Félix.
  \emph{Essa crise que não é só econômica}.}.
\end{quote}

A social"-democracia é pensada por Guattari como uma forma de tentar
controlar os movimentos sociais que se inserem nas novas
subjetividades e resistências, mas que falha totalmente neste aspecto.
De acordo com essa lógica, tem"-se argumentado em vários lugares que a crise aludida
é, portanto, fundamentalmente uma crise da representação, por sua elevação à enésima potência de
representação sem representado. É uma crise da representação por domínio
generalizado do espetáculo, representação da própria representação, e é
apenas nesse sentido que a crise toma vida própria como manutenção do
modo de vida que ao mesmo tempo coloca em questão, e é também apenas
nesse sentido que podemos entender que ela se trata também de uma crise
existencial.

\begin{quote}
Esta catástrofe é, acima de tudo, existencial, afetiva, metafísica.
Reside na incrível estranheza do homem ocidental em relação ao mundo,
estranheza que exige, por exemplo, que ele se faça amo e possuidor da
natureza --- só se procura dominar aquilo que se teme. Não foi por acaso
que ele colocou tantas telas entre si e o mundo. Ao se subtrair do
existente, o homem ocidental criou essa extensão desolada, esse nada
sombrio e hostil, mecânico, absurdo que ele tem que transformar
incessantemente por meio de seu trabalho (\ldots{}). A mentira de todo e
qualquer apocalíptico ocidental consiste em projetar sobre o mundo o
luto que nós não lhe podemos fazer. Não foi o mundo que se perdeu, fomos
nós que perdemos o mundo e o perdemos sem parar; não é ele que em breve
vai acabar, somos nós que estamos acabados, amputados, cortados, nós que
recusamos alucinadamente o contato vital com o real. A crise não é
econômica, ecológica ou política, a crise é antes de tudo crise da
\emph{presença}\footnote{\versal{COMITÊ INVISÍVEL}, 2016, p. 35.}.
\end{quote}

Tal como os autores de \emph{Crise e insurreição} analisam, nosso modo
de vida consiste numa fuga perpétua para o mundo virtual, para o mundo
das imagens e das representações que se tornaram mais reais do que a
própria realidade, isto é, são representações espetaculares e que,
portanto, não funcionam. Este virtual não mais corresponde ao real, mas
se torna mais importante do que ele e opera no vazio. A impressão de se
poder estar em todos os locais ao mesmo tempo foge à condição
espaço"-temporalmente situada do humano concreto, e nos deixa, assim, sem
mundo. É nesse sentido que a crise existencial é também a crise da
representação.

As respostas à altura da conjuntura, que rompem com as velhas
dicotomias requentadas apenas no nível espetacular, são sempre mais ou
menos inssurrecionárias, no sentido em que são imprevisíveis. Um bom
exemplo disso foi 2013, um outro bom exemplo foram as ocupações das
escolas por parte dos secundaristas. Estas alternativas foram bem
sucedidas porque funcionaram como táticas de guerra que pegam o inimigo
desprevenido, muito mais do que como programa político estruturado.
Afinal, ainda em Foucault: ``a inversão do aforisma de Clausewitz nos
diz ainda uma terceira coisa: a decisão final só pode vir da
guerra''\footnote{\versal{FOUCAULT}, Michel, 1999, p. 23.}. Ou entendemos de uma vez por
todas que estamos em uma guerra e que nossas ações funcionam como parte
de enfrentamentos permanentes no jogo de forças, inclusive nossas
práticas discursivas, ou continuaremos a perder a guerra.

As experiências que romperiam com a falência existencial na qual nos
encontramos seriam aquelas capazes de romper também com o primado da
representação e da compreensão dualista de realidade que a acompanha.

\begin{quote}
Há nas inssurreições contemporâneas algo que desconcerta de modo
particular: elas não partem mais de ideologias políticas, mas de
verdades éticas. Aqui estão duas palavras cuja aproximação soa como um
oximoro a qualquer espírito moderno. Estabelecer o que é verdadeiro é o
papel da ciência, não é mesmo? A ciência, esta que não tem nada a ver
com as normas morais e com outros valores contingentes. Para os
modernos, há o Mundo de um lado, eles de outro, e a linguagem para
superar o abismo. Uma verdade, conforme nos ensinaram, é um ponto sólido
sobre o abismo --- um enunciado que descreve de maneira adequada o Mundo.
Convenientemente esquecemos a longa aprendizagem ao longo da qual
adquirimos, com a linguagem, uma relação direta com o mundo. A linguagem,
longe de servir primariamente para descrever o mundo, ajuda"-nos
sobretudo a construir um. As verdades éticas não são, assim, verdades
sobre o Mundo, mas as verdades a partir das quais neles permanecemos.
São verdades, afirmações, enunciadas ou silenciosas que se experimentam,
mas não se demonstram. (\ldots{}) São verdades que nos ligam, a nós
mesmos, ao que nos rodeia e uns aos outros. Elas nos introduzem de
imediato numa vida comum, a uma experiência não separada, sem
consideração pelos muros ilusórios do nosso Eu\footnote{\versal{COMITÊ INVISÍVEL}, 2016, pp. 54-55.}.
\end{quote}

Trata"-se agora de nos voltarmos para as experiências insurrecionárias e
vermos o que podemos aprender com elas através da noção de
\emph{acontecimento}, isto é, de uma verdade que não se insere mais no
âmbito da correspondência, mas da ação e que é capaz de instaurar deste
modo novas potências reestabelecendo nossa \emph{ligação interna} com o
real concreto que a contemporaneidade espetacular havia nos feito
esquecer.

\chapter{Arte e política}

Incialmente, pode parecer que a chamada ``bela arte'' e a política são âmbitos
extremamente diversos, já que a primeira tradicionalmente é pensada como
uma atividade desinteressada, com um fim nela mesma\footnote{``Num juízo
  de que algo é belo, o sujeito não está fascinado pelo objeto nem
  instruído por sua perfeição; a relação de um tal juízo envolve a forma
  de finalidade num objeto nele mesmo à parte a representação de um
  fim''. \versal{KANT}, Immanuel. \emph{Crítica da faculdade do juízo.} Trad.
  Valério Rohden e Antônio Marques. Rio de Janeiro: Forense
  Universitária, 2008, § 17.}, enquanto a política é por definição uma
atividade interessada, sendo um meio para a gestão da vida
pública\footnote{``A ciência a qual cabe indagar qual deve ser a melhor
  constituição da Polis''. \versal{ARISTÓTELES}. \emph{A política}. São Paulo: Atena
  Editora, sd.}.

Por outro lado, são muitas --- ainda que controversas --- as relações
estabelecidas entre arte e política ao longo da história, desde a
chamada ``arte engajada'', isto é, explícita e voluntariamente à
serviço da uma ideologia (entendendo também que todo discurso é
ideológico e que, portanto, não poderia deixar de ser desta forma), até a
ação política pensada ela mesma como inserida ou pertencendo a um
movimento artístico.

Mas eu não gostaria de propor aqui fazer política com arte, e sim
de apontar uma relação interna, de pensar um pouco nossa
\emph{vida como obra de arte}, assim como quais relações esta noção tem com a
nossa resistência atual\footnote{``Temos a nossa dignidade suprema em
  nossa significação como obra de arte''. \versal{NIETZSCHE}, Friedrich. \emph{O nascimento da tragédia} (1872). Trad. J. Guinsburg. São Paulo: Companhia das Letras, 1992, p. 52.}.

Para tanto, me parece necessário pensar a arte não como uma labuta
distinta, separada das nossas demais ações cotidianas, pois apenas
dessa maneira é possível reencontrar uma noção de política que é também
um fim em si e não apenas um caminho para algo
prefigurado. Nesse sentido eu gostaria de retomar a seguinte afirmação
do filósofo Michel Foucault no período tardio de seu pensamento:

\begin{quote}
o que me surpreende é que em nossa sociedade a arte esteja relacionada
apenas aos objetos e nunca aos indivíduos e à vida; e, também, que a
arte esteja em um domínio especializado, o dos experts que são artistas.
Mas a vida de todo indivíduo não é uma obra de arte? Por que uma mesa ou
uma casa são objetos de arte, mas não as nossas vidas?\footnote{\versal{FOUCAULT},
  \versal{M}. \emph{Dits et écrits}. Paris: Gallimard, 4 vols, 1994, p. 617.}
\end{quote}

A tarefa de compreender a vida como obra de arte corresponde a um
``horizonte ético de ação'', ou, como Foucault denominava, um \emph{êthos
filosófico}, que se institui como a uma luta perpétua contra
os poderes hegemônicos. Tratam"-se de ações capazes de mudar os limites
do possível, ações concretas que modificam o que tomamos como abstrato,
portanto necessário. Este singular que
encarna o geral institui com isso uma verdade, que não é a verdade da
teoria confirmada, mas a verdade \emph{vivida}.

\begin{quote}
É preciso considerar a ontologia crítica de nós mesmos não certamente
como uma teoria, um doutrina, nem mesmo como um corpo permanente de
saber que se acumula; é preciso concebê"-la como uma atitude, um êthos,
uma via filosófica em que a crítica do que somos é simultaneamente
análise histórica dos limites que nos são colocados e prova de sua
ultrapassagem possível\footnote{\versal{FOUCAULT}, Michel. \emph{Ditos e
  escritos \versal{II}: arqueologia das ciências e história dos sistemas de
  pensamento}. Rio de Janeiro: Editora Forense Universitária, 2005, p. 351.}.
\end{quote}

Esta verdade estética sempre foi a noção de verdade que Nietzsche
identificou na arte, a verdade que não é do âmbito da correspondência,
que não é uma representação, mas que pode ultrapassá"-la e até mesmo
destituí"-la. A verdade da arte é justamente a que rompe com o conceito
ocidental de verdade. Este modelo cognitivo e ontológico ideológico é o
fundamento das sociedades modernas e funda"-se na separação rígida entre
um âmbito da realidade tomado como abstrato e um tomado como
concreto. Esta separação, que apareceu em Descartes com a divisão
entre uma substância pensante e uma substância material, tem como premissa básica, no âmbito
do conhecimento, a posição pela qual conhecer,
significar, organizar a multiplicidade empírica é, antes de tudo,
representar. Estabelecer conhecimentos é estabelecer poder, e, neste
paradigma, também o exercício do poder é visto como um correspondente
abstrato da realidade concreta.

Assim, a verdade se diz não de um acontecimento, mas de
um representante, e este não pode colapsar jamais com aquilo que
pretende representar. Ele é um meio, não um fim em si, embora seja dito
verdadeiro precisamente por corresponder a ele. Esta separação rígida
tem obviamente um limite, que aponta para a necessidade de diálogo
fulcral entre abstrato e concreto. Uma das expressões deste limite
aparece tradicionalmente na arte, constituindo
aquilo que se entende como verdade estética, por oposição a uma
verdade cotidiana, representacional ou científica. Mas não é apenas na
arte que encontramos este limite. Se ele não existisse, inclusive,
nenhuma mudança estrutural seria possível. O próprio paradigma da
representação funda"-se em uma verdade vivida, que não é do âmbito da
representação, mas da apresentação, e da ação direta. É essa política,
da \emph{ação direta}, que rompe com a representação enquanto \emph{ação
indireta}, e que portanto se aproxima da arte ou do que podemos chamar
aqui de uma vivência estética. Sobre isso, é particularmente claro o que
nos diz a literatura insurrecionalista do \emph{Comitê Invisível}:

\begin{quote}
Nenhuma ordem social pode se basear de modo duradouro no princípio de
que nada é verdadeiro. É preciso também sustentá"-la. A aplicação a tudo
do conceito de ``segurança'' nos tempos que correm exprime este projeto
de integrar nos próprios seres, nos comportamentos e nos locais, a ordem
ideal a qual estes já não estão dispostos a sujeitar"-se. ``Nada é
verdade'' não diz nada acerca do mundo, mas tudo acerca do conceito
ocidental de verdade. A verdade aqui não é entendida como um atributo
dos seres ou das coisas, mas da sua representação. É tida como
verdadeira a representação conforme a experiência. A ciência é, em
última instância, o império da verificação universal. Ora, todos os
comportamentos humanos, dos mais vulgares aos mais eruditos, se baseiam
numa base de evidências formuladas de forma desigual, sendo que todas as
práticas partem de um ponto onde as coisas e as suas representações
estão indistintamente colapsadas, e em todas as vidas entra uma dose de
verdade que ignora o conceito ocidental de representação. Daí que os
ocidentais sejam universalmente tidos, pelos que colonizaram, como
mentirosos e hipócritas. É por isso que pode até ser cobiçado o que eles
têm --- o avanço tecnológico --- mas nunca o que eles são, que se vê
justamente desprezado. Não se poderia ensinar Sade, Nietzsche e Artaud
nas Universidades, se essa noção de verdade que ultrapassa a mera
representação não tivesse sido antecipadamente desqualificada. Conter ao
infinito todas as afirmações, mas sempre como letra morta, desativar
passo a passo todas as certezas vividas, este é o longo trabalho da
inteligência ocidental. Assim, polícia e filosofia podem tornar"-se meios
convergentes, ainda que formalmente distintos\footnote{\versal{COMITÊ}
  \versal{INVISÍVEL}. \emph{A insurreição que vem}, 2010, p. 101-102.}.
\end{quote}

Neste sentido, a verdade estética é a verdade como ato, como
acontecimento, que instaura uma nova relação com a realidade e um novo
horizonte de possibilidades. Porque a verdade na arte não é a verdade da
representação, não é a verdade da ciência, é a verdade do colapso entre
meio e mensagem, entre o que se usa pra dizer e aquilo que se diz. Toda
transformação fundamental em uma forma de vida
alarga os limites do possível e, portanto, é uma mudança também na
linguagem, em suas normas, em seus fundamentos. E como se muda o
necessário? Somente incidindo diretamente nesta relação entre o abstrato
e o concreto. Dito de outro modo, só com um significado que é tomado
como concreto é que o concreto pode ganhar novos significados. Aqui
talvez encontremos o cerne do que vem a ser a criação artística e, ao
mesmo tempo, as ações de resistência política. É preciso que isso seja
dado como um alargamento de possibilidades. É preciso que algo que tinha
um valor de necessidade seja tomado como apenas uma contingência. Na
medida em que algo inconcebível se torna real, esta não é uma
modificação cotidiana, ordinária, mas uma que, poderíamos
dizer, alarga os limites do nosso mundo. Não se trata assim de mudar uma
imagem, de fazer uma reforma, de eleger um deputado, mas sim de mudar o
arranjo entre o que é pano de fundo e o que é figura neste pano de
fundo. Uma vez realizada tal transformação, também deixa de haver
caminhos para o que existia antes. Portanto, não há um caminho dado no
mundo tal como está que nos possa levar a uma revolução, a mudança que
queremos é uma mudança nos próprios critérios de significação, é um
alargamento dos limites do que é um fato, é uma modificação no que se
toma como necessário. Também por isso a possibilidade da revolução não
pode ser vislumbrada de dentro da nossa sociedade tal como ela se
encontra hoje, somente em alguns momentos, momentos artísticos, momentos
insurrecionais, quando a sociedade se revira em um levante popular e
vemos o que antes parecia impossível acenar no horizonte. Por isso, a
partir disso, alguns são capazes de olhar o mundo de outro modo, de
mudar o seu arranjo estrutural. E esta mudança no modo de ver não é uma
idealidade, mas é antes de tudo uma prática revolucionária, uma verdade
como \emph{acontecimento}.

\begin{quote}
Um encontro, uma descoberta nova, um vasto movimento de greve, um
tremor de terra: todo acontecimento produz uma verdade, ao alterar a
nossa maneira de estar no mundo. Inversamente, uma constatação a qual
ficamos indiferentes, que não nos modifica, que não nos compromete,
ainda não merece o nome de verdade. Existe em cada gesto, em cada
prática, em cada relação, em cada situação, uma verdade subjacente que
não é do âmbito da representação e não poderia ser porque é constitutiva
daquilo que somos. (\ldots{}) Uma verdade não é uma visão de mundo
particular, mas o que nos mantém ligados ao mundo de forma irredutível.
Uma verdade não é algo que se detenha, mas algo que nos move. Ela faz"-me
e desfaz"-me, constitui"-me e destitui"-me, afasta"-me de muita coisa e
torna"-me parecido com aqueles que a experimentam. O ser isolado que a
ela se agarra encontra fatalmente alguns de seus semelhantes. Na
realidade, todo processo insurrecional parte duma verdade a qual não se
cede\footnote{\emph{Idem}, p. 112.}.
\end{quote}

Ora, essa é a ação política insurrecional, a política como um fim em si
e como tática de resistência. Podemos aqui pensar então no exemplo
concreto das recentes ocupações estudantis. Elas reivindicavam algo, mas
não eram apenas o meio para alcançar uma pauta, elas eram em si a
ação que pleiteavam, a escola autogerida, a célula social
horizontal. Esta é a política"-arte, a política fim em si e como estética
da existência.

Traduzir tais (im)possibilidades virtuais --- que não são e não podem ser
meras idealidades --- em efetividades demanda ações diferenciadas, que são
sempre ações políticas e podem ser concebidas como formas de
resistência: ``não há relação de poder sem resistência, sem escapatória
ou fuga, sem inversão eventual; toda a relação de poder implica, então,
pelo menos de forma virtual, uma estratégia de luta''\footnote{\versal{FOUCAULT},
  Michel. (1995) ``O sujeito e o poder''. In: \versal{DREYFUS, H}. e \versal{RABINOW, P}.
  (1995). \emph{Michel Foucault: uma trajetória filosófica para além do
  estruturalismo e da hermenêutica}. Trad. V. P. Carrero. Rio de
  Janeiro: Forense Universitária, p. 249.}. Mas é fundamental considerar
que a resistência diz respeito antes de tudo a uma atitude afirmativa de
um modo de existência. A resistência não quer simplesmente escapar ao
controle, ou ser ``contra'' suas potências geradoras, mas quer
pervertê"-lo, alterá"-lo, modificá"-lo.

A resistência jamais é apenas e fundamentalmente contra algo, mas
antes de tudo a favor das possibilidades de existência que afirma porque
cria concretamente a chance de que surjam outros ``modos de
experimentar'' a vida --- o que parece se relacionar intimamente com a
criação artística. A resistência cria valores e é ao mesmo tempo
imanente às nossas possibilidades concretas. Segundo Deleuze, ``o ato de
resistência possui duas faces. Ele é humano e é também um ato de
arte''\footnote{\versal{DELEUZE}. ``O ato de criação''. Palestra proferida em
  Paris em 1987, transcrita e publicada em \emph{Folha de São Paulo}, 27/06/1999, Caderno Mais!, p. 05.}.

Esta compreensão da própria dimensão política humana como portadora de uma
dimensão estética é também expressa na noção de revolta sustentada por
Camus, na medida em que esta se estabelece como uma recusa afirmativa,
isto é, que não renuncia ao absurdo constatado, mas que o transvalora.

\begin{quote}
Que é um homem revoltado? Um homem que diz ``não''. Mas se ele recusa,
não renuncia: é também um homem que diz sim, desde o seu primeiro
movimento. Um escravo que recebe ordens durante toda a sua vida, julga
subItamente inaceitável um novo comando. Qual o significado desse ``não''?
Significa por exemplo: ``as coisas já duraram demais''; ``até aí, sim, a
partir daí, não''; ``há um limite que você não vai ultrapassar''. Em suma,
este ``não'' afirma a existência de uma fronteira\footnote{\versal{CAMUS},
  Albert. \emph{O homem revoltado}. Rio de Janeiro: Record, 1999, p. 03.}.
\end{quote}

O modo mesmo como Camus lidava com o pensamento relaciona política e
arte, já que o autor usava de seus contos para mostrar seus conceitos.
Trata"-se, portanto, de conciliar em sua obra o singular com o geral, de
encarnar o universal. A apresentação de suas personagens aborda
conceitos instanciados nas vivências concretas. Mas que fazer diante do
absurdo? Se, por um lado, o sujeito não se mata diante do absurdo, é
preciso recusá"-lo por meio da revolta. Mas esta recusa é totalmente
afirmativa, porquanto é uma recusa que não renuncia ao absurdo da
existência. Revoltar"-se significa ir contra tudo aquilo capaz de
deteriorar, rebaixar, diminuir a condição humana: a miséria, a
morte ou a vida medíocre. É o próprio ser humano que se afirma em sua
negação, onde o sentimento de opressão se opõe a uma necessidade
interior de não se deixar oprimir. Esta negação é a própria afirmação da
existência e decorre do fato de todo o sentimento absurdo derivar de uma
constatação da ausência de sentido da própria vida.

O sentimento de revolta funda"-se tanto na negação de algo que se julga
intolerável quanto em uma certeza, por mais que confusa, da existência
de um direito efetivo. Isto é, na afirmação de um sentimento,
em se dar sentido a existência. A revolta aparece, então, como um
rebelar"-se que não está ligado a um valor pré"-existente no que diz
respeito ao indivíduo revoltado. Porém, toda revolta é criadora de
valores. O revoltado é aquele que contrapõe o que é aceitável ao que não
é, e portanto a revolta é também o que cria horizontes possíveis. Se
entendemos esta ação como a criação estética, entendemos também a
relação interna entre arte e política em uma vida como obra de arte.