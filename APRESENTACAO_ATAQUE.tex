\chapter*{}

\vspace*{\fill}

\epigraph{No domínio especializado da erudição, tanto como no saber
desqualificado das pessoas jazia a memória dos combates, aquela,
precisamente, que até então tinha sido mantida sob tutela. E assim se
delineou o que se poderia chamar uma genealogia, ou, antes, assim se
delinearam pesquisas genealógicas múltiplas, a um só tempo redescoberta
exata das lutas e memória bruta dos combates; e essas genealogias, como
acoplamentos desse saber erudito e desse saber das pessoas, só foram
possíveis, e inclusive só puderam ser tentadas, com uma condição: que
fosse revogada a tirania dos discursos englobantes, com suas hierarquias
e com todos os privilégios das vanguardas teóricas.}{(Michel Foucault, Aula de 7 de janeiro de 1976)\footnotemark}

\footnotetext{\versal{FOUCAULT}, Michel. \emph{Em defesa da sociedade: curso no Collége de France
  (1975-1976)}. Trad. Maria Ermantina Galvão. São Paulo: Martins
  Fontes, 1999.}

\chapter{Coleção Ataque Frontal}

\emph{Organizadores Acácio Augusto \& Renato Rezende}

\medskip

\noindent\versal{Editoras Circuito e Hedra}

\medskip

\noindent\versal{Conselho Editorial}: Acácio Augusto (Unifesp), Felipe Musetti
(Hedra), Jorge Sallum (Hedra), Renato Rezende (Circuito)

\bigskip

A coleção \emph{Ataque Frontal} irrompe sob efeito de junho de 2013.
Esse acontecimento recente da história das lutas sociais no Brasil, a um só
tempo, ecoa combates passados e lança novas dimensões para os
enfrentamentos presentes. O critério zero da coleção é o choque com os
poderes ocorrido durante as \emph{jornadas de
junho}, mas não só. Busca-se captar ao menos uma pequena parte do fluxo de
radicalidade (anti)política que escorre pelo planeta a despeito da
tristeza cívica ordenada no discurso da esquerda institucionalizada. Um
contra-fluxo ao que se convencionou chamar de onda conservadora. Os textos reunidos são, nesse sentido,
anárquicos, mas não apenas de autores e temas ligados aos
anarquismos. Versam sobre batalhas de
rua, grupos de enfrentamento das forças policiais, demolição da forma-prisão que
ultrapassa os limites da prisão-prédio. Trazem também análises sobre os
modos de controle social e sobre o terror do racismo de Estado. Enfim, temas de enfrentamento com
escritas que possuem um alvo. 

O nome da coleção foi tomado de um antigo
selo punk de São Paulo que, em 1985, lançou a coletânea \emph{Ataque
Sonoro}. Na capa do disco dois mísseis, um soviético e outro
estadunidense, apontam para a cidade de São Paulo, uma metrópole do que
ainda se chamava de terceiro mundo. Um anúncio, feito ao estilo audaz
dos punks, do que estava em jogo: as forças rivais atuam juntas contra o
que não é governado por uma delas. Se a configuração mudou de lá para
cá, a lógica e os alvos seguem os mesmos. Diante das mediações e
identidades políticas, os textos desta coleção optam pela tática do
ataque frontal, conjurando as falsas dicotomias que organizam a
estratégia da ordem. Livros curtos para serem levados no bolso, na
mochila ou na bolsa, como pedras ou coquetéis molotovs.
Pensamento-tática que anima o enfrentamento colado à urgência do
presente. Ao serem lançados, não se espera desses livros mais do que
efeitos de antipoder, como a beleza de exibições pirotécnicas. Não há
ordem, programa, receita ou estratégia a serem seguidos. Ao atacar
radicalmente a única esperança possível é que se perca o controle e,
como isso, dançar com o caos dentro de si. Que as leituras produzam
efeitos no seu corpo.

\section*{Títulos:}

\begin{Parskip}
Camila Jourdan. \emph{Chocolate e gás lacrimogênio: luta, prisão e
  resistência em 2013}.

Acácio Augusto. \emph{Desde junho: anarquia, antipolítica e terror de
  Estado no Brasil}.

Murilo Duarte Costa Correa. \emph{Filosofia black bloc.}

Carlos Taibo. \emph{Repensar a anarquia. ação direta, autogestão e
  autonomia.} (Traduzir do espanhol)

Gustavo Simões. \emph{O desconcerto anarquista de jonh cage.}

José Oiticica. \emph{Ação direta. Meio século de pregação libertária}.

Mark Bray. \emph{Antifa. um manual antifascista}. (Traduzir do inglês)

Hakim Bey. Caos. \emph{Terrorismo poético e outros crimes exemplares.}

Vários Autores. \emph{Anarquistas e as prisões}. (Coletânea a ser
  preparada)

Richard Day. \emph{Gramsci is dead. anarquia e movimentos pós-Seattle
  1999.}

Voltarine De Claire. \emph{Ação direta}. (tradução do inglês, segue em
  espanhol)
\end{Parskip}
