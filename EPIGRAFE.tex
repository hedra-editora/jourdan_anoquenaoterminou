\chapter*{}

\vspace*{\fill}

\thispagestyle{empty}

\epigraph{No domínio especializado da erudição, tanto como no saber
desqualificado das pessoas jazia a memória dos combates, aquela,
precisamente, que até então tinha sido mantida sob tutela. E assim se
delineou o que se poderia chamar uma genealogia, ou, antes, assim se
delinearam pesquisas genealógicas múltiplas, a um só tempo redescoberta
exata das lutas e memória bruta dos combates; e essas genealogias, como
acoplamentos desse saber erudito e desse saber das pessoas, só foram
possíveis, e inclusive só puderam ser tentadas, com uma condição: que
fosse revogada a tirania dos discursos englobantes, com suas hierarquias
e com todos os privilégios das vanguardas teóricas.}{(Michel Foucault, Aula de 7 de janeiro de 1976)\footnotemark}

\footnotetext{\versal{FOUCAULT}, Michel. \emph{Em defesa da sociedade: curso no Collége de France
  (1975-1976)}. Trad. Maria Ermantina Galvão. São Paulo: Martins
  Fontes, 1999.}
