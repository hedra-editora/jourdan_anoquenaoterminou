\chapter*{}

\vspace*{\fill}

\begin{flushright}
\begin{adjustwidth}{2.0cm}{}
\raggedleft\scriptsize\emph{A \versal{\emph{Coleção Ataque}} irrompe sob efeito de junho de 2013.
Esse acontecimento recente da história das lutas sociais no Brasil, a um só
tempo, ecoa combates passados e lança novas dimensões para os
enfrentamentos presentes. O critério zero da coleção é o choque com os
poderes ocorrido durante as \emph{jornadas de
junho}, mas não só. Busca"-se captar ao menos uma pequena parte do fluxo de
radicalidade (anti)política que escorre pelo planeta a despeito da
tristeza cívica ordenada no discurso da esquerda institucionalizada. Um
contrafluxo ao que se convencionou chamar de onda conservadora. Os
textos reunidos são, nesse sentido,
anárquicos, mas não apenas de autores e temas ligados aos
anarquismos. Versam sobre batalhas de
rua, grupos de enfrentamento das forças policiais, demolição da forma"-prisão que
ultrapassa os limites da prisão"-prédio. Trazem também análises sobre os
modos de controle social e sobre o terror do racismo de Estado. Enfim, temas de enfrentamento com
escritas que possuem um alvo.}

\emph{O nome da coleção foi tomado de um antigo
selo punk de São Paulo que, em 1985, lançou a coletânea \emph{Ataque
Sonoro}. Na capa do disco dois mísseis, um soviético e outro
estadunidense, apontam para a cidade de São Paulo, uma metrópole do que
ainda se chamava de terceiro mundo. Um anúncio, feito ao estilo audaz
dos punks, do que estava em jogo: as forças rivais atuam juntas contra o
que não é governado por uma delas. Se a configuração mudou de lá para
cá, a lógica e os alvos seguem os mesmos. Diante das mediações e
identidades políticas, os textos desta coleção optam pela tática do
ataque frontal, conjurando as falsas dicotomias que organizam a
estratégia da ordem. Livros curtos para serem levados no bolso, na
mochila ou na bolsa, como pedras ou coquetéis molotov.
Pensamento"-tática que anima o enfrentamento colado à urgência do
presente. Ao serem lançados, não se espera desses livros mais do que
efeitos de antipoder, como a beleza de exibições pirotécnicas. Não há
ordem, programa, receita ou estratégia a serem seguidos. Ao atacar
radicalmente a única esperança possível é que se perca o controle e,
como isso, dançar com o caos dentro de si. Que as leituras produzam
efeitos no seu corpo.}

\medskip

\textsc{acácio augusto \& renato rezende}
\end{adjustwidth}
\end{flushright}
\thispagestyle{empty}

\chapter*{Prefácio\\
\emph{No fogo dos combates}}
\addcontentsline{toc}{chapter}{No fogo dos combates, \footnotesize{\emph{por Edson Passetti}}}

\begin{flushright}
\textsc{edson passetti\footnote{Edson Passetti é professor da Pontifícia Universidade
Católica de São Paulo, e atualmente coordena o Nu"-Sol (Núcleo de Sociabilidade Libertária
\versal{PUC"-SP}).}}
\end{flushright}
\medskip

\noindent Camila Jourdan, jovem filósofa, vira e revira o acontecimento
\emph{2013}. Esteve e está dentro dos eventos. Passou pela prisão, pelo
tribunal e vive os fluxos de revoltas. Presença constante na \versal{UERJ} como
professora de filosofia e na \versal{FIP} (Frente Independente Popular) como
militante anarquista --- e em parceria com outros jovens iracundos ---,
permaneceu mãe, mulher apaixonada e contestadora afiada contra as forças
repressivas de dentro e de fora do Estado. Tudo isso é situado aos leitores
em dimensões que surpreendem mesmo aos bem inteirados, no
exaustivo levantamento de matérias e depoimentos encadeados inicialmente
para alimentarem as análises inventivas que ela realizou para esse livro
tanto de imediato como posteriormente aos eventos.

É salutar e empolgante encerrar a leitura seguindo uma análise de fôlego
sobre o acontecimento. É uma franca contribuição aos que lidam com o
insuportável, por uma perspectiva ontológica que conjuga filósofos
contemporâneos, jornalismo e humanidades, atravessando e demolindo as
paredes do conforto das autoridades hierárquicas e dos cidadãos
conformistas, dispostos para dentro e para fora da universidade e das
ruas.

O livro está composto em duas circulações: depoimentos diretos de prisão
e vida encarcerada com matérias produzidas de imediato, no fogo dos
combates, e acompanhados de análises em tópicos que atiçam as brasas
jamais dormidas. Um livro sobre o Rio de Janeiro e o Brasil insurgido,
com o fogo de jovens e nem tão jovens que expõem e realizam suas
revoltas. \emph{Junho de 2013} é um ponto de inflexão irreversível sobre
a história política, econômica, social, cultural e de costumes do Brasil
em brasa.

A polícia invade a casa de Camila e rapidamente ela já está na cela
branca da Polinter fedendo a merda, mas contando com a atávica
solidariedade entre as prisioneiras, que se estabelece com seus cantos,
gestos e palavras avessas à ordem. Depois, transferência para Bangu,
mais uma das várias prisões de segurança máxima do país, repleta de
encarceradas negras obrigadas a se subjugarem: às carcereiras brancas,
investidas de sua lógica autoritária, trajadas como paródia de burguesas ---
e à nuvem de mosquitos atormentando cada final da tarde. Na prisão,
somente pássaros e gatos circulam livremente, e sob as ameaças das
carcereiras toda prisioneira pode, a qualquer momento, ser destinada
para a ``tranca'' ou para o ``buraco''.

O movimento sabe que não é inaugural, mas instaurador, e está
sincronizado com outras ocorrências violentas sancionadas contra populações
pobres, a altos custos de segurança para o Estado; sabe que enfrentam também o
execrável exército de reserva de poder composto por miseráveis delatores
e infiltrados, recrutados pela polícia e pelo exército entre e contra o
povo. O movimento sabe que enfrentará a criminalização dos protestos e
que a punição aos 23 e à tática black bloc é simplesmente a
confirmação do terrorismo de Estado.

Camila Jourdan, com secura exata nas palavras, informa, situa, analisa,
espanta e mostra a clareza do insuportável da revolta, da luta
``insurrecionária''. Leva"-nos à sala do tribunal com seu crucifixo
centralizado e a figura do juiz que diz: ``Aqui quem manda sou eu, aqui
não tem punhos cerrados não, aqui não é a rua.'' E assim está
consagrada, mais uma vez, a chamada isenção e a neutralidade da justiça.
E assim, também, prossegue a sessão com base na delação do infiltrado e
de uma segunda testemunha que possibilita, às pessoas na sala,
estrondosas gargalhadas abrilhantando sua mediocridade. Mas pouco
importa qualquer objeção, pois a neutra justiça, de antemão, já sabe o
que fazer!

Ao mesmo tempo, outros eventos na cidade do Rio de Janeiro não cessam de
escancarar o insuportável da revolta. E novas acusações se avolumam, ante as quais
Camila mantém a tranquilidade de quem pratica liberdades e
desvencilha"-se dos propositais emaranhados sob a forma de ciladas. O
militantismo, esta prática que dispensa condutores pelo alto e
lideranças por baixo, produz relações horizontalizadas e
autogestionárias, fortalecendo cada um, ética e esteticamente. Isto
aparece na sua entrevista à \emph{Folha de S. Paulo}, mas também quando,
proibida pela justiça de dar uma palestra, Camila posta: ``Mais um
absurdo sem precedentes. Acabo de ser censurada. O juiz responsável
pelo caso indeferiu meu pedido de ir a Dourados"-\versal{MS}, dar uma
palestra no \versal{IV} Encontro de Integração: Dias de História, na \versal{UFGD}. A
justificativa do magistrado é que a atividade de dar palestras não é
essencial ao exercício da minha atividade profissional''. O juiz é o
sujeito que pretende ter a autoridade inquestionável, saber sobre tudo e
todos, e a exerce.

Na entrevista ao \emph{Le Monde Diplomatique}, Camila atinge
uma incisiva reflexão sobre o estupro, para além de considerações
pertinentes sobre as justificativas dos partidos de esquerda contra os
efeitos das \emph{jornadas de junho de 2013}. As soldadas conexões
eleitorais e a necessidade de criminalização do movimento pelas forças
político"-partidárias e pelo Estado reaparecem e são ampliadas em
questionamento à representação na entrevista ao \emph{Diário do Centro do
Mundo}, em 2016. Como anarquista, conclui com a sugestão aos eleitores,
nesta democracia do voto obrigatório, com um ``não vote''.

Esta densa preparação neste segmento do livro convida às análises
detalhadas que se seguem, promovendo reflexões acompanhadas de filósofos
e pesquisadores de diferentes procedências, e assim realçando a
importância da revolta. Camila Jourdan conversa com os anarquistas
Proudhon e Bakunin, mas também com Michel Foucault, Gilles Deleuze, Félix
Guattari, Maurizio Lazzarato, Eric Alliez, Hannah Arendt, Guy Debord,
Francis Depuis"-Déri, Albert Camus, Nietzsche e o Comitê Invisível.

Camila Jourdan traça a formosura da \emph{vida como obra de arte}, suas
intensidades, resistências, invenções, como uma prática de liberdade que
toca com força e leveza a vida, a vida principalmente libertária, a vida
dos destemidos e corajosos que escancara as dissimulações inconfessáveis
dos condutores do Estado e de suas arrogâncias ao ambicionarem governar
cada um na sociedade. Clarifica os meandros das acusações, como os
anarquistas (e não só eles) são construídos como inimigos da sociedade e
do Estado, e como os libertários permanecem alertas e atiçando por
liberdades outras.

A generosidade analítica de Camila Jourdan soma e ao mesmo tempo
sinaliza para como jamais esmorecer diante dos justos juramentados. Abre
e reabre as conversações sobre as diferenças nas análises entre os
iracundos e não dá descanso aos institucionalizados e suas reais
fantasias alinhavadas para manter o espetáculo das chamadas
desobediências. E que fique definitivamente claro: a desobediência
civil, desde Bill Clinton, se transmutou em política da ordem em nome da
não"-violência, obviamente destinada aos que se acostumaram a obedecer.

Outro certo juiz, em 17 de julho de 2018, condena a 7 anos de prisão os
20 adultos processados, e a 5 anos e 10 meses os outros três que nos eventos de 2013"-2014
eram menores de idade. \emph{2013} é um acontecimento que não tem data
e hora para acabar.
